\documentclass[article,type=msc,colorback,accentcolor=tud2a]{tudthesis}
%\documentclass[article,type=msc,colorback,accentcolor=tud2a]{tudreport}

%%%%%%%%%%%%%%%%%%%%%%%%%%%%%%%%%%%%%%%%%%%%%%%%%%%%%%%%%%%%%%%%%%%%%%%
%\usepackage{ngerman}

%%%%%%%%%%%%%%%%%%%%%%%%%%%%%%%%%%%%%%%%%%%%%%%%%%%%%%%%%%%%%%%%%%%%%%%

\newcommand{\getmydate}{%
  \ifcase\month%
    \or Januar\or Februar\or M\"arz%
    \or April\or Mai\or Juni\or Juli%
    \or August\or September\or Oktober%
    \or November\or Dezember%
  \fi\ \number\year%
}

\usepackage{masterthesis}

\begin{document}
\thesistitle{Implementation and Performance Analyses of a Highly Efficient Algorithm for Pressure-Velocity Coupling}{Implementierung und Untersuchung einer hoch effizienten Methode zur Druck-Geschwindigkeits-Kopplung}
%\title{Implementation and Performance Analyses of a Highly Efficient Algorithm for Pressure-Velocity Coupling}{Implementierung und Untersuchung einer hoch effizienten Methode zur Druck-Geschwindigkeits-Kopplung}
  \author{Fabian Nuraddin Alexander Gabel}
  \referee{Prof. Dr. rer. nat. Michael Schäfer}{Dipl.-Ing Ulrich Falk}
  \department{Studienbereich CE}
  \group{FNB}
% \tuprints{12345}{1234}
  \makethesistitle
% \maketitle
  \affidavit{F. Gabel}
  \tableofcontents

    %GEOMETRY
    \nomenclature{$x_i, i \in \{1,2,3\}$}{Cartesian coordinates}
    \nomenclature{$\vec{x}$}{Coordinate Vector}
    \nomenclature{$n_i, i \in \{1,2,3\}$}{Surface normal unit vector components}
    \nomenclature{$\vec{n}$}{Surface normal unit vector}
    \nomenclature{$t$}{Time}
    \nomenclature{$V$}{Volume}
    \nomenclature{$S$}{Surface}
    \nomenclature{$\left( \vec{e}_i \right)_{i=1,\dots,3}$}{Cartesian canonical basis}
    %PHYSICS
    \nomenclature{$\rho$}{Density}
    \nomenclature{$\mu$}{Dynamic viscosity}
    %\nomenclature{$\lambda$}{
    \nomenclature{$p$}{Pressure}
    \nomenclature{$u_i, i \in \{1,2,3\} $}{Cartesian velocity components}
    \nomenclature{$\vec{u}$}{Velocity vector}
    \nomenclature{$t_i, i \in \{1,2,3\}$}{Stress vector components}
    \nomenclature{$\vec{t}$}{Stress vector}
    \nomenclature{$k_i, i \in \{1,2,3\}$}{Mass specific force vector components}
    \nomenclature{$\vec{k}$}{Mass specific force vector}
    \nomenclature{$S_{ij}, i,j \in \{1,2,3\}$}{Symmetric part of the transpose of the jacobian of the velocity}
    \nomenclature{$\sigma_{ij}, i,j \in \{1,2,3\}$}{Deviatoric stress tensor}
    \nomenclature{$\tau_{ij}, i,j \in \{1,2,3\}$}{Coefficient matrix of stress mapping \(T\)}
    \nomenclature{$\vec{T}$}{Linear stress mapping}
    \nomenclature{$T$}{Temperature}
    \nomenclature{$\kappa$}{Thermal conductivity}
    \nomenclature{$q_T$}{Source or sink of heat}
    \nomenclature{$\rho_0$}{Reference density at \(T_0\)}
    \nomenclature{$T_0$}{Reference temperature}
    \nomenclature{$g_i, i \in \{1,2,3\}$}{Components of the gravitational acceleration vector}
    \nomenclature{$\vec{g}$}{Gravitational acceleration vector}
    \nomenclature{$\hat{p}$}{Pressure without hydrostatic pressure part}
    \nomenclature{$\beta$}{Coefficient of thermal expansion}
    %MATHEMATICS
    \nomenclature{$\delta_{ij}$}{Kronecker-Delta}
    %DIMENSIONLESS QUANTITIES
    \nomenclature{$Ma$}{Mach number}
    \printnomenclature



  \section{Introduction}

  This thesis is about. 

  \section{Fundamentals of Continuum Physics for Thermo-Hydrodynamical Problems}
\label{sec:fundamentals}

This section covers the set of fundamental equations for thermo-hydrodynamical problems which the numerical solution techniques of the following chapters are aiming to solve. Furthermore, the notation regarding the physical quantities to be used throughout this thesis is introduced. The following paragraphs are based on \cite{andersson84,ferziger02,kundu12,spurk10}. For a thorough derivation of the matter to be presented the reader may consult the mentioned sources. Since the present thesis focuses on the application of finite-volume methods, the focus lays on stating the integral forms of the relevant conservation laws. However, the process of deriving the final set of equations requires the use of differential formulations of the stated laws. Einstein's convention for taking sums over repeated indices simplifies certain expressions. The remainder of this thesis uses non-moving inertial frames in a Cartesian coordinate system with the coordinates \( x_i, i=1,...,3 \). This approach is also known as \emph{Eulerian approach}.  

\subsection{Conservation of Mass -- Continuity Equation}

The conservation law of mass, also known as the continuity equation, embraces the physical concept that, neglecting relativistic effects and nuclear reactions, mass cannot be created or destroyed. Using the notion of a mathematical control volume to denote a constant domain of integration, one can state the integral mass balance of a control volume \(V\) with control surface \(S\) with surface unit normal vector \(\vec{n} = \left( n_i \right)_{i=1,\dots,3}\) using Gauss's theorem as
\begin{displaymath}
  \iiint\limits_V \frac{\partial \rho}{\partial t} + \frac{\partial}{\partial x_i}\left( \rho u_i \right) \mathrm{d}V 
  =  \iiint\limits_V \frac{\partial \rho}{\partial t} \mathrm{d}V + \iint\limits_S \rho u_i n_i \mathrm{d}S
  = 0,
\end{displaymath}
where \( \rho \) denotes the material density, \(t\) denotes the independent variable of time and \(\vec{u} = \left( u_i \right)_{i=1,\dots,3}\) is the velocity vector field. Since this equation remains valid for arbitrary control volumes, the equality has to hold for the integrands as well. In this sense, the differential form of the conservation law of mass can be formulated as
\begin{equation}
  \label{eq:contifull}
  \frac{\partial \rho}{\partial t} + \frac{\partial}{\partial x_i}\left( \rho u_i \right)
  = 0.
\end{equation}

\subsection{Conservation of Momentum -- Cauchy-Equations}

The conservation law of momentum, also known as Newton's Second Law, axiomatically demands the balance of the temporal change of momentum and the sum of all attacking forces on a body. These forces can be divided into body forces and surface forces. Let \(\vec{k} = \left( k_i \right)_{i=1,\dots,3}\) denote a mass specific force and \(\vec{t} = \left(t_i\right)_{i=1,\dots,3}\) the stress vector. A first form of the integral momentum balance in the direction of \(x_i\) can be formulated as
\begin{equation}
\label{eq:cauchy}
\iiint\limits_V \frac{\partial }{\partial t}\left(\rho u_i \right) \mathrm{d}V + \iint\limits_S \rho u_i \left( u_j n_j \right) \mathrm{d}S = \iiint\limits_V \rho k_i \mathrm{d}V + \iint\limits_S t_i \mathrm{d}S.
\end{equation}

In general, the stress vector \(\vec{t}\) is a function not only of the location \(\vec{x} = \left( x_i \right)_{i = 1,\dots,3}\) and of the time \(t\) but also of the surface unit normal vector \(\vec{n}\). A central simplification can be introduced, namely Cauchy's stress theorem, which states that the stress vector is the image of the unit normal vector under a linear mapping \(\vec{T}\). With respect to the Cartesian canonical basis \(\left(\vec{e}_i \right)_{i = 1, \dots, 3}\), the mapping \(\vec{T}\) is represented by the coefficient matrix \( \left(\tau_{ji}\right)_{i,j = 1,\dots,3}\), and Cauchy's stress theorem reads
\begin{displaymath}
  \vec{t}\left(\vec{x},t,\vec{n}\right) = \vec{T}(\vec{x},t,\vec{n}) = \left(\tau_{ji} n_j\right)_{i = 1, \dots, 3}.
\end{displaymath}
Assuming the validity of Cauchy's stress theorem, one can derive Cauchy's first law of motion, which in differential form can be formulated as
\begin{equation}
  \label{eq:cauchymotion}
  \frac{\partial }{\partial t} \left(\rho u_i \right)
  + \frac{\partial}{\partial x_j}\left( \rho u_i u_j \right) 
  = \rho k_i + \frac{\partial \tau_{ji}}{\partial x_j}
\end{equation}
representing the starting point for the modeling of fluid mechanical problems. One should note that Cauchy's first law of motion does not make any assumptions regarding material properties, which is why the set of equations (\ref{eq:contifull},\ref{eq:cauchymotion}) is not closed, meaning that there does not exist an independent equation for each of the dependent variables.

%\subsection{Conservation of Angular Momentum}
\subsection{Closing the System of Equations -- Newtonian Fluids}
\label{sec:fundclosing}

As a result of Cauchy's theorem, the stress vector \( \vec{t} \) can be specified once the nine components \(\tau_{ji}\) of the coefficient matrix are known. As shown in \cite{kundu12,spurk10}, by formulating the conservation law of angular momentum the coefficient matrix is symmetric, 
\begin{equation}
  \label{eq:stresssymetry}
  \tau_{ji} = \tau_{ij},
\end{equation}
hence the number of unknown coefficients may be reduced to six unknown components. In a first step, it is assumed that the coefficient matrix can be decomposed into fluid-static and fluid-dynamic contributions,
\begin{displaymath}
  \tau_{ij} = -p \delta_{ij} + \sigma_{ij},
\end{displaymath}
where \(p\) is the thermodynamic pressure, \(\delta_{ij}\) is the \emph{Kronecker}-Delta and \( \sigma_{ij} \) is the so called \emph{deviatoric stress tensor}. 
    
In the present thesis, viscous fluids are modeled using a linear relation between the components of the deviatoric stress tensor and the symmetric part of the transpose of the Jacobian of the velocity field \(\left( S_{ij} \right)_{i,j=1,\dots,3}\),
\begin{displaymath}
  S_{ij} = \frac{1}{2} \left( \frac{\partial u_i}{\partial x_j} + \frac{\partial u_j}{\partial x_i} \right).
\end{displaymath}

If one now imposes material-isotropy and the mentioned stress-symmetry (\ref{eq:stresssymetry}) restriction, it can be shown \cite{aris62} that the constitutive equation for the deviatoric stress tensor reads 
\begin{displaymath}
  \sigma_{ij} = 2 \mu S_{ij} + \lambda S_{mm} \delta_{ij},
\end{displaymath}
where \(\lambda\) and \(\mu\) denominate scalars which depend on the local thermodynamical state. Taking everything into account, (\ref{eq:cauchymotion}) can be formulated as the differential conservation law of momentum for Newtonian fluids, better known as the \emph{Navier-Stokes equations} in differential form:
\begin{equation}
\label{eq:nsfull}
\frac{\partial }{\partial t} \left(\rho u_i \right)
+ \frac{\partial}{\partial x_j} \left( \rho u_i  u_j \right) 
= \rho k_i
- \frac{\partial p}{\partial x_i}
+ \frac{\partial}{\partial x_j} \left( \mu  \left( \frac{\partial u_i}{\partial x_j} 
                                        + \frac{\partial u_j}{\partial x_i} \right) \right)
+ \frac{\partial}{\partial x_i} \left(\lambda \frac{\partial u_m}{\partial x_m} \right)
\end{equation}

\subsection{Conservation of Scalar Quantities}

The modeling of the transport of scalar or vector quantities by a flow field \(\vec{u}\), also known as convection, is necessary if the fluid mechanical problem to be analyzed includes, for example, heat transfer. Other scenarios that involve the necessity to model scalar transport surge, when turbulent flows are to be modelled by two-equation models like the \(k\)-\(\varepsilon\)-model \cite{pope00}. 
    
Since this thesis focuses on the transport of the scalar temperature \(T\), this section introduces the conservation law of energy in differential form, formulated in terms of the temperature,
\begin{displaymath}
  \frac{\partial \left(\rho T \right)}{\partial t} + \frac{\partial}{x_j} \left( \rho u_j - \kappa \frac{\partial T}{\partial x_j} \right) = q_T,
\end{displaymath}
where \(\kappa\) denotes the thermal conductivity of the modelled material and \(q_T\) is a scalar field representing sources and sinks of heat throughout the domain of the problem. This equation is also known as the temperature equation.

        %Check also Peric p12 or Bird et al. (1962).

\subsection{Necessary Simplification of Equations}

The purpose of this section is to introduce and justify further common simplifications of the previously presented set of constitutive equations. 

\subsubsection{Incompressible Flows and Hydrostatic Pressure}

A common simplification when modeling low Mach number flows (\(Ma < 0.3\)), is the assumption of \emph{incompressibility}, or the assumption of an \emph{isochoric} flow. If one furthermore assumes homogeneous density \(\rho\) in space and time, a restrictive assumption that will be partially alleviated in the following section, the continuity equation in differential form (\ref{eq:contifull}) can be simplified to
\begin{displaymath}
  %\label{eq:contiinc}
  \frac{\partial u_i}{\partial x_i} = 0.
\end{displaymath}
In other words: In order for a velocity vector field \(\vec{u}\) to be valid for an incompressible flow, it has to be free of divergence, in other terms \emph{solenoidal} \cite{spurk10,aris62}.

If furthermore one assumes the dynamic viscosity \(\mu\) to be constant,  which can be suitable in the case of isothermal flow or if the temperature differences within the flow are small, the Navier-Stokes equations in differential form can be reduced to 
\begin{subequations}
\label{eq:navierstokes}
\begin{align}
  \frac{\partial}{\partial t}   \left(\rho u_i \right)
  + \rho \frac{\partial}{\partial x_j} \left( u_i  u_j \right) 
  =& \rho k_i
  - \frac{\partial p}{\partial x_i}
+ \frac{\partial}{\partial x_j} \left( \mu  \left( \frac{\partial u_i}{\partial x_j} 
+ \frac{\partial u_j}{\partial x_i} \right) \right) \\[0.5em]
  =& \rho k_i
  - \frac{\partial p}{\partial x_i}
  + \mu \frac{\partial}{\partial x_j} \left( \frac{\partial u_i}{\partial x_j} \right)
\end{align}
\end{subequations}
by using \emph{Schwartz}'s lemma to interchange the order of differentiation. Another common step to further simplify the set of equations is the assumption of a volume specific force \(\rho \vec{k}\) that can be modelled by a potential, in such a way that it can be represented as the gradient of a scalar field \(\Phi_\vec{k}\) as
\begin{displaymath}
 - \rho k_i = \frac{\partial \Phi_\vec{k}}{\partial x_i}.
\end{displaymath}

In the case of this thesis, this assumption is valid, since the mass specific force is the mass specific gravitational force \(\vec{g} = \left( g_i \right)_{i = 1,\dots,3}\), and the density is assumed to be constant, so that the potential can be modeled as
\begin{displaymath}
  \Phi_g = - \rho g_j x_j.
\end{displaymath}
This term can be interpreted as the hydrostatic pressure \(p_{hyd}\) and can be added to the thermodynamical pressure \(p\) to simplify calculations 
\begin{align*}
  \rho g_i - \frac{\partial p}{\partial x_i} 
  =& \frac{\partial}{\partial x_i} \left( \rho g_j x_j \right) - \frac{\partial p}{\partial x_i} \nonumber \\[0.5em]
  =& \frac{\partial}{\partial x_i} \left( \rho g_j x_j \right) - \frac{\partial}{\partial x_i}  \left(\hat{p} + p_{hyd} \right) \nonumber \\[0.5em]
  =& - \frac{\partial \hat{p}}{\partial x_i}.
\end{align*}
Since in incompressible fluids only pressure differences matter, this has no effect on the solution. After finishing the calculations \(p_{hyd}\) can be calculated and added to the resulting pressure \(\hat{p}\).

\subsubsection{Variation of Fluid Properties -- The Boussinesq Approximation}
\label{sec:boussinesq}

Modeling of an incompressible flow with heat transfer has to take into account that fluid properties, as for example density, change with varying temperature. If the variation of temperature is small, one can still assume constant density to maintain the structure of the advection and diffusion terms in (\ref{eq:nsfull}), and only consider changes of the density in the gravitational term. If linear variation of density with respect to temperature is assumed, this approximation is called \emph{Boussinesq} approximation \cite{gray76}. This approximation furthermore consists in assuming that all other fluid properties are constant and viscous dissipation can be neglected. In this case, the Navier-Stokes equations are formulated using a reference density \(\rho_0\) at the reference temperature \(T_0\) and the following temperature-dependent density \(\rho\)
\begin{displaymath}
  \rho \left( T \right) = \rho_0 \left( 1 - \beta \left( T - T_0 \right) \right).
\end{displaymath}
Here, \(\beta\) denotes the coefficient of thermal expansion. Using the Boussinesq approximation, the incompressible Navier-Stokes equations in differential form can be formulated as
\begin{align*}
  \rho_0 \frac{\partial \left( u_i \right)}{\partial t} 
  + \rho_0 \frac{\partial}{\partial x_j} \left( u_i  u_j \right) 
  =& \rho_0 g_i + \left(\rho - \rho_0 \right) g_i
  - \frac{\partial p}{\partial x_i}
  +  \mu \frac{\partial}{\partial x_j} \left( \frac{\partial u_i}{\partial x_j} 
  + \frac{\partial u_j}{\partial x_i} \right)  \\[0.5em]
  =& \frac{\partial}{x_i}\left(\rho_0 g_j x_j \right) 
  + \left(\rho - \rho_0 \right) g_i
  - \frac{\partial p}{\partial x_i}
  +  \mu \frac{\partial}{\partial x_j} \left( \frac{\partial u_i}{\partial x_j} 
  + \frac{\partial u_j}{\partial x_i} \right)  \\[0.5em]
  =& - \frac{\partial \hat{p}}{\partial x_i} 
  + \left(\rho - \rho_0 \right) g_i
  +  \mu \frac{\partial}{\partial x_j} \left( \frac{\partial u_i}{\partial x_j} 
  + \frac{\partial u_j}{\partial x_i} \right)  \\[0.5em]
  =& - \frac{\partial \hat{p}}{\partial x_i} 
  - \rho_0 \beta \left( T - T_0 \right) \, g_i
  +  \mu \frac{\partial}{\partial x_j} \left( \frac{\partial u_i}{\partial x_j} 
                                       + \frac{\partial u_j}{\partial x_i} \right) 
\end{align*}
using \(\rho \vec{g}\) as the mass specific force. 

\subsection{Final Form of the Set of Equations}

In the previous subsections different simplifications have been introduced, which will be used throughout the thesis. The final form of the set of equations to be used is thereby presented. As a further simplification, the modified pressure \(\hat{p}\) will be treated as \(p\), and since the use of the Boussinesq approximation replaces the variable \(\rho\) by a linear function of the temperature \(T\), the reference density \(\rho_0\) for the remainder of this thesis will be referred to as \(\rho\). It should be noted that incompressibility has been taken into account
\begin{subequations}
\label{eq:completeset}
\begin{align}
\label{eq:contidiff}
\frac{\partial u_i}{\partial x_i} =& 0. \\[1em]
\label{eq:momentumdiff}
\rho \frac{\partial \left( u_i \right)}{\partial t} 
+ \rho \frac{\partial}{\partial x_j} \left( u_i  u_j \right) 
=& - \frac{\partial p}{\partial x_i} 
- \rho \beta \left( T - T_0 \right)\, g_i
+  \mu \frac{\partial}{\partial x_j} \left( \frac{\partial u_i}{\partial x_j} 
+ \frac{\partial u_j}{\partial x_i} \right) \\[1em]
\label{eq:temperaturediff}
\frac{\partial \left(\rho T \right)}{\partial t} + \frac{\partial}{x_j} \left( \rho u_j T - \kappa \frac{\partial T}{\partial x_j} \right) =& q_T.
\end{align}
\end{subequations}


  \section{Finite Volume Method for Incompressible Flows -- Theoretical Basics and Realisation in Code}

    \subsection{Fundamentals of Discretization}
      
      \subsubsection{Numerical Grid}
      \subsubsection{Approximation of Integrals}

    \subsection{Discretization of the Momentum Balance}
      
      \subsubsection{Semi Discretized Linearized Form of the Navier-Stokes Equations}
      \subsubsection{Treatment of Non-Orthogonality of Grid Cells -- Deferred Correction Approach}
        Cite Jsak and make some pretty pictures. Motivate each technique for non-orthogonal correction.
      \subsubsection{Calculation of Mass Flux -- Rhie-Chow Interpolation}
      \subsubsection{Discretization of the Convective Term}
      \subsubsection{Discretization of the Diffusive Term}
      \subsubsection{Discretization of the Source Term}
      \subsubsection{Assembly of Linear Systems -- Final Form of Equations}
        Coefficients of matrices for momentum are identical except in case of different factors for under-relaxation (underrelaxation (Andersson) )(when does this happen) for the main diagonal coefficient. Small example in code, then show image of assembled system.

    \subsection{Discretization of the Generic Transport Equation}

    \subsection{Segregated Methods -- the SIMPLE-Algorithm}
      
      \subsubsection{Pressure Correction Equation}
      \subsubsection{Characteristic Properties of Projection Methods}

        Under-relaxation, slow convergence, inner iterations outer iterations, relative tolerances, also talk about staggered and collocated variable positioning

      \subsubsection{Coupling of Temperature Equation}

    \subsection{Boundary Conditions on Domain and Block Boundaries}
        Introduce chapter by talking about the nature of partial differential equations (Hackbusch). Always start with a simple implementation for the generic transport equation, then specialize to Navier-Stokes equation.
      \subsubsection{Dirichlet Boundary Condition}
        Only talk about dirichlet for velocities not for pressure.
      \subsubsection{Neumann Boundary Condition}
        Problematics of outlet boundary conditions
      \subsubsection{Symmetry Boundary Condition}
      \subsubsection{Wall Boundary Condition}
      \subsubsection{Block Boundary Condition}
      
    \subsection{Coupled Solution of the Navier-Stokes Equations}

      \subsubsection{Discretization of the Navier-Stokes Equations}
      \subsubsection{Differences to the Segregated Approach -- Implicit Coupling of Velocities, Pressure and Temperature}

          \begin{itemize}
            \item Implicit treatment of Pressure Gradient
            \item Implicit Treatment of Temperature possible
            \item Boussinesq approximation brings velocity-to-temperature-coupling (vakilipour), Newton-Raphson Linearization
            \item Temperature dependent densities also possible
          \end{itemize}

      \subsubsection{Assembly of Linear System}

      \subsubsection{Boundary Conditions}

      \begin{itemize}
        \item Dirichlet Velocities (implies Neumann for Pressure
        \item Dirichlet Pressure (implies Neumann for Velocities
        \item Symmetry and Outled Boundary Condition
        \item Wall Boundary Condition
      \end{itemize}

    \subsection{Characteristic Properties of the Fully Coupled Solution Approach}

        Bad condition, singularity, usually faster convergence if efficient linear solver is chosen, coupling in Buoyancy flows (s.a. Peric page 448, Galpin Raithby)
        Design of algorithm does not need to inforce continuity (is inherently fullfilled because of the coupling of the equations)

    \subsection{Numerical Solution of Linear Systems}

        \subsubsection{Stone's SIP Solver}

          Basic Idea as in Schäfer or Peric

        \subsubsection{Krylov Subspace Methods}

          \begin{itemize}
            \item General concept of cyclic vector spaces of \(\mathbb{R}^n\), 
            \item talk about bases of krylov subspaces and the arnoldi algorithm, talk about polynomials and linear combinations
            \item mention the two major branches (minimum residual approach, petrov and ritz-galerkin approach) 
            \item name some representative ksp algorithms, importance of preconditioning, not as detailed as in bachelor thesis
            \item in cases there is a nonempty Nullspace what happens?
          \end{itemize}

  \section{CAFFA Framework}

    \subsection{PETSc Framework}
        Keep in mind not to copy the manual
      \subsubsection{About PETSc}

        Bell Prize, MPI Programming

      \subsubsection{Basic Data Types}

        Vec,Mat (Different Matrix Types and Their effect on complex methods)

      \subsubsection{KSP and PC Objects and Their Usage}

        Singularities

      \subsubsection{Profiling}

        PETSc Log 

      \subsubsection{Common Errors}

      Optimization, Interfaces, (ROWMAJOR,COLUMNMAJOR), Compiler Errors not helpful, Preallocation vs. Mallocs

    \subsection{Grid Generation and Conversion}

      Generation of block structured locally refined grids with nonmatching block interfaces, neighbouring relations are represented by a special type of boundary conditions; Random number generator to move grid points within a epsilon neighbourhood while maintaining the grid intact. Show in a graph how preallocation impacts on runtime.
    \subsection{Preprocessing}
    Matching algorithm -- the idea behind clipper and the used projection technique; alt.: Opencascade. Efficient calculation of values for discretization. Important for dynamic mesh refinement, arbitrary polygon matching
    \subsection{Implementation Details of CAFFA}

      \subsubsection{MPI Programming Model}
        Basic idea of distributed memory programming model, emphasize the differences to shared memory model. Have a diagram at hand that shows how CAFFA sequentially works (schedule) and point out the locations where and of which type (global reduce, etc.) communication is, or when synchronization is necessary.
        \begin{itemize}
          \item after each solve
          \item pressure reference
          \item error calculation
          \item gradient calculation
        \end{itemize}
        
        Point out that one should try to minimize the number of this points such that parallel perfomance stays high. Better to calculate Velocity and Pressure Gradients at once not by seperately calling this routine.

        \subsubsection{Convergence Control} 
        Explain how the criterion for convergence is met 

        \subsubsection{Modi of Calculation}
          there are different modi of calculation, (NS segregated, then scalar; NS and Scalar Segregated; NS coupled and Scalar segregated; Fully coupled (wath out with fully coupled, this term seems to have already another meaning)). Note that for comparison of solvers it is crucial to develop programs on the same basis. This establishes comparability.

      \subsubsection{Indexing of Variables and Treatment of Boundary Values}
      Describe MatZeroValues and how it is used to simplify the code. Also loose a word on PCREDISTRIBUTE its advantages and downsides. Compliance of PETSc zero based indexing and CAFFA indexing which considers boundary values. Problems with boundary entries
      \subsubsection{Field Interlacing}
      Realization through special arrangement of variables and the use of index sets (subvector objects) and/or preprocessor directives. Advantages (there was a paper I cited in my thesis). Note that not all variables are interlaced (Velocities are, but their gradients are not). Great impact on Matrix structure.
      \subsubsection{Domain Decomposition, Exchange of Ghost Values and Parallel Matrix Assembly}

      \begin{itemize}
        \item Ghost values are stored in local representations of the global vector (state the mapping for those entries). 
        \item Matrix coefficients are calculated on one processor and sent to the neighbour. 
        \item Preallocation as crucial aspect for program performance. For the coupled system the matrix is assembled in a 2-3 step process to save memory for coefficients. 
        \item Present a simple method for balancing the matrix related load by letting PETSc take care of matrix distribution. 
        \item Use Spy function of Matlab to visualize the sparse matrices. Point out advantages of calculating coefficients for the neighbouring cells locally (no need to update mass fluxes, geometric data doesn't need to be shared, small communication overhead since processors assemble matrix parts that don't belong to them (visualize)). 
        \item Paradigm: Each time new information is available perform global updates. Advantages of using matrices: Show structure of matrix when using arbitrary matching vs. higher memory requirements vs. better convergence
      \end{itemize}

    \subsection{Postprocessing}
    
      Visualization of Results with Paraview and Tecplot
      Export matrices as binaries and visualize them using matlab scripts.

  \section{Verification of CAFFA}
    
    Different parts, describe incremental approach, only present final results.
    Refer to next section for Validation of CAFFA

    \subsection{Theoretical Discretization Error}
      present the Taylor-Series Expansion

    \subsection{Method of Manufactured Solutions}
      basically sum up the important points of salari's technical report, symmetry of solution/domain/grid is bad
      point out that mms is not able to detect errors in the physical model
      Also loose a word or two about discontinuous manufactured solutions

    \subsection{Exact and Manufactured Solutions for the Navier-Stokes Equations and the Energy Equation}
    Not always there is an exact solution. Divergence free approach. Presentation of the used manufactured solution. What if solution is not divergence free? Derivation of equations and modifications to continuity equation. analyze the problem of too complicated manufactured solutions. also use temperature dependent density function
    \begin{itemize}
      \item \url{http://scicomp.stackexchange.com/questions/6943/manufactured-solutions-for-incompressible-navier-stokes-how-to-find-divergenc}
      \item \url{http://link.springer.com/article/10.1007/BF00948290}
      \item \url{http://physics.stackexchange.com/questions/60476/exact-solutions-to-the-navier-stokes-equations}
      \item \url{http://www.annualreviews.org/doi/pdf/10.1146/annurev.fl.23.010191.001111}
    \end{itemize}
    \subsection{Measurement of Error and Calculation of Order}
      Different error measures (L2-Norm,completeness of function space, consistency etc.)

      \subsubsection{Testcase on Single Processor on Orthogonal Grid}
      \subsubsection{Testcase on Multiple Processors on Non-Orthogonal Grid}

        Give a measure of the grid quality.

  \section{Comparison of Solver Concepts}
  
    \subsection{Convergence Behaviour on Locally Refined Block Structured Grids}

      Show how the implicit treatment of block boundaries maintains (high) convergence rates. Plot Residual over number of iterations.

    \subsection{Parallel Performance}
      \subsubsection{Employed Hardware and Software -- The Lichtenberg-High Performance Computer }
        \begin{itemize}
          \item Networking
          \item Mem Section and processes in between islands (calculating across islands)
          \item Versioning information (PETSc,INTEL COMPILERS,CLIPPER,MPI IMPLEMENTATION,BLAS/LAPACK)
          \item Software not designed to perform well on desktop PCs.
        \end{itemize}

      \subsubsection{Measures of Performance}
        \begin{itemize}
          \item Maße definieren
          \item Nochmal Hager,Wellein studieren
          \item Guidelines for measuring performance (bias through system processes or user interaction), only measure calculation time do not consider I/O in the beginning and the end
          \item Cite Schäfer and Peric with their different indicators for parallel efficiency, load balancing and numerical efficiency
        \end{itemize}
      \subsubsection{Preliminary Upper Bounds on Performance -- The STREAM Benchmark}
        Pinning of processes (picture), preliminary constraints by hardware and operating systems, identification of bottlenecks and explain possible workarounds, history and results of STREAM. Bandwidth as Bottleneck, how to calculate a Speedup estimate based on the measured bandwidth. PETSc Implementation of STREAM

      \subsubsection{Discussion of Results for Parallel Efficiency}
      \subsubsection{Speedup Measurement for Analytic Test Cases}

    \subsection{Test Cases with Varying Degree of Non-Linearity}
      
      As Peric says I want to prove that the higher the non-linearity of NS, the better relative convergence rates can be achieved with a coupled solver. Fi

      \subsubsection{Transport of a Passive Scalar -- Forced Convection}
      \subsubsection{Buoyancy Driven Flow -- Natural Convection}
      \subsubsection{Flow with Temperature Dependent Density -- A Highly Non-Linear Test Case}
        Maybe I could consider two test cases, one with oscillating density and one with a quadratic polynomial. Interesting would be also to consider the dependence of convergence on another scalar transport equation

    \subsection{Realistic Testing Scenarios -- Benchmarking}
        Also consider simple load balancing by distributing matrix rows equally
      
      \subsubsection{Flow Around a Cylinder 3D -- Stationary}
        Describe Testing Setup (Boundary conditions and grid). Present results and compare them with literature.
      \subsubsection{Flow Around a Cylinder 3D -- Instationary}
        \begin{itemize}
          \item\url{http://www.featflow.de/en/benchmarks/cfdbenchmarking/flow/dfg_flow3d/dfg_flow3d_configuration.html}
        \end{itemize}
        Describe Testing Setup (Boundary conditions and grid). Present results and compare them with literature.

      \subsubsection{Heat-Driven Cavity Flow}
        \begin{itemize}
          \item \url{http://www.featflow.de/en/benchmarks/cfdbenchmarking/mit_benchmark.html}
        \end{itemize}
        Describe Testing Setup (Boundary conditions and grid). Present results and compare them with literature.
    \subsection{Realistic Testing Scenario -- Complex Geometry}
        
  \section{Conclusion and Outlook}
  Turbulence (turbulent viscosity has to be updated in each iteration), Multiphase (what about discontinuities), GPU-Accelerators, Load-Balancing, dynamic mesh refinement, Counjugate Heat Transfer with other requirements for the numerical grid, grid movement, list some papers here)
    Identify the optimal regimes / conditions for maximizing performance. Each solver concept has its strengths and weaknesses.
    Try other variants of Projection Methods like SIMPLEC, SIMPLER, PISO or PIMPLE (OpenFOAM)

%testing purposes
    \nocite{*}

\clearpage
\addcontentsline{toc}{section}{References}
\bibliography{bib/masterthesis.articles}{}
\bibliographystyle{acm}
\end{document}
