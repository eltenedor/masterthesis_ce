\section{Introduction}

The growing involvement of numerical methods in sciences and industrial practice due to the advancements in computer architecture and algorithms has had a great impact on academic research and industrial development processes in the past decades. Numerical methods nowadays form an integral part in forming a better understanding of physical phenomena and modelling the behaviour of technical systems besides purely theoretical or experimental approaches. Especially numerical methods for the simulation of fluid flow, also known as \emph{CFD} (\emph{Computational Fluid Dynamics}), have continuously been creating new challenges for the ongoing process of hardware and software improvements. Not only more complexity has to be reflected in computational methods to solve fluid flow problems but also less time to generate results is preferable to accelerate development processes. Parallel computing on high performance computer clusters represents an additional field of study since many numerical methods rely on hardware requirements that cannot be met by office workstations. As are the fields of study, the possibilities to further increase performance of scientific applications are manifold. Advancements in numerical methods and mathematical models, programming paradigms and tailored libraries for new computer architectures that use the hardware to full capacity and the new hardware itself contribute to the continuous expansion of the boundaries of what is \emph{computable}.

The incompressible Navier-Stokes equations represent the commonly used mathematical model for incompressible flows. Among other properties this model incorporates a coupling between the involved variables velocity and pressure. Algorithms that tend to solve for incompressible flow fields thus have to address the so called pressure-velocity coupling. A common approach is to resolve this coupling through segregated solution methods. With this methods the algebraic equations for each of the dependent variables are solved sequentially, using approximate values for the other involved variables. Achieving inter variable coupling is part of an iterative solution process such that in the end the solution field will fulfill all involved equations. One common representative of this type of algorithms is the SIMPLE algorithm that has been widely used in scientific codes or industrial solvers due to its small memory requirements. Reference \cite{acharya07} reviews the evolution of this type of algorithms.

The use of segregated solution algorithms comes with significant drawbacks. To stabilize the iteration for the coupling process, the obtained solutions have to be under-relaxed, where the amount of under-relaxation is furthermore dependent on the problem to be solved. Not only does this degrade the solver performance, but also there don't exist strict rules for the optimal choosing of the under-relaxation. The resolution of the pressure-velocity coupling forms an essential part for the performance of a flow solver and hence creates the need for new coupling algorithms that are more robust or even independent of parameters and at the same time resource efficient in their application. An analysis of the pressure-velocity coupling can be found in \cite{peric90}.

One alternative approach to pressure-velocity coupling is by means of the simultaneous solution of the momentum and mass balance equations. Algorithms of this type that work on co-located grids have been used in \cite{chen10,darwish09,falk13,galpin86,klaij13,mangani14,vakilipour12}. It was shown that the principal advantages of coupled solution approaches complement the drawbacks of segregated solution algorithms. Not only the computational time was reduced, also the robustness of the solution algorithm was increased. According to \cite{darwish09} no under-relaxation at all was necessary to obtain a robust implementation. 

So far only \cite{galpin86,vakilipour12} additionally coupled a temperature equation to the two-dimensional Navier-Stokes equations and investigated the impact of different methods to achieve implicit temperature-to-velocity/pressure coupling on the solver performance. The objectives of the present thesis are to implement a fully coupled solution algorithm for the three-dimensional Navier-Stokes equations and the temperature equation and analyse their single and multi-processor performance.

The thesis begins by introducing the mathematical and physical fundamentals of flow problems. In this initial section the main frame of partial differential equations to be solved numerically is introduced. Each subsection presents a brief derivation from the continuum mechanical point of view and shows common or necessary simplifications. This embraces not only a presentation of the Navier-Stokes equations for incompressible fluids but also the Boussinesq approximation that is widely used for incompressible flows that experience buoyant forces.

Section number three outlines the fundamentals of finite-volume methods in general. At first the necessary terminology for numerical grids is established. Then the main approximation principles for the discretization of integral or differential operators are introduced. Since in the present work no further requirements are demanded other than grid structure, one subsection gives an overview over common methods to handle non-orthogonal grids. The chapter concludes by presenting the main method of linearization for the Navier-Stokes equations used throughout the work.

The fourth section focusses on segregated solution methods as many ideas of the commonly used SIMPLE algorithm are relevant for the implementation of coupled solution methods. The section starts by discretizing the mass balance coming directly from the continuum mechanical introduction of the first section, outlining its drawbacks on their applicability to co-located variable arrangements. The following sections then first introduce a remedy to the \emph{checker-boarding} effect that would surge from an unmodified discretization of the mass balance by introducing a modification of the commonly used Rhie-Chow \cite{rhie82} momentum interpolation scheme. This modification is proven to yield results that do not depend on under-relaxation which is important for later comparison studies with the coupled solution algorithm. Then the popular SIMPLE algorithm is derived using the previously demonstrated interpolation scheme as basis for the construction of a pressure correction equation. In the following subsections the pressure equation and the two remaining sets of equations, the three momentum balances and the temperature equation, are discretized. Each section concludes with a complete list of the resulting discretized equations and calculations of the coefficients as they are implemented in the developed solver framework. The next subsections are dedicated to the implementation of boundary conditions. For each boundary condition, including the treatment of block boundaries, the necessary modifications to the discretized equations are presented. Since in most of the cases the boundary conditions lead to a singular system for the pressure correction, the following section describes methods to constrain the pressure in incompressible flows. The section concludes by presenting the non-zero structure of the resulting assembled linear system, taking into account the special treatment of block boundaries with hanging nodes.

The structure of the fifth section, which describes the implemented coupled solution algorithm, is similar to the fourth section, since both solution approaches share a great part of their components. This is why special emphasis lays on pointing out the differences between both algorithms. The first subsection of this section therefore is a revision of the pressure velocity coupling that accounts for the implicit consideration of the pressure in the momentum balances. A major difference to segregated solution methods is then presented in the subsequent section and subsections: the coupling to the temperature equation. Here, different methods, varying in their relation of implicit to explicit coupling of velocities, pressure and temperature are derived. Then the necessary modifications for boundary conditions are presented. The final subsection presents the final form of the discretized equations and the resulting structure of the assembled linear systems with special emphasis on the effects of the different methods to achieve coupling of the velocities to the temperature equation and vice versa.

In section six the developed \emph{CAFFA} (\emph{Computer Aided Fluid Flow Analysis}) framework is introduced for which the preceding chapters present the theoretical basis. The framework is linked to the PETSc library, a sophisticated framework for the parallelisation of solver applications as they surge in scientific computing and a set of efficient preconditioners and Krylov subspace solvers for the solution of linear systems. The following sections then introduce the other building blocks of the framework, beginning with a grid generator for arbitrarily refined block-structured grids and a preprocessor program, ending with implementation details of the CAFFA solver. These details embrace the used message passing model, the control of convergence and the data composition in the context of parallel computations.

Section seven presents the results of a conducted convergence study to verify of the CAFFA framework. After a brief introduction to the theory behind the verification of scientific codes via manufactured solutions is given, an own manufactured solution for the Navier-Stokes equations using the Boussinesq approximation is proposed, followed by the results of a grid convergence study to prove the order of accuracy of the developed solvers within the CAFFA framework. The section concludes with the results of a short study on the effect of the under-relaxation factors of the velocities on the final result of calculations.

Section eight presents the results of the performance analyses conducted on the high performance cluster \emph{HHLR} (\emph{Hessischer Hochleistungsrechner}) of TU Darmstadt. After introducing the used hardware and deployed software configuration and the relevant performance metrics, program runtime is measured and application speedup is calculated. Furthermore a study on the weak scalability of the developed fully coupled solution algorithm is conducted. The section concludes with performance analyses regarding the efficiency of coupled and segregated solution algorithms for flow problems involving complex geometries and buoyancy forces.

The last section concludes the thesis by summarizing the core results of the conducted work and giving an outlook for further investigations.

