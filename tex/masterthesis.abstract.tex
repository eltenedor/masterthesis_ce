%\renewcommand{\abstractname}{Abstract}
\begin{abstract}

%\vspace*{6 cm}

One crucial decision in designing algorithms in order to solve the incompressible Navier-Stokes equations numerically is how to treat the inherent coupling between the dependent variables, velocity and pressure. Due to advances regarding processor and memory hardware technology during the past decades, fully coupled solution algorithms have become attractive in the field of CFD (\emph{Computational Fluid Dynamics}). These algorithms denominate methods which simultaneously solve a system of partial differential equations for all involved variables.
The objective of the present thesis is to implement and analyze a pressure-based, fully coupled solution algorithm based on a finite-volume discretization of the Navier-Stokes equations including additional coupling to the temperature equation. The method uses a cell-centered, co-located variable arrangement and is designed to handle non-orthogonal, three-dimensional, locally refined, block-structured grids with hanging nodes. The implementation has been parallelized using the PETSc framework. After the verification using a manufactured solution, the single processor and parallel performance of the developed application are analyzed on a parallel benchmark to assess strong and weak scalability. A benchmark involving a duct flow with complex geometry and a temperature-driven cavity benchmark for buoyancy-driven flows are performed. The performance analysis always consists of a comparison with an implementation of a segregated solution algorithm, which was developed using the same environment. Furthermore, three correctors addressing non-orthogonality of numerical grids are analyzed with respect to their effect on the convergence of the coupled solution algorithm. The conducted test shows that the \emph{over-relaxed approach} is the most robust for different degrees of non-orthogonality of the grid. A simple technique to achieve ideal load balancing for the linear equation solvers is presented. The effects of this method on solver performance for an unbalanced data partitioning is assessed, showing that parallel calculations benefit from this load balancing technique.
In all tests, the fully coupled algorithm outperformed the segregated algorithm, provided that the number of involved unknowns was high enough. The solution of the associated linear systems was carried out using black-box solvers from the PETSc framework. It is assumed that significant improvements regarding the parallel scalability of the implemented algorithm may be achieved by special algebraic multigrid methods as preconditioner to the PETSc Krylov subspace solvers.

\end{abstract}
