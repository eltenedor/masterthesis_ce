\renewcommand{\abstractname}{Abstract}
\begin{abstract}

%\vspace*{6 cm}

One crucial decision in designing algorithms to numerically solve the incompressible Navier-Stokes equations is how to treat the inherent coupling between the dependent variables velocity and pressure. Due to advances regarding processor and memory hardware technology in the past decades, fully-coupled solution algorithms, which denominate algorithms that simultaneously solve a system of partial differential equations for all involved variables, have become attractive in the field of CFD (\emph{Computational Fluid Dynamics}). 
The objective of the present thesis is to implement and analyse a pressure based fully-coupled solution algorithm based on a finite-volume discretization of the Navier-Stokes equations including additional coupling to the temperature equation. The method uses a cell-centered co-located variable arrangement and is able to handle non-orthogonal, three-dimensional, locally refined block structured grids with hanging nodes. The implementation has been parallelized using the PETSc framework. After the verification using a manufactured solution, the single processor and parallel performance of the implementation are analysed on a parallel benchmark, a benchmark involving a duct flow with complex geometry and the heated cavity benchmark for buoyancy driven flows. The performance analysis always consists of a comparison involving an implementation of a segregated solution algorithm that was developed using the same environment.
In all tests the fully coupled algorithm did outperform the segregated algorithm, provided that the number of involved unknowns was high enough. The solution of the involved linear systems was carried out using black-box solvers from the PETSc framework. It is assumed that major improvements regarding the parallel scalability of the implemented algorithm can be achieved through the use of a special algebraic multigrid method as preconditioner.

\end{abstract}
