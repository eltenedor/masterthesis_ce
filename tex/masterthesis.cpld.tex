  \section{Implicit Finite Volume Method for Incompressible Flows -- Coupled Approach}

    Since the antecedent section already discussed the discretization of the momentum balance the focus of this section will be on highlighting the differences and presenting various approaches to incorporate different degrees of velocity-temperature and temperature-velocity/pressure coupling. It should be noted that the discretization of the equations to be solved is not changed in any way, so the presented differences only are due to difference in the solution algorithm. As in the previous section the final forms of the presented equations are presented as they are implemented in the developed solver framework. 

    \subsection{The Coupled Algorithm}
      
      \subsubsection{Pressure Equation}

      \subsubsection{Characteristic Properties of Coupled Solution Methods}

        No Underrelaxation needed, higher memory requirements

        Bad condition, singularity, usually faster convergence if efficient linear solver is chosen, coupling in Buoyancy flows (s.a. Peric page 448, Galpin Raithby)
        Design of algorithm does not need to enforce continuity (is inherently fulfilled because of the coupling of the equations)

        Explicitly mention the differences

        \begin{itemize}
          \item Implicit treatment of Pressure Gradient
          \item Implicit Treatment of Temperature possible
          \item Boussinesq approximation brings velocity-to-temperature-coupling (vakilipour), Newton-Raphson Linearization
          \item Temperature dependent densities also possible
        \end{itemize}

    \subsection{Coupling the Temperature Equation}
      
      \subsubsection{Decoupled Approach}
      \subsubsection{Velocity-Temperature Coupling}
      \subsubsection{Temperature-Velocity/Pressure Coupling -- Newton-Raphson Linearization}

    \subsection{Boundary Conditions on Domain and Block Boundaries}

      \subsubsection{Dirichlet Boundary Condition for Velocity}

      %\subsubsection{Dirichlet Boundary Condition for Pressure}

      \subsubsection{Wall Boundary Condition}

      \subsubsection{Block Boundary Condition}

    \subsection{Assembly of Linear Systems -- Final Form of Equations}

