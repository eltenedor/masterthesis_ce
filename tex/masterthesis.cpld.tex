  \section{Finite Volume Method for Incompressible Flows -- Coupled Approach}

    \subsection{The Coupled Algorithm}
      
      \subsubsection{Pressure Equation}

      \subsubsection{Characteristic Properties of Coupled Solution Methods}

        No Underrelaxation needed, higher memory requirements

        Bad condition, singularity, usually faster convergence if efficient linear solver is chosen, coupling in Buoyancy flows (s.a. Peric page 448, Galpin Raithby)
        Design of algorithm does not need to inforce continuity (is inherently fullfilled because of the coupling of the equations)

        Explicitely mention the differences

        \begin{itemize}
          \item Implicit treatment of Pressure Gradient
          \item Implicit Treatment of Temperature possible
          \item Boussinesq approximation brings velocity-to-temperature-coupling (vakilipour), Newton-Raphson Linearization
          \item Temperature dependent densities also possible
        \end{itemize}

    \subsection{Coupling the Temperature Equation}
      
      \subsubsection{Decoupled Approach}
      \subsubsection{Velocity-Temperature Coupling}
      \subsubsection{Temperature-Velocity/Pressure Coupling -- Newton-Raphson Linearization}

    \subsection{Boundary Conditions on Domain and Block Boundaries}

      \subsubsection{Dirichlet Boundary Condition for Velocity}

      \subsubsection{Dirichlet Boundary Condition for Pressure}

      \subsubsection{Symmetry Boundary Condition}

      \subsubsection{Wall Boundary Condition}

      \subsubsection{Block Boundary Condition}

    \subsection{Assembly of Linear Systems -- Final Form of Equations}

