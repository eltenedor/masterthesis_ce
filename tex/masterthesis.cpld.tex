\section{Implicit Finite-Volume Method for Incompressible Flows -- Fully Coupled Approach}
\label{sec:cpld}

Since the antecedent section \ref{sec:seg} already discussed the discretization details of the involved equations, this section aims at a comparison on the algorithmic level of the SIMPLE-algorithm, presented in section \ref{sec:simple}, with an implementation of a fully coupled solution algorithm. It should be noted, that the discretization of the equations to be solved is not changed in any way to maintain comparability, so the presented differences are due to differences in the solution algorithm exclusively. Successful implementations of a fully coupled solution algorithm for incompressible Navier-Stokes equations have been presented in \cite{chen10,darwish09,falk13,vakilipour12}. The presented work will extend the solution approach presented in \cite{falk13} to three-dimensional domains. Furthermore, this section will present various approaches to incorporate different degrees of velocity-to-temperature and temperature-to-velocity/pressure coupling, as they are presented in \cite{vakilipour12}. Finally, the structure of the resulting linear system to be solved is discussed.

\subsection{The Fully Coupled Algorithm -- Pressure-Velocity Coupling Reconsidered}
\label{sec:reconsider}

This subsection motivates the use of a fully coupled algorithm to address pressure-velocity coupling and mentions the differences to the approach presented in subsection \ref{sec:simple}. The mentioned subsection presented a common solution approach to solve the incompressible Navier-Stokes equations. After linearizing the equations, the momentum balances are solved, using the pressure from the previous iteration as a guess. Generally the velocity field obtained by solving a momentum balance with a guessed pressure does not obey continuity, hence the velocity field and the pressure field have to be corrected. This in turn leads to an inferior solution with respect to the residual of the momentum balances. To avoid this iterative guess-and-correct solution process, another class of approaches dealing with the pressure-velocity coupling problematic, represented by algorithms that are \emph{fully coupled}, will be introduced now.

The central aspect of fully coupled solution methods for Navier-Stokes equations is, that instead of solving for the velocities and pressure corrections sequentially, the velocity field and the pressure are solved for simultaneously \cite{schaefer99}. This way, every velocity field that is calculated, obeys conservation of momentum and mass, without the need to be corrected. As a result, the under-relaxation due to the resolution of the pressure-velocity coupling is no longer required, which accelerates convergence significantly. The only reason for still needing an iterative solution process, is the non-linearity of the Navier-Stokes equations, which is accounted for by the use of the Picard iteration process, which has been presented in subsection \ref{sec:nonlinear}. The iterations fortunately make room for deferred corrections as introduced in subsection \ref{sec:segdiscretization}. It should be noted that it is also possible to use the coupled solution algorithm to calculate pressure corrections. Reference \cite{klaij13} uses this approach to solve the Navier-Stokes equations by using a SIMPLE-type method as preconditioner.

On the downside, implementations of fully coupled solution methods require significantly more system memory than segregated methods. This is due to the higher amount of information, that has to be available at the same time, and the resulting linear systems, which also require more memory, due to the increased amount of unknowns to solve for. Furthermore, the bad condition of the linear algebraic system \cite{schaefer99} tends to slow down the convergence of the equation solver algorithm.

As in the case of segregated methods, it is not advisable to solve the mass balance equation directly \cite{schaefer99}. Instead of deriving an equation for the pressure correction \(p'\), as shown in \ref{sec:simple}, an equation for the pressure itself is derived, using the pressure-weighted interpolation method, which has been introduced in \ref{sec:massflux}. Concretely equation (\ref{eq:pwim}) is adapted for the use in a fully coupled algorithm
\begin{displaymath}
  u_{i,f}^{(n)} 
  =
  \left[\left(1 - \gamma_f\right) u_{i,P}^{(n)} + \gamma_f u_{i,Q}^{(n)} \right]
  - \left(\left(1 - \gamma_f\right) \frac{V_P}{a_P^{u_i}} + \gamma_f \frac{ V_Q}{a_Q^{u_i}}\right)
  \left[ 
    \underline{ \left(\frac{\partial p}{\partial x_i}\right)_f^{(n)}}
  - \frac{1}{2} 
  \left( 
    \left( \frac{\partial p}{\partial x_i} \right)_P^{(n-1)} 
  + \left(\frac{\partial p}{\partial x_i}\right)_Q^{(n-1)} 
  \right)
  \right].
\end{displaymath}
The adaption comprises the removal of last term of equation (\ref{eq:pwim}) accounting for the under-relaxation, since according to \cite{darwish09} no under-relaxation is needed in fully coupled algorithms, which is equivalent to an under-relaxation factor \(\alpha_\vec{u} = 1\). Furthermore, the underlined partial derivatives are going to be treated implicitly. This leads to the semi-implicit pressure equation
\begin{align}
  \sum_{F \in NB(P)} 
  \rho
  \left[\left(1 - \gamma_f\right) u_{i,P}^{(n)} + \gamma_f u_{i,F}^{(n)} \right]\,  S_f
  - \rho \left(\left(1 - \gamma_f\right) \frac{ V_P}{a_P^{u_i}} + \gamma_f \frac{ V_F}{a_F^{u_i}}\right)
  \left[ 
  \left(\frac{\partial p}{\partial x_i}\right)_f^{(n)}
  \right]\,  S_f \nonumber \\
  =
  - \sum_{F \in NB(P)}
  \rho
  \left(\left(1 - \gamma_f\right) \frac{ V_P}{a_P^{u_i}} + \gamma_f \frac{ V_F}{a_F^{u_i}}\right)
  \left[ 
  \frac{1}{2} 
  \left( 
    \left( \frac{\partial p}{\partial x_i} \right)_P^{(n-1)} 
  + \left(\frac{\partial p}{\partial x_i}\right)_F^{(n-1)} 
  \right)
  \right] \, S_f.
\end{align}

Even though this equation is solved for the pressure \(p\), while equation (\ref{eq:presscorr}) is solved for the pressure correction \(p'\), the matrix coefficients from the discretization of the pressure correction equation (\ref{eq:segpresscorrcoeff}) and the calculation of the right hand side given in equation (\ref{eq:presscorrb}), remain the same. For the implicit coupling to the velocities \(u_i\), additional matrix coefficients have to be considered. These can be calculated as
\begin{displaymath}
  a_F^{p,u_i} = \rho \, \gamma_f S_f \quad \text{and} \quad a_P^{p,u_i} = \sum_{F \in NB(P)} \rho \, (1-\gamma_f) \, S_f.
\end{displaymath}

On the other side, the discretized momentum balance yields matrix coefficients to account for the implicit velocity-pressure coupling. They are calculated as
\begin{displaymath}
  a_F^{u_i,p} =  \gamma_f S_f \quad \text{and} \quad a_P^{u_i,p} = \sum_{F \in NB(P)} (1-\gamma_f) \, S_f.
\end{displaymath}
If the Boussinesq approximation is used implicitly, an additional coefficient \(a_P^{u_i,T}\) to account for the velocity temperature coupling has to be considered. This will be handled in subsection \ref{sec:temperaturecoupling}. The main steps of the fully coupled solution algorithm are shown in algorithm \ref{al:coupled}.

\alglanguage{pseudocode}
\begin{algorithm}
\label{al:coupled}
\caption{Fully Coupled Solution Algorithm}
\begin{algorithmic}
\State{\textit{INITIALIZE} variables}
\While{(convergence criterion not accomplished)}
\If{(temperature coupling)}
  \State{\textit{SOLVE} the linear system for velocities, pressure and temperature}
\Else
  \State{\textit{SOLVE} the linear system for velocities and pressure}
\EndIf
\State{\textit{CALCULATE} mass fluxes using \textbf{(\ref{eq:pwim})}}
\If{(coupled scalar equation)}
  \State{\textit{SOLVE} scalar equation as described in section(\ref{sec:discretetemperature})}
\EndIf
\EndWhile
\If{(decoupled scalar equation)}
  \State{\textit{SOLVE} scalar equation as described in section(\ref{sec:discretetemperature})}
\EndIf
\end{algorithmic}
\end{algorithm}

\subsection{Coupling to the Temperature Equation}
\label{sec:temperaturecoupling}

Since the present work also aims at analyzing the efficiency of different methods to couple the temperature equation to the velocities and vice-versa, this subsection discusses different approaches to handle the coupling of the temperature equation to the Navier-Stokes equations if the Boussinesq approximation, as introduced in subsection \ref{sec:boussinesq}, is used. The effectiveness of realizing a strong coupling of the temperature equation to the Navier-Stokes equations depends on the flow problem. The physical coupling tends to increase in flow scenarios of natural convection, where the temperature difference, and hence the buoyancy term in the Navier-Stokes equations, dominates the fluid movement \cite{ferziger02,vakilipour12}. It is supposed that, the higher the physical coupling of temperature and flow, the greater the benefits of using an implementation that uses a strong coupling to the temperature equation. The following subsections present different approaches that can be used for different intensities of coupling.
      
\subsubsection{Decoupled Approach -- Explicit Velocity-to-Temperature Coupling}

The decoupled approach for addressing velocity-to-temperature coupling is similar to the treatment of the velocity-to-temperature coupling described in subsection \ref{sec:segdiscretization}. If this approach is chosen, no special measures have to be taken. The momentum balances receive a contribution \( b_P^{u_i,T}\) with
\begin{displaymath}
  b_P^{u_i,T} := - \rho \beta \left( T_P^{(n-1)} - T_0 \right) g_i V_P,
\end{displaymath}
that handles the coupling explicitly by using the temperature result of the previous outer iteration. In some cases it might be necessary to under-relax the temperature each iteration \cite{ferziger02}.

\subsubsection{Implicit Velocity-to-Temperature Coupling}

In the same way, it is possible to realize the coupling described in the previous subsection by implicitly considering the temperature in the momentum balances, which leads to an additional matrix coefficient \(a_P^{u_i,T}\) and an additional contribution to the right hand side, accounting for the coupling to the temperature equation,
\begin{displaymath}
  a_P^{u_i,T} = \rho \beta g_i V_P 
  \quad \text{and} \quad
  b_P^{u_i,T} = \rho \beta T_0 g_i V_P.
\end{displaymath}

\subsubsection{Temperature-to-Velocity/Pressure Coupling -- Newton-Raphson Linearization}
\label{sec:nrcoupled}

The previous section discussed the velocity-to-temperature coupling and so far, only the momentum balances were affected by the realization of the coupling methods. Independently, it is possible to couple the temperature equation, not only to the momentum balances, but also to the pressure equation. The purpose of this subsection is to present a discretization that achives this opposite type of coupling, which will be referred to as \emph{temperature-to-velocity/pressure} coupling.

In section \ref{sec:discretetemperature}, the non-linear partial differential equations were linearized using a Picard iteration method. Specifically the convective term \(\rho u_j T\) was linearized by taking the mass flux from the antecedent solve of the momentum and pressure correction equations. If the decoupled approach is used, this treatment remains valid. But, if an implicit temperature coupling is sought, a coupling of the temperature equation to the momentum balances which considers the mass fluxes is desirable. Methods that exhibit the described property can be found in \cite{galpin86,oliveira01,sheu04,vakilipour12}. A common denomination of the therein used linearization method is called \emph{Newton-Raphson linearization}, which can be interpreted as a bilinear approximation of the convective term by the linear terms of the respective Taylor polynomial, similar to Newton-like methods to solve non-linear equations \cite{ferziger02}. It should be noted that, even though Newton-like methods are known for their faster, up to quadratic convergence compared to methods with linear convergence like the Picard iteration, the intention of using a Newton-like linearization is to create coupling to accelerate convergence. For a Newton-like linearization to accelerate convergence, instead of only the convective term, the whole equation should use a Newton method for the non-linear solution algorithm. For this, the \emph{Newton-Raphson linearization} has not been used within segregated, SIMPLE-like solution algorithms \cite{ferziger02}.

The Newton-Raphson linearization technique is applied to the convective term of the temperature equation as follows. A first order Taylor approximation of the mass specific convective flux through the boundary face \(S_f\) around the \((n-1)\)st iteration value yields for the approximation at the \(n\)th iteration
\begin{align*}
  \left(u_{j,f} T_f \right)^{(n)} 
  &\approx 
  \left(u_{j,f} T_f \right)^{(n-1)} 
  + \frac{\partial}{\partial u_{j,f}}\left( u_{j,f} T_f \right)^{(n-1)} \left(u_{j,f}^{(n)} - u_{j,f}^{(n-1)} \right) 
  + \frac{\partial}{\partial T_f}\left( u_{j,f} T_f \right)^{(n-1)} \left( T_f^{(n)} - T_f^{(n-1)} \right) \\[0.5em]
  &=
  (u_{j,f} T_f )^{(n-1)} 
  + \vphantom{(u_{j,f} T_f)}{T_f}^{(n-1)} (\vphantom{(u_{j,f} T_f)}{u_{j,f}}^{(n)} - \vphantom{(u_{j,f} T_f)}{u_{j,f}}^{(n-1)} ) 
  +  \vphantom{(u_j T)}{u_{j,f}}^{(n-1)} ( \vphantom{(u_j T)}{T_f}^{(n)} - \vphantom{(u_j T)}{T_f}^{(n-1)} ) \\[0.7em]
  &=
  \underline{\vphantom{(u_j T)}{u_{j,f}}^{(n-1)} \vphantom{(u_{j,f})}{T_f}^{(n)}}  + \vphantom{(u_{j,f} T_f)}{u_{j,f}}^{(n)} \vphantom{(u_jT)}{T_f}^{(n-1)}  -  \vphantom{(u_j T)}{u_{j,f}}^{(n-1)} \vphantom{(u_j T)}{T_f}^{(n-1)}.
\end{align*}
A comparison with section \ref{sec:segconvective} shows, that the underlined term coincides with the term from the usual linearization. The first of the two new terms will be treated implicitly and explicitly after using the pressure-weighted interpolation method from section \ref{sec:massflux} to interpolate the value of \(u_{j,f}\). The second term will be treated explicitly. The use of the pressure-weighted interpolation method does not only create a temperature-to-velocity but also a temperature-to-pressure coupling. Applying the pressure-weighted interpolation, the equation for the mass specific convective flux reads
\begin{align*}
  \left(u_{j,f} T_f \right)^{(n)} 
  &\approx 
  \underline{\vphantom{(u_j T)}{u_{j,f}}^{(n-1)} \vphantom{(u_{j,f})}{T_f}^{(n)}} + \vphantom{(u_jT)}{T_f}^{(n-1)}  \left[\left(1 - \gamma_f\right) u_{j,P}^{(n)} + \gamma_f u_{j,Q}^{(n)} \right] \nonumber\\[1em]
    &\quad\quad 
    - T_f^{(n-1)}\left(\left(1 - \gamma_f\right) \frac{\alpha_\vec{u} V_P}{a_P^{u_j}} + \gamma_f \frac{\alpha_\vec{u} V_Q}{a_Q^{u_j}}\right)
    \left[ 
    \left(\frac{\partial p}{\partial x_j}\right)_f^{(n)} 
    -  \frac{1}{2} \left( \frac{\partial p}{\partial x_j} \right)_P^{(n-1)} 
    - \frac{1}{2} \left(\frac{\partial p}{\partial x_j}\right)_Q^{(n-1)} 
    \right] \nonumber \\[1em]
    &  \quad\quad -\vphantom{(u_j T)}{u_{j,f}}^{(n-1)} \vphantom{(u_j T)}{T_f}^{(n-1)}.
\end{align*}
If this semi-implicit, semi-discrete equation is discretized completely, new matrix and right hand side contributions can be calculated
\begin{align*}
  \allowdisplaybreaks
  &a_F^{T,u_i} = \rho T_f^{(n-1)} \gamma_f n_{f,i} S_f \\[1.0em] 
  &a_P^{T,u_i} = \sum_{F \in NB(P)} \rho T_f^{(n-1)} \left(1-\gamma_f\right) n_{f,i} S_f \\[1.0em]
  &a_F^{T,p} = -\rho T_f^{(n-1)} \left(\left(1 - \gamma_f\right) \frac{\alpha_\vec{u} V_P}{a_P^{u_i}} + \gamma_f \frac{\alpha_\vec{u} V_F}{a_F^{u_i}}\right) \frac{S_f}{\left(\vec{x}_P - \vec{x}_F\right) \cdot \vec{n}_f} = \rho T_f^{(n-1)} a_F^{p} \\[1.0em] 
  &\text{and} \quad
  a_P^{T,p} = - \sum_{F \in NB(P)} T_f^{(n-1)} a_F^{T,p}.
\end{align*}
The explicit parts resulting from the Newton-Raphson linearization yield additional contributions to the right hand side
\begin{displaymath}
  b_P^{T,\text{NR}} 
  = 
  \rho u_{j,f}^{(n-1)} T_f^{(n-1)} n_{f,j} S_f 
  - \rho T_f^{(n-1)} \left(\left(1 - \gamma_f\right) \frac{\alpha_\vec{u} V_P}{a_P^{u_j}} + \gamma_f \frac{\alpha_\vec{u} V_Q}{a_Q^{u_j}}\right)
    \left[ 
    \frac{1}{2} \left( \frac{\partial p}{\partial x_j} \right)_P^{(n-1)} 
    + \frac{1}{2} \left(\frac{\partial p}{\partial x_j}\right)_Q^{(n-1)} 
    \right] n_{j,f} S_f.
\end{displaymath}

\subsection{Boundary Conditions on Domain and Block Boundaries}

The treatment of boundary conditions and transitional conditions at block boundaries is very similar to the way presented in section \ref{sec:segboundary}. The main difference results from the fact, that, in addition to the necessary changes from section \ref{sec:segboundary}, modifications are necessary to the new matrix coefficients from the discretized momentum balance and the pressure equation in the coupled solution algorithm. The purpose of this section is therefore to only focus on these additional changes.

How Dirichlet boundary conditions, also known as inlet boundaries, affect the matrix coefficients that only correspond to the velocities has been presented in section \ref{sec:segDirichlet}. It is common to use Neumann boundary conditions for the pressure at Dirichlet boundaries with respect to the velocities \cite{darwish09}. Under the assumption that the gradient of the pressure vanishes at inlet boundaries, the matrix coefficients for the velocity-to-pressure coupling have to be modified to 
\begin{displaymath}
  a_P^{u_i,p} = a_P^{u_i,p} + n_{f,i} S_f.
\end{displaymath}
Wall boundary conditions are treated likewise with the additional modifications to the right hand side contribution as presented in section \ref{sec:walls}.

The treatment of velocity Dirichlet boundary conditions in the pressure equation is straightforward, since they correspond to mass fluxes through the boundary faces. Hence, the matrix coefficients remain unchanged and only the right hand side contributions embrace the fact that no implicit interpolation is necessary to calculate the mass flux through the boundary face.

In the case of Dirichlet pressure boundary conditions, an additional contribution to the right hand side of the momentum balances has to be considered
\begin{displaymath}
  b_P^{u_i} = b_P^{u_i} + p_f n_{i,f} S_f.
\end{displaymath}
As presented in subsection \ref{sec:segboundary} a \emph{lagged} Dirichlet boundary condition is used for the velocities in the discretized momentum balances. Different to the presentation in subsection \ref{sec:segboundary} the boundary mass fluxes in the pressure equation are considered implicitly. This creates additional contributions to the coefficients accounting for the pressure-to-velocity coupling in the pressure equation
\begin{displaymath}
  a_P^{p,u_i} = a_P^{p,u_i} + \rho n_{f,i} S_f.
\end{displaymath}
The coefficients for the pressure are modified as usual for Dirichlet boundary conditions, which was shown in section \ref{sec:segboundary}.

Finally, it should be noted that the that the Newton-Raphson linearization technique does not apply to the velocities on boundary faces, since the value of the velocities is known exactly and a linearization of the convective contribution becomes unnecessary.

%\subsubsection{Dirichlet Boundary Condition for Pressure} NOT TO BE USED

At block boundaries the new matrix coefficients \(a_L^{u_i,p}\) and \(a_R^{u_i,p}\) for the velocity-to-pressure coupling and the matrix coefficients \(a_L^{p,u_i}\) and \(a_R^{p,u_i}\) have to be taken into account. This does not present any drawbacks on the presented approach.

\subsection{Assembly of Linear Systems -- Final Form of Equations}
\label{sec:cpldassembly}

Concluding this section, the final form of the derived linear algebraic equations are presented. Based on the presented discretization, each equation only contains variable values of neighboring control volumes. The set of five equations reads
\begin{align}
  a_P^{u_i} u_{P,i}^{\vphantom{u_i}}
  + \sum_{F \in NB(P)} a_F^{u_i} u_{F,i}^{\vphantom{u_i}} 
  + \underbrace{\vphantom{\sum_F}{a_P^{u_i,p} p_P^{\vphantom{u_i,p}}
  + \sum_{F \in NB(P)} a_F^{u_i,p} p_F^{\vphantom{u_i,p}}}}_{\hbox{Pressure-velocity coupling}} 
  + \underbrace{\vphantom{\sum_F}{ a_P^{u_i,T} T_P^{\vphantom{u_i,T}}}}_{\hbox{Boussinesq approximation}} 
  &= b_{P,u_i}^{\vphantom{u_i,T}} \quad i=1,...,3  \\[2.0em]
  a_P^{\, p} p_{P}^{\vphantom{p}} 
  + \sum_{F \in NB(P)} a_F^{\,p} p_F^{\vphantom{\,p}} 
  + \underbrace{ \sum_{j=1}^3 \left(a_P^{\,p,u_j} u_{P,j}^{\vphantom{p,u_j}}
  + \sum_{F \in NB(P)} a_F^{\, p,u_j} u_{F,j}^{\vphantom{p,u_j}} \right)}_{\hbox{Pressure-velocity coupling}} 
  &= b_{P,p}^{\vphantom{p,u_j}}  \\[2.0em]
  a_P^{T} T_{P}^{\vphantom{T}} 
  + \sum_{F \in NB(P)} a_F^{T} T_F^{\vphantom{T}} 
  + \underbrace{\sum_{j=1}^3 \left(a_P^{T,u_j} u_{P,j}^{\vphantom{T,u_j}}
  + \sum_{F \in NB(P)} a_F^{T,u_j} u_{F,j}^{\vphantom{T,u_j}} \right) 
  + a_P^{T,p} p_P^{\vphantom{T,p}} 
+ \sum_{F \in NB(P) } a_F^{T,p} p_F^{\vphantom{T,p}}}_{\hbox{Newton-Raphson linearization}} 
  &= b_{P,T}^{\vphantom{T,p}}.
\end{align}

The different degrees of coupling between the different equations will affect the matrix structure. Depending on the variable, arrangement other matrix structures are possible. Figure \ref{fig:cpldassemble} shows the resulting block matrix structure if the fields are treated as interlaced quantities. Figure \ref{fig:interlacemat} shows the non-zero structure of a block matrix for the same case if no field interlacing is used. It should be noted, that the variable arrangement only affects the storage of the matrices not the way the matrix is used, since it is possible to arbitrarily reorder the matrix entries through the underlying index sets to obtain other matrix representations. One commonly used matrix representation is shown in figure \ref{fig:nointerlacemat}.

\begin{figure}
  \centering
  
\newcommand*{\xMin}{0}%
\newcommand*{\xMax}{12}%
\newcommand*{\xStep}{1.5}%
\newcommand*{\nStep}{5}
\newcommand*{\yMin}{0}%
\newcommand*{\yMax}{1}%
\newcommand\markcell[3] {
  \fill[gray,draw=black] (#2*\xStep,\xStep*\nStep-#1*\xStep) -- (#2*\xStep+\xStep,\xStep*\nStep-#1*\xStep) -- (#2*\xStep+\xStep,\xStep*\nStep-#1*\xStep-\xStep) -- (#2*\xStep,\xStep*\nStep-#1*\xStep-\xStep) -- cycle;
  \node[anchor=center,font=\large] at (#2*\xStep+0.5*\xStep,\xStep*\nStep-#1*\xStep-0.5*\xStep) {#3};
  }

\newcommand\markcellblue[3] {
  \fill[cyan,draw=black] (#2*\xStep,\xStep*\nStep-#1*\xStep) -- (#2*\xStep+\xStep,\xStep*\nStep-#1*\xStep) -- (#2*\xStep+\xStep,\xStep*\nStep-#1*\xStep-\xStep) -- (#2*\xStep,\xStep*\nStep-#1*\xStep-\xStep) -- cycle;
  \node[anchor=center,font=\large] at (#2*\xStep+0.5*\xStep,\xStep*\nStep-#1*\xStep-0.5*\xStep) {#3};
  }

\newcommand\markcellred[3] {
  \fill[red,draw=black] (#2*\xStep,\xStep*\nStep-#1*\xStep) -- (#2*\xStep+\xStep,\xStep*\nStep-#1*\xStep) -- (#2*\xStep+\xStep,\xStep*\nStep-#1*\xStep-\xStep) -- (#2*\xStep,\xStep*\nStep-#1*\xStep-\xStep) -- cycle;
  \node[anchor=center,font=\large] at (#2*\xStep+0.5*\xStep,\xStep*\nStep-#1*\xStep-0.5*\xStep) {#3};
  }

\newcommand\markcellgreen[3] {
  \fill[green,draw=black] (#2*\xStep,\xStep*\nStep-#1*\xStep) -- (#2*\xStep+\xStep,\xStep*\nStep-#1*\xStep) -- (#2*\xStep+\xStep,\xStep*\nStep-#1*\xStep-\xStep) -- (#2*\xStep,\xStep*\nStep-#1*\xStep-\xStep) -- cycle;
  \node[anchor=center,font=\large] at (#2*\xStep+0.5*\xStep,\xStep*\nStep-#1*\xStep-0.5*\xStep) {#3};
  }

\newcommand*{\xStepp}{0.15}%
\newcommand*{\nStepp}{35}
\newcommand*{\offset}{5}
\newcommand\markcelll[3] {
  \fill[gray,draw=black] (#2*\xStepp-\offset,\xStepp*\nStepp-#1*\xStepp-\offset) -- (#2*\xStepp+\xStepp-\offset,\xStepp*\nStepp-#1*\xStepp-\offset) -- (#2*\xStepp+\xStepp-\offset,\xStepp*\nStepp-#1*\xStepp-\xStepp-\offset) -- (#2*\xStepp-\offset,\xStepp*\nStepp-#1*\xStepp-\xStepp-\offset) -- cycle;
  }

\newcommand\markcellbluee[3] {
  \fill[orange,draw=black] (#2*\xStepp-\offset,\xStepp*\nStepp-#1*\xStepp-\offset) -- (#2*\xStepp+\xStepp-\offset,\xStepp*\nStepp-#1*\xStepp-\offset) -- (#2*\xStepp+\xStepp-\offset,\xStepp*\nStepp-#1*\xStepp-\xStepp-\offset) -- (#2*\xStepp-\offset,\xStepp*\nStepp-#1*\xStepp-\xStepp-\offset) -- cycle;
  }

\newcommand\markcellblackk[3] {
  \fill[black,draw=black] (#2*\xStepp-\offset,\xStepp*\nStepp-#1*\xStepp-\offset) -- (#2*\xStepp+\xStepp-\offset,\xStepp*\nStepp-#1*\xStepp-\offset) -- (#2*\xStepp+\xStepp-\offset,\xStepp*\nStepp-#1*\xStepp-\xStepp-\offset) -- (#2*\xStepp-\offset,\xStepp*\nStepp-#1*\xStepp-\xStepp-\offset) -- cycle;
  }

\begin{tikzpicture}


    %SMALL MATRIX%%%%%%%%%%%%%%%%%%%%%%%%%%%%%%%%%%%%%%%%%%%%%%%%%%%%%%%%%%%%%%%%%%%%%%%%%%%%%%%%%%%%%%%%%%%%%%%%%%%%%%%%%%%%%%%%%%
    %BACKGROUND COLORING
    \fill[gray!40] (0-\offset,\xStepp*\nStepp-8*\xStepp-\offset) -- (8*\xStepp-\offset,\xStepp*\nStepp-8*\xStepp-\offset) -- (8*\xStepp-\offset,0-\offset) -- (0*\xStepp-\offset,0-\offset) -- cycle;
    \fill[gray!40] (8*\xStepp-\offset,\xStepp*\nStepp-\offset) -- (35*\xStepp-\offset,\xStepp*\nStepp-\offset) -- (35*\xStepp-\offset,\xStepp*\nStepp-\offset-8*\xStepp) -- (8*\xStepp-\offset,\xStepp*\nStepp-8*\xStepp-\offset) -- cycle;

    \foreach \j in {0,...,35} {
        \draw [very thin,gray] (0-\offset,\j*\xStepp-\offset) -- (\xStepp*\nStepp-\offset,\j*\xStepp-\offset)  ;
        \draw [very thin,gray] (\j*\xStepp-\offset,0-\offset) -- (\j*\xStepp-\offset,\xStepp*\nStepp-\offset)  ;
    }

    \foreach \i in {0,...,34} {
      \markcelll {\i}{\i}{};
    }

    %BLOCK 1
    \markcelll {0}{1}{};
    \markcelll {0}{2}{};
    \markcelll {0}{4}{};

    \markcelll {1}{0}{};
    \markcelll {1}{3}{};
    \markcelll {1}{5}{};

    \markcelll {2}{3}{};
    \markcelll {2}{0}{};
    \markcelll {2}{6}{};

    \markcelll {3}{2}{};
    \markcelll {3}{1}{};
    \markcelll {3}{7}{};

    \markcelll {4}{5}{};
    \markcelll {4}{6}{};
    \markcelll {4}{0}{};

    \markcelll {5}{4}{};
    \markcelll {5}{7}{};
    \markcelll {5}{1}{};

    \markcelll {6}{7}{};
    \markcelll {6}{4}{};
    \markcelll {6}{2}{};

    \markcelll {7}{6}{};
    \markcelll {7}{5}{};
    \markcelll {7}{3}{};
     
    %BLOCK2
    \markcelll {8 }{9 }{};
    \markcelll {8 }{11}{};
    \markcelll {8 }{17}{};

    \markcelll {9 }{8 }{};
    \markcelll {9 }{10}{};
    \markcelll {9 }{12}{};
    \markcelll {9 }{18}{};

    \markcelll {10}{9 }{};
    \markcelll {10}{13}{};
    \markcelll {10}{19}{};

    \markcelll {11}{8 }{};
    \markcelll {11}{12}{};
    \markcelll {11}{14}{};
    \markcelll {11}{20}{};

    \markcelll {12 }{9 }{};
    \markcelll {12 }{11}{};
    \markcelll {12 }{13}{};
    \markcelll {12 }{15}{};
    \markcelll {12 }{21}{};

    \markcelll {13}{10}{};
    \markcelll {13}{12}{};
    \markcelll {13}{16}{};
    \markcelll {13}{22}{};

    \markcelll {14}{11}{};
    \markcelll {14}{15}{};
    \markcelll {14}{23}{};

    \markcelll {15 }{12}{};
    \markcelll {15 }{14}{};
    \markcelll {15 }{16}{};
    \markcelll {15 }{24}{};

    \markcelll {16}{13}{};
    \markcelll {16}{15}{};
    \markcelll {16}{25}{};

    \markcelll {17}{18 }{};
    \markcelll {17}{20}{};
    \markcelll {17}{26}{};
    \markcelll {17}{8 }{};

    \markcelll {18}{17}{};
    \markcelll {18}{19}{};
    \markcelll {18}{21}{};
    \markcelll {18}{27}{};
    \markcelll {18}{9 }{};

    \markcelll {19}{18}{};
    \markcelll {19}{22}{};
    \markcelll {19}{28}{};
    \markcelll {19}{10}{};

    \markcelll {20}{17}{};
    \markcelll {20}{21}{};
    \markcelll {20}{23}{};
    \markcelll {20}{29}{};
    \markcelll {20}{11}{};

    \markcelll {21}{18}{};
    \markcelll {21}{20}{};
    \markcelll {21}{22}{};
    \markcelll {21}{24}{};
    \markcelll {21}{30}{};
    \markcelll {21}{12}{};

    \markcelll {22}{19}{};
    \markcelll {22}{21}{};
    \markcelll {22}{25}{};
    \markcelll {22}{31}{};
    \markcelll {22}{13}{};

    \markcelll {23}{20}{};
    \markcelll {23}{24}{};
    \markcelll {23}{32}{};
    \markcelll {23}{14}{};

    \markcelll {24}{21}{};
    \markcelll {24}{23}{};
    \markcelll {24}{25}{};
    \markcelll {24}{33}{};
    \markcelll {24}{15}{};

    \markcelll {25}{22}{};
    \markcelll {25}{24}{};
    \markcelll {25}{34}{};
    \markcelll {25}{16}{};



    \markcelll {26}{27}{};
    \markcelll {26}{29}{};
    \markcelll {26}{17}{};

    \markcelll {27}{26}{};
    \markcelll {27}{28}{};
    \markcelll {27}{30}{};
    \markcelll {27}{18}{};

    \markcelll {28}{27}{};
    \markcelll {28}{31}{};
    \markcelll {28}{19}{};

    \markcelll {29}{26}{};
    \markcelll {29}{30}{};
    \markcelll {29}{32}{};
    \markcelll {29}{20}{};

    \markcelll {30}{27}{};
    \markcelll {30}{29}{};
    \markcelll {30}{31}{};
    \markcelll {30}{33}{};
    \markcelll {30}{21}{};

    \markcelll {31}{28}{};
    \markcelll {31}{30}{};
    \markcelll {31}{34}{};
    \markcelll {31}{22}{};

    \markcelll {32}{29}{};
    \markcelll {32}{33}{};
    \markcelll {32}{23}{};

    \markcelll {33}{30}{};
    \markcelll {33}{32}{};
    \markcelll {33}{34}{};
    \markcelll {33}{24}{};

    \markcelll {34}{31}{};
    \markcelll {34}{33}{};
    \markcelll {34}{25}{};

    %Connectivities
    \markcellbluee {2}{8}{};
    \markcellbluee {2}{9}{};
    \markcellbluee {2}{17}{};
    \markcellbluee {2}{18}{};

    \markcellbluee {3}{9}{};
    \markcellbluee {3}{10}{};
    \markcellbluee {3}{18}{};
    \markcellbluee {3}{19}{};

    \markcellbluee {6}{17}{};
    \markcellbluee {6}{18}{};
    \markcellbluee {6}{26}{};
    \markcellbluee {6}{27}{};

    \markcellbluee {7}{18}{};
    \markcellbluee {7}{19}{};
    \markcellbluee {7}{27}{};
    \markcellbluee {7}{28}{};

    \markcellbluee {8}{2}{};
    \markcellbluee {9}{2}{};
    \markcellbluee {17}{2}{};
    \markcellbluee {18}{2}{};

    \markcellbluee {9}{3}{};
    \markcellbluee {10}{3}{};
    \markcellbluee {18}{3}{};
    \markcellbluee {19}{3}{};

    \markcellbluee {17}{6}{};
    \markcellbluee {18}{6}{};
    \markcellbluee {26}{6}{};
    \markcellbluee {27}{6}{};

    \markcellbluee {18}{7}{};
    \markcellbluee {19}{7}{};
    \markcellbluee {27}{7}{};
    \markcellbluee {28}{7}{};

    \draw[ultra thick] (0-\offset,\xStepp*\nStepp-\offset) -- (8*\xStepp-\offset,\xStepp*\nStepp-\offset) -- (8*\xStepp-\offset,\xStepp*\nStepp-8*\xStepp-\offset) -- (0-\offset,\xStepp*\nStepp-8*\xStepp-\offset) -- cycle;
    \draw[ultra thick] (8*\xStepp-\offset,\xStepp*\nStepp-\offset) -- (35*\xStepp-\offset,\xStepp*\nStepp-\offset) -- (35*\xStepp-\offset,\xStepp*\nStepp-8*\xStepp-\offset) -- (8*\xStepp-\offset,\xStepp*\nStepp-8*\xStepp-\offset) -- cycle;
    \draw[ultra thick] (8*\xStepp-\offset,\xStepp*\nStepp-8*\xStepp-\offset) -- (35*\xStepp-\offset,\xStepp*\nStepp-8*\xStepp-\offset) -- (35*\xStepp-\offset,0-\offset) -- (8*\xStepp-\offset,0-\offset) -- cycle;
    \draw[ultra thick] (0-\offset,\xStepp*\nStepp-8*\xStepp-\offset) -- (8*\xStepp-\offset,\xStepp*\nStepp-8*\xStepp-\offset) -- (8*\xStepp-\offset,0-\offset) -- (0*\xStepp-\offset,0-\offset) -- cycle;

    \coordinate (A1) at (21*\xStepp-\offset,\xStepp*\nStepp-21*\xStepp-\offset);
    \coordinate (A2) at (21*\xStepp+\xStepp-\offset,\xStepp*\nStepp-21*\xStepp-\offset);
    \coordinate (A3) at (21*\xStepp+\xStepp-\offset,\xStepp*\nStepp-21*\xStepp-\xStepp-\offset);
    \coordinate (A4) at (21*\xStepp-\offset,\xStepp*\nStepp-21*\xStepp-\xStepp-\offset);
    \markcellblackk {21}{21}{};

    %\draw[fill=black] (A1) circle (0.15em);
    %\draw[fill=black] (A2) circle (0.15em);
    %\draw[fill=black] (A3) circle (0.15em);
    %\draw[fill=black] (A4) circle (0.15em);

    \draw[thick] (A1) -- (0,7.5);
    \draw[thick] (A2) -- (7.5,7.5);
    \draw[thick] (A3) -- (7.5,0);
    \draw[thick] (A4) -- (0,0);

    %CPLD MATRIX%%%%%%%%%%%%%%%%%%%%%%%%%%%%%%%%%%%%%%%%%%%%%%%%%%%%%%%%%%%%%%%%%%%%%%%%%%%%%%%%%%%%%%%%%%%%%%%%%%%%%%%%%%%%%%%%%%
    \foreach \j in {0,...,5} {
        \draw [very thin,gray] (0,\j*\xStep) -- (\xStep*\nStep,\j*\xStep)  ;
        \draw [very thin,gray] (\j*\xStep,0) -- (\j*\xStep,\xStep*\nStep)  ;
    }

    \markcell {0}{0}{$a_P^{u_1}$};
    \markcell {1}{1}{$a_P^{u_2}$};
    \markcell {2}{2}{$a_P^{u_3}$};
    \markcell {3}{3}{$a_P^{p}$};
    \markcell {4}{4}{$a_P^{T}$};

    \markcellblue{0}{3}{$a_P^{u_1,p}$};
    \markcellblue{1}{3}{$a_P^{u_2,p}$};
    \markcellblue{2}{3}{$a_P^{u_3,p}$};

    \markcellblue{3}{0}{$a_P^{p,u_1}$};
    \markcellblue{3}{1}{$a_P^{p,u_2}$};
    \markcellblue{3}{2}{$a_P^{p,u_3}$};

    \markcellred{0}{4}{$a_P^{u_1,T}$};
    \markcellred{1}{4}{$a_P^{u_2,T}$};
    \markcellred{2}{4}{$a_P^{u_3,T}$};

    \markcellgreen{4}{0}{$a_P^{T,u_1}$};
    \markcellgreen{4}{1}{$a_P^{T,u_2}$};
    \markcellgreen{4}{2}{$a_P^{T,u_3}$};
    \markcellgreen{4}{3}{$a_P^{T,p}$};

   \draw[ultra thick] (0,\xStep*\nStep) -- (4*\xStep,\xStep*\nStep) -- (4*\xStep,\xStep*\nStep-4*\xStep) -- (0,\xStep*\nStep-4*\xStep) -- cycle;


\end{tikzpicture}

  \caption{Non-zero structure of block submatrices of the linear systems used in the coupled solution algorithm for a block-structured grid consisting of one $2\times2\times2$ cell block and one $3\times3\times3$ cell block. The blue coefficients represent the pressure-velocity coupling, the red coefficients correspond to the velocity-to-temperature coupling due to the implicit Boussinesq-approximation and the green coefficients result from the Newton-Raphson linearization technique addressing the temperature-to-velocity/pressure coupling.}
  \label{fig:cpldassemble}
\end{figure}

\begin{figure}
  \centering
  \newcommand*{\xMin}{0}%
\newcommand*{\xMax}{12}%
\newcommand*{\xStep}{0.094}%
\newcommand*{\nStep}{175}
\newcommand*{\yMin}{0}%
\newcommand*{\yMax}{1}%
\newcommand\markcell[2] {
  \fill[gray,draw=black] (#2*\xStep,\xStep*\nStep-#1*\xStep) -- (#2*\xStep+\xStep,\xStep*\nStep-#1*\xStep) -- (#2*\xStep+\xStep,\xStep*\nStep-#1*\xStep-\xStep) -- (#2*\xStep,\xStep*\nStep-#1*\xStep-\xStep) -- cycle;
  }

\begin{tikzpicture}

    %BACKGROUND COLORING
    %\fill[gray!40] (0,\xStep*\nStep-8*\xStep) -- (8*\xStep,\xStep*\nStep-8*\xStep) -- (8*\xStep,0) -- (0*\xStep,0) -- cycle;
    %\fill[gray!40] (8*\xStep,\xStep*\nStep) -- (35*\xStep,\xStep*\nStep) -- (35*\xStep,\xStep*\nStep-8*\xStep) -- (8*\xStep,\xStep*\nStep-8*\xStep) -- cycle;

    \foreach \j in {0,5} {
    \draw [thick,gray] (0,\j*\xStep*35) -- (\xStep*\nStep,\j*\xStep*35)  ;
    \draw [thick,gray] (\j*\xStep*35,0) -- (\j*\xStep*35,\xStep*\nStep)  ;
    }

    \foreach \j in {0,...,35} {
        \draw [very thin,gray] (0,\j*\xStep*5) -- (\xStep*\nStep,\j*\xStep*5)  ;
        \draw [very thin,gray] (\j*\xStep*5,0) -- (\j*\xStep*5,\xStep*\nStep)  ;
    }

    \draw [thick,gray] (0,\xStep*\nStep-\xStep*40) -- (\xStep*\nStep,\xStep*\nStep-\xStep*40)  ;
    \draw [thick,gray] (\xStep*40,0) -- (\xStep*40,\xStep*\nStep)  ;

\markcell{  0}{  0};
\markcell{  0}{  3};
\markcell{  0}{  4};
\markcell{  0}{  5};
\markcell{  0}{ 10};
\markcell{  0}{ 13};
\markcell{  0}{ 20};
\markcell{  1}{  1};
\markcell{  1}{  3};
\markcell{  1}{  4};
\markcell{  1}{  6};
\markcell{  1}{  8};
\markcell{  1}{ 11};
\markcell{  1}{ 21};
\markcell{  2}{  2};
\markcell{  2}{  3};
\markcell{  2}{  4};
\markcell{  2}{  7};
\markcell{  2}{ 12};
\markcell{  2}{ 22};
\markcell{  2}{ 23};
\markcell{  3}{  0};
\markcell{  3}{  1};
\markcell{  3}{  2};
\markcell{  3}{  3};
\markcell{  3}{  6};
\markcell{  3}{  8};
\markcell{  3}{ 10};
\markcell{  3}{ 13};
\markcell{  3}{ 22};
\markcell{  3}{ 23};
\markcell{  4}{  0};
\markcell{  4}{  1};
\markcell{  4}{  2};
\markcell{  4}{  3};
\markcell{  4}{  4};
\markcell{  4}{  6};
\markcell{  4}{  8};
\markcell{  4}{  9};
\markcell{  4}{ 10};
\markcell{  4}{ 13};
\markcell{  4}{ 14};
\markcell{  4}{ 22};
\markcell{  4}{ 23};
\markcell{  4}{ 24};
\markcell{  5}{  0};
\markcell{  5}{  5};
\markcell{  5}{  8};
\markcell{  5}{  9};
\markcell{  5}{ 15};
\markcell{  5}{ 18};
\markcell{  5}{ 25};
\markcell{  6}{  1};
\markcell{  6}{  3};
\markcell{  6}{  6};
\markcell{  6}{  8};
\markcell{  6}{  9};
\markcell{  6}{ 16};
\markcell{  6}{ 26};
\markcell{  7}{  2};
\markcell{  7}{  7};
\markcell{  7}{  8};
\markcell{  7}{  9};
\markcell{  7}{ 17};
\markcell{  7}{ 27};
\markcell{  7}{ 28};
\markcell{  8}{  1};
\markcell{  8}{  3};
\markcell{  8}{  5};
\markcell{  8}{  6};
\markcell{  8}{  7};
\markcell{  8}{  8};
\markcell{  8}{ 15};
\markcell{  8}{ 18};
\markcell{  8}{ 27};
\markcell{  8}{ 28};
\markcell{  9}{  1};
\markcell{  9}{  3};
\markcell{  9}{  4};
\markcell{  9}{  5};
\markcell{  9}{  6};
\markcell{  9}{  7};
\markcell{  9}{  8};
\markcell{  9}{  9};
\markcell{  9}{ 15};
\markcell{  9}{ 18};
\markcell{  9}{ 19};
\markcell{  9}{ 27};
\markcell{  9}{ 28};
\markcell{  9}{ 29};
\markcell{ 10}{  0};
\markcell{ 10}{  3};
\markcell{ 10}{ 10};
\markcell{ 10}{ 13};
\markcell{ 10}{ 14};
\markcell{ 10}{ 15};
\markcell{ 10}{ 30};
\markcell{ 10}{ 40};
\markcell{ 10}{ 43};
\markcell{ 10}{ 45};
\markcell{ 10}{ 48};
\markcell{ 10}{ 85};
\markcell{ 10}{ 88};
\markcell{ 10}{ 90};
\markcell{ 10}{ 93};
\markcell{ 11}{  1};
\markcell{ 11}{ 11};
\markcell{ 11}{ 13};
\markcell{ 11}{ 14};
\markcell{ 11}{ 16};
\markcell{ 11}{ 18};
\markcell{ 11}{ 31};
\markcell{ 11}{ 41};
\markcell{ 11}{ 46};
\markcell{ 11}{ 86};
\markcell{ 11}{ 91};
\markcell{ 12}{  2};
\markcell{ 12}{ 12};
\markcell{ 12}{ 13};
\markcell{ 12}{ 14};
\markcell{ 12}{ 17};
\markcell{ 12}{ 32};
\markcell{ 12}{ 33};
\markcell{ 12}{ 42};
\markcell{ 12}{ 47};
\markcell{ 12}{ 87};
\markcell{ 12}{ 92};
\markcell{ 13}{  0};
\markcell{ 13}{  3};
\markcell{ 13}{ 10};
\markcell{ 13}{ 11};
\markcell{ 13}{ 12};
\markcell{ 13}{ 13};
\markcell{ 13}{ 16};
\markcell{ 13}{ 18};
\markcell{ 13}{ 32};
\markcell{ 13}{ 33};
\markcell{ 13}{ 40};
\markcell{ 13}{ 43};
\markcell{ 13}{ 45};
\markcell{ 13}{ 48};
\markcell{ 13}{ 85};
\markcell{ 13}{ 88};
\markcell{ 13}{ 90};
\markcell{ 13}{ 93};
\markcell{ 14}{  0};
\markcell{ 14}{  3};
\markcell{ 14}{  4};
\markcell{ 14}{ 10};
\markcell{ 14}{ 11};
\markcell{ 14}{ 12};
\markcell{ 14}{ 13};
\markcell{ 14}{ 14};
\markcell{ 14}{ 16};
\markcell{ 14}{ 18};
\markcell{ 14}{ 19};
\markcell{ 14}{ 32};
\markcell{ 14}{ 33};
\markcell{ 14}{ 34};
\markcell{ 14}{ 40};
\markcell{ 14}{ 43};
\markcell{ 14}{ 44};
\markcell{ 14}{ 45};
\markcell{ 14}{ 48};
\markcell{ 14}{ 49};
\markcell{ 14}{ 85};
\markcell{ 14}{ 88};
\markcell{ 14}{ 89};
\markcell{ 14}{ 90};
\markcell{ 14}{ 93};
\markcell{ 14}{ 94};
\markcell{ 15}{  5};
\markcell{ 15}{  8};
\markcell{ 15}{ 10};
\markcell{ 15}{ 15};
\markcell{ 15}{ 18};
\markcell{ 15}{ 19};
\markcell{ 15}{ 35};
\markcell{ 15}{ 45};
\markcell{ 15}{ 48};
\markcell{ 15}{ 50};
\markcell{ 15}{ 53};
\markcell{ 15}{ 90};
\markcell{ 15}{ 93};
\markcell{ 15}{ 95};
\markcell{ 15}{ 98};
\markcell{ 16}{  6};
\markcell{ 16}{ 11};
\markcell{ 16}{ 13};
\markcell{ 16}{ 16};
\markcell{ 16}{ 18};
\markcell{ 16}{ 19};
\markcell{ 16}{ 36};
\markcell{ 16}{ 46};
\markcell{ 16}{ 51};
\markcell{ 16}{ 91};
\markcell{ 16}{ 96};
\markcell{ 17}{  7};
\markcell{ 17}{ 12};
\markcell{ 17}{ 17};
\markcell{ 17}{ 18};
\markcell{ 17}{ 19};
\markcell{ 17}{ 37};
\markcell{ 17}{ 38};
\markcell{ 17}{ 47};
\markcell{ 17}{ 52};
\markcell{ 17}{ 92};
\markcell{ 17}{ 97};
\markcell{ 18}{  5};
\markcell{ 18}{  8};
\markcell{ 18}{ 11};
\markcell{ 18}{ 13};
\markcell{ 18}{ 15};
\markcell{ 18}{ 16};
\markcell{ 18}{ 17};
\markcell{ 18}{ 18};
\markcell{ 18}{ 37};
\markcell{ 18}{ 38};
\markcell{ 18}{ 45};
\markcell{ 18}{ 48};
\markcell{ 18}{ 50};
\markcell{ 18}{ 53};
\markcell{ 18}{ 90};
\markcell{ 18}{ 93};
\markcell{ 18}{ 95};
\markcell{ 18}{ 98};
\markcell{ 19}{  5};
\markcell{ 19}{  8};
\markcell{ 19}{  9};
\markcell{ 19}{ 11};
\markcell{ 19}{ 13};
\markcell{ 19}{ 14};
\markcell{ 19}{ 15};
\markcell{ 19}{ 16};
\markcell{ 19}{ 17};
\markcell{ 19}{ 18};
\markcell{ 19}{ 19};
\markcell{ 19}{ 37};
\markcell{ 19}{ 38};
\markcell{ 19}{ 39};
\markcell{ 19}{ 45};
\markcell{ 19}{ 48};
\markcell{ 19}{ 49};
\markcell{ 19}{ 50};
\markcell{ 19}{ 53};
\markcell{ 19}{ 54};
\markcell{ 19}{ 90};
\markcell{ 19}{ 93};
\markcell{ 19}{ 94};
\markcell{ 19}{ 95};
\markcell{ 19}{ 98};
\markcell{ 19}{ 99};
\markcell{ 20}{  0};
\markcell{ 20}{ 20};
\markcell{ 20}{ 23};
\markcell{ 20}{ 24};
\markcell{ 20}{ 25};
\markcell{ 20}{ 30};
\markcell{ 20}{ 33};
\markcell{ 21}{  1};
\markcell{ 21}{ 21};
\markcell{ 21}{ 23};
\markcell{ 21}{ 24};
\markcell{ 21}{ 26};
\markcell{ 21}{ 28};
\markcell{ 21}{ 31};
\markcell{ 22}{  2};
\markcell{ 22}{  3};
\markcell{ 22}{ 22};
\markcell{ 22}{ 23};
\markcell{ 22}{ 24};
\markcell{ 22}{ 27};
\markcell{ 22}{ 32};
\markcell{ 23}{  2};
\markcell{ 23}{  3};
\markcell{ 23}{ 20};
\markcell{ 23}{ 21};
\markcell{ 23}{ 22};
\markcell{ 23}{ 23};
\markcell{ 23}{ 26};
\markcell{ 23}{ 28};
\markcell{ 23}{ 30};
\markcell{ 23}{ 33};
\markcell{ 24}{  2};
\markcell{ 24}{  3};
\markcell{ 24}{  4};
\markcell{ 24}{ 20};
\markcell{ 24}{ 21};
\markcell{ 24}{ 22};
\markcell{ 24}{ 23};
\markcell{ 24}{ 24};
\markcell{ 24}{ 26};
\markcell{ 24}{ 28};
\markcell{ 24}{ 29};
\markcell{ 24}{ 30};
\markcell{ 24}{ 33};
\markcell{ 24}{ 34};
\markcell{ 25}{  5};
\markcell{ 25}{ 20};
\markcell{ 25}{ 25};
\markcell{ 25}{ 28};
\markcell{ 25}{ 29};
\markcell{ 25}{ 35};
\markcell{ 25}{ 38};
\markcell{ 26}{  6};
\markcell{ 26}{ 21};
\markcell{ 26}{ 23};
\markcell{ 26}{ 26};
\markcell{ 26}{ 28};
\markcell{ 26}{ 29};
\markcell{ 26}{ 36};
\markcell{ 27}{  7};
\markcell{ 27}{  8};
\markcell{ 27}{ 22};
\markcell{ 27}{ 27};
\markcell{ 27}{ 28};
\markcell{ 27}{ 29};
\markcell{ 27}{ 37};
\markcell{ 28}{  7};
\markcell{ 28}{  8};
\markcell{ 28}{ 21};
\markcell{ 28}{ 23};
\markcell{ 28}{ 25};
\markcell{ 28}{ 26};
\markcell{ 28}{ 27};
\markcell{ 28}{ 28};
\markcell{ 28}{ 35};
\markcell{ 28}{ 38};
\markcell{ 29}{  7};
\markcell{ 29}{  8};
\markcell{ 29}{  9};
\markcell{ 29}{ 21};
\markcell{ 29}{ 23};
\markcell{ 29}{ 24};
\markcell{ 29}{ 25};
\markcell{ 29}{ 26};
\markcell{ 29}{ 27};
\markcell{ 29}{ 28};
\markcell{ 29}{ 29};
\markcell{ 29}{ 35};
\markcell{ 29}{ 38};
\markcell{ 29}{ 39};
\markcell{ 30}{ 10};
\markcell{ 30}{ 20};
\markcell{ 30}{ 23};
\markcell{ 30}{ 30};
\markcell{ 30}{ 33};
\markcell{ 30}{ 34};
\markcell{ 30}{ 35};
\markcell{ 30}{ 85};
\markcell{ 30}{ 88};
\markcell{ 30}{ 90};
\markcell{ 30}{ 93};
\markcell{ 30}{130};
\markcell{ 30}{133};
\markcell{ 30}{135};
\markcell{ 30}{138};
\markcell{ 31}{ 11};
\markcell{ 31}{ 21};
\markcell{ 31}{ 31};
\markcell{ 31}{ 33};
\markcell{ 31}{ 34};
\markcell{ 31}{ 36};
\markcell{ 31}{ 38};
\markcell{ 31}{ 86};
\markcell{ 31}{ 91};
\markcell{ 31}{131};
\markcell{ 31}{136};
\markcell{ 32}{ 12};
\markcell{ 32}{ 13};
\markcell{ 32}{ 22};
\markcell{ 32}{ 32};
\markcell{ 32}{ 33};
\markcell{ 32}{ 34};
\markcell{ 32}{ 37};
\markcell{ 32}{ 87};
\markcell{ 32}{ 92};
\markcell{ 32}{132};
\markcell{ 32}{137};
\markcell{ 33}{ 12};
\markcell{ 33}{ 13};
\markcell{ 33}{ 20};
\markcell{ 33}{ 23};
\markcell{ 33}{ 30};
\markcell{ 33}{ 31};
\markcell{ 33}{ 32};
\markcell{ 33}{ 33};
\markcell{ 33}{ 36};
\markcell{ 33}{ 38};
\markcell{ 33}{ 85};
\markcell{ 33}{ 88};
\markcell{ 33}{ 90};
\markcell{ 33}{ 93};
\markcell{ 33}{130};
\markcell{ 33}{133};
\markcell{ 33}{135};
\markcell{ 33}{138};
\markcell{ 34}{ 12};
\markcell{ 34}{ 13};
\markcell{ 34}{ 14};
\markcell{ 34}{ 20};
\markcell{ 34}{ 23};
\markcell{ 34}{ 24};
\markcell{ 34}{ 30};
\markcell{ 34}{ 31};
\markcell{ 34}{ 32};
\markcell{ 34}{ 33};
\markcell{ 34}{ 34};
\markcell{ 34}{ 36};
\markcell{ 34}{ 38};
\markcell{ 34}{ 39};
\markcell{ 34}{ 85};
\markcell{ 34}{ 88};
\markcell{ 34}{ 89};
\markcell{ 34}{ 90};
\markcell{ 34}{ 93};
\markcell{ 34}{ 94};
\markcell{ 34}{130};
\markcell{ 34}{133};
\markcell{ 34}{134};
\markcell{ 34}{135};
\markcell{ 34}{138};
\markcell{ 34}{139};
\markcell{ 35}{ 15};
\markcell{ 35}{ 25};
\markcell{ 35}{ 28};
\markcell{ 35}{ 30};
\markcell{ 35}{ 35};
\markcell{ 35}{ 38};
\markcell{ 35}{ 39};
\markcell{ 35}{ 90};
\markcell{ 35}{ 93};
\markcell{ 35}{ 95};
\markcell{ 35}{ 98};
\markcell{ 35}{135};
\markcell{ 35}{138};
\markcell{ 35}{140};
\markcell{ 35}{143};
\markcell{ 36}{ 16};
\markcell{ 36}{ 26};
\markcell{ 36}{ 31};
\markcell{ 36}{ 33};
\markcell{ 36}{ 36};
\markcell{ 36}{ 38};
\markcell{ 36}{ 39};
\markcell{ 36}{ 91};
\markcell{ 36}{ 96};
\markcell{ 36}{136};
\markcell{ 36}{141};
\markcell{ 37}{ 17};
\markcell{ 37}{ 18};
\markcell{ 37}{ 27};
\markcell{ 37}{ 32};
\markcell{ 37}{ 37};
\markcell{ 37}{ 38};
\markcell{ 37}{ 39};
\markcell{ 37}{ 92};
\markcell{ 37}{ 97};
\markcell{ 37}{137};
\markcell{ 37}{142};
\markcell{ 38}{ 17};
\markcell{ 38}{ 18};
\markcell{ 38}{ 25};
\markcell{ 38}{ 28};
\markcell{ 38}{ 31};
\markcell{ 38}{ 33};
\markcell{ 38}{ 35};
\markcell{ 38}{ 36};
\markcell{ 38}{ 37};
\markcell{ 38}{ 38};
\markcell{ 38}{ 90};
\markcell{ 38}{ 93};
\markcell{ 38}{ 95};
\markcell{ 38}{ 98};
\markcell{ 38}{135};
\markcell{ 38}{138};
\markcell{ 38}{140};
\markcell{ 38}{143};
\markcell{ 39}{ 17};
\markcell{ 39}{ 18};
\markcell{ 39}{ 19};
\markcell{ 39}{ 25};
\markcell{ 39}{ 28};
\markcell{ 39}{ 29};
\markcell{ 39}{ 31};
\markcell{ 39}{ 33};
\markcell{ 39}{ 34};
\markcell{ 39}{ 35};
\markcell{ 39}{ 36};
\markcell{ 39}{ 37};
\markcell{ 39}{ 38};
\markcell{ 39}{ 39};
\markcell{ 39}{ 90};
\markcell{ 39}{ 93};
\markcell{ 39}{ 94};
\markcell{ 39}{ 95};
\markcell{ 39}{ 98};
\markcell{ 39}{ 99};
\markcell{ 39}{135};
\markcell{ 39}{138};
\markcell{ 39}{139};
\markcell{ 39}{140};
\markcell{ 39}{143};
\markcell{ 39}{144};
\markcell{ 40}{ 10};
\markcell{ 40}{ 13};
\markcell{ 40}{ 40};
\markcell{ 40}{ 43};
\markcell{ 40}{ 44};
\markcell{ 40}{ 45};
\markcell{ 40}{ 55};
\markcell{ 40}{ 58};
\markcell{ 40}{ 85};
\markcell{ 41}{ 11};
\markcell{ 41}{ 41};
\markcell{ 41}{ 43};
\markcell{ 41}{ 44};
\markcell{ 41}{ 46};
\markcell{ 41}{ 48};
\markcell{ 41}{ 56};
\markcell{ 41}{ 86};
\markcell{ 42}{ 12};
\markcell{ 42}{ 42};
\markcell{ 42}{ 43};
\markcell{ 42}{ 44};
\markcell{ 42}{ 47};
\markcell{ 42}{ 57};
\markcell{ 42}{ 87};
\markcell{ 42}{ 88};
\markcell{ 43}{ 10};
\markcell{ 43}{ 13};
\markcell{ 43}{ 40};
\markcell{ 43}{ 41};
\markcell{ 43}{ 42};
\markcell{ 43}{ 43};
\markcell{ 43}{ 46};
\markcell{ 43}{ 48};
\markcell{ 43}{ 55};
\markcell{ 43}{ 58};
\markcell{ 43}{ 87};
\markcell{ 43}{ 88};
\markcell{ 44}{ 10};
\markcell{ 44}{ 13};
\markcell{ 44}{ 14};
\markcell{ 44}{ 40};
\markcell{ 44}{ 41};
\markcell{ 44}{ 42};
\markcell{ 44}{ 43};
\markcell{ 44}{ 44};
\markcell{ 44}{ 46};
\markcell{ 44}{ 48};
\markcell{ 44}{ 49};
\markcell{ 44}{ 55};
\markcell{ 44}{ 58};
\markcell{ 44}{ 59};
\markcell{ 44}{ 87};
\markcell{ 44}{ 88};
\markcell{ 44}{ 89};
\markcell{ 45}{ 10};
\markcell{ 45}{ 13};
\markcell{ 45}{ 15};
\markcell{ 45}{ 18};
\markcell{ 45}{ 40};
\markcell{ 45}{ 45};
\markcell{ 45}{ 48};
\markcell{ 45}{ 49};
\markcell{ 45}{ 50};
\markcell{ 45}{ 60};
\markcell{ 45}{ 63};
\markcell{ 45}{ 90};
\markcell{ 46}{ 11};
\markcell{ 46}{ 16};
\markcell{ 46}{ 41};
\markcell{ 46}{ 43};
\markcell{ 46}{ 46};
\markcell{ 46}{ 48};
\markcell{ 46}{ 49};
\markcell{ 46}{ 51};
\markcell{ 46}{ 53};
\markcell{ 46}{ 61};
\markcell{ 46}{ 91};
\markcell{ 47}{ 12};
\markcell{ 47}{ 17};
\markcell{ 47}{ 42};
\markcell{ 47}{ 47};
\markcell{ 47}{ 48};
\markcell{ 47}{ 49};
\markcell{ 47}{ 52};
\markcell{ 47}{ 62};
\markcell{ 47}{ 92};
\markcell{ 47}{ 93};
\markcell{ 48}{ 10};
\markcell{ 48}{ 13};
\markcell{ 48}{ 15};
\markcell{ 48}{ 18};
\markcell{ 48}{ 41};
\markcell{ 48}{ 43};
\markcell{ 48}{ 45};
\markcell{ 48}{ 46};
\markcell{ 48}{ 47};
\markcell{ 48}{ 48};
\markcell{ 48}{ 51};
\markcell{ 48}{ 53};
\markcell{ 48}{ 60};
\markcell{ 48}{ 63};
\markcell{ 48}{ 92};
\markcell{ 48}{ 93};
\markcell{ 49}{ 10};
\markcell{ 49}{ 13};
\markcell{ 49}{ 14};
\markcell{ 49}{ 15};
\markcell{ 49}{ 18};
\markcell{ 49}{ 19};
\markcell{ 49}{ 41};
\markcell{ 49}{ 43};
\markcell{ 49}{ 44};
\markcell{ 49}{ 45};
\markcell{ 49}{ 46};
\markcell{ 49}{ 47};
\markcell{ 49}{ 48};
\markcell{ 49}{ 49};
\markcell{ 49}{ 51};
\markcell{ 49}{ 53};
\markcell{ 49}{ 54};
\markcell{ 49}{ 60};
\markcell{ 49}{ 63};
\markcell{ 49}{ 64};
\markcell{ 49}{ 92};
\markcell{ 49}{ 93};
\markcell{ 49}{ 94};
\markcell{ 50}{ 15};
\markcell{ 50}{ 18};
\markcell{ 50}{ 45};
\markcell{ 50}{ 50};
\markcell{ 50}{ 53};
\markcell{ 50}{ 54};
\markcell{ 50}{ 65};
\markcell{ 50}{ 68};
\markcell{ 50}{ 95};
\markcell{ 51}{ 16};
\markcell{ 51}{ 46};
\markcell{ 51}{ 48};
\markcell{ 51}{ 51};
\markcell{ 51}{ 53};
\markcell{ 51}{ 54};
\markcell{ 51}{ 66};
\markcell{ 51}{ 96};
\markcell{ 52}{ 17};
\markcell{ 52}{ 47};
\markcell{ 52}{ 52};
\markcell{ 52}{ 53};
\markcell{ 52}{ 54};
\markcell{ 52}{ 67};
\markcell{ 52}{ 97};
\markcell{ 52}{ 98};
\markcell{ 53}{ 15};
\markcell{ 53}{ 18};
\markcell{ 53}{ 46};
\markcell{ 53}{ 48};
\markcell{ 53}{ 50};
\markcell{ 53}{ 51};
\markcell{ 53}{ 52};
\markcell{ 53}{ 53};
\markcell{ 53}{ 65};
\markcell{ 53}{ 68};
\markcell{ 53}{ 97};
\markcell{ 53}{ 98};
\markcell{ 54}{ 15};
\markcell{ 54}{ 18};
\markcell{ 54}{ 19};
\markcell{ 54}{ 46};
\markcell{ 54}{ 48};
\markcell{ 54}{ 49};
\markcell{ 54}{ 50};
\markcell{ 54}{ 51};
\markcell{ 54}{ 52};
\markcell{ 54}{ 53};
\markcell{ 54}{ 54};
\markcell{ 54}{ 65};
\markcell{ 54}{ 68};
\markcell{ 54}{ 69};
\markcell{ 54}{ 97};
\markcell{ 54}{ 98};
\markcell{ 54}{ 99};
\markcell{ 55}{ 40};
\markcell{ 55}{ 43};
\markcell{ 55}{ 55};
\markcell{ 55}{ 58};
\markcell{ 55}{ 59};
\markcell{ 55}{ 60};
\markcell{ 55}{ 70};
\markcell{ 55}{ 73};
\markcell{ 55}{100};
\markcell{ 56}{ 41};
\markcell{ 56}{ 56};
\markcell{ 56}{ 58};
\markcell{ 56}{ 59};
\markcell{ 56}{ 61};
\markcell{ 56}{ 63};
\markcell{ 56}{ 71};
\markcell{ 56}{101};
\markcell{ 57}{ 42};
\markcell{ 57}{ 57};
\markcell{ 57}{ 58};
\markcell{ 57}{ 59};
\markcell{ 57}{ 62};
\markcell{ 57}{ 72};
\markcell{ 57}{102};
\markcell{ 57}{103};
\markcell{ 58}{ 40};
\markcell{ 58}{ 43};
\markcell{ 58}{ 55};
\markcell{ 58}{ 56};
\markcell{ 58}{ 57};
\markcell{ 58}{ 58};
\markcell{ 58}{ 61};
\markcell{ 58}{ 63};
\markcell{ 58}{ 70};
\markcell{ 58}{ 73};
\markcell{ 58}{102};
\markcell{ 58}{103};
\markcell{ 59}{ 40};
\markcell{ 59}{ 43};
\markcell{ 59}{ 44};
\markcell{ 59}{ 55};
\markcell{ 59}{ 56};
\markcell{ 59}{ 57};
\markcell{ 59}{ 58};
\markcell{ 59}{ 59};
\markcell{ 59}{ 61};
\markcell{ 59}{ 63};
\markcell{ 59}{ 64};
\markcell{ 59}{ 70};
\markcell{ 59}{ 73};
\markcell{ 59}{ 74};
\markcell{ 59}{102};
\markcell{ 59}{103};
\markcell{ 59}{104};
\markcell{ 60}{ 45};
\markcell{ 60}{ 48};
\markcell{ 60}{ 55};
\markcell{ 60}{ 60};
\markcell{ 60}{ 63};
\markcell{ 60}{ 64};
\markcell{ 60}{ 65};
\markcell{ 60}{ 75};
\markcell{ 60}{ 78};
\markcell{ 60}{105};
\markcell{ 61}{ 46};
\markcell{ 61}{ 56};
\markcell{ 61}{ 58};
\markcell{ 61}{ 61};
\markcell{ 61}{ 64};
\markcell{ 61}{ 66};
\markcell{ 61}{ 68};
\markcell{ 61}{ 76};
\markcell{ 61}{106};
\markcell{ 62}{ 47};
\markcell{ 62}{ 57};
\markcell{ 62}{ 62};
\markcell{ 62}{ 63};
\markcell{ 62}{ 64};
\markcell{ 62}{ 67};
\markcell{ 62}{ 77};
\markcell{ 62}{107};
\markcell{ 62}{108};
\markcell{ 63}{ 45};
\markcell{ 63}{ 48};
\markcell{ 63}{ 56};
\markcell{ 63}{ 58};
\markcell{ 63}{ 60};
\markcell{ 63}{ 62};
\markcell{ 63}{ 63};
\markcell{ 63}{ 66};
\markcell{ 63}{ 68};
\markcell{ 63}{ 75};
\markcell{ 63}{ 78};
\markcell{ 63}{107};
\markcell{ 63}{108};
\markcell{ 64}{ 45};
\markcell{ 64}{ 48};
\markcell{ 64}{ 49};
\markcell{ 64}{ 56};
\markcell{ 64}{ 58};
\markcell{ 64}{ 59};
\markcell{ 64}{ 60};
\markcell{ 64}{ 61};
\markcell{ 64}{ 62};
\markcell{ 64}{ 63};
\markcell{ 64}{ 64};
\markcell{ 64}{ 66};
\markcell{ 64}{ 68};
\markcell{ 64}{ 69};
\markcell{ 64}{ 75};
\markcell{ 64}{ 78};
\markcell{ 64}{ 79};
\markcell{ 64}{107};
\markcell{ 64}{108};
\markcell{ 64}{109};
\markcell{ 65}{ 50};
\markcell{ 65}{ 53};
\markcell{ 65}{ 60};
\markcell{ 65}{ 65};
\markcell{ 65}{ 68};
\markcell{ 65}{ 69};
\markcell{ 65}{ 80};
\markcell{ 65}{ 83};
\markcell{ 65}{110};
\markcell{ 66}{ 51};
\markcell{ 66}{ 61};
\markcell{ 66}{ 63};
\markcell{ 66}{ 66};
\markcell{ 66}{ 68};
\markcell{ 66}{ 69};
\markcell{ 66}{ 81};
\markcell{ 66}{111};
\markcell{ 67}{ 52};
\markcell{ 67}{ 62};
\markcell{ 67}{ 67};
\markcell{ 67}{ 68};
\markcell{ 67}{ 69};
\markcell{ 67}{ 82};
\markcell{ 67}{112};
\markcell{ 67}{113};
\markcell{ 68}{ 50};
\markcell{ 68}{ 53};
\markcell{ 68}{ 61};
\markcell{ 68}{ 63};
\markcell{ 68}{ 65};
\markcell{ 68}{ 66};
\markcell{ 68}{ 67};
\markcell{ 68}{ 68};
\markcell{ 68}{ 80};
\markcell{ 68}{ 83};
\markcell{ 68}{112};
\markcell{ 68}{113};
\markcell{ 69}{ 50};
\markcell{ 69}{ 53};
\markcell{ 69}{ 54};
\markcell{ 69}{ 61};
\markcell{ 69}{ 63};
\markcell{ 69}{ 64};
\markcell{ 69}{ 65};
\markcell{ 69}{ 66};
\markcell{ 69}{ 67};
\markcell{ 69}{ 68};
\markcell{ 69}{ 69};
\markcell{ 69}{ 80};
\markcell{ 69}{ 83};
\markcell{ 69}{ 84};
\markcell{ 69}{112};
\markcell{ 69}{113};
\markcell{ 69}{114};
\markcell{ 70}{ 55};
\markcell{ 70}{ 58};
\markcell{ 70}{ 70};
\markcell{ 70}{ 73};
\markcell{ 70}{ 74};
\markcell{ 70}{ 75};
\markcell{ 70}{115};
\markcell{ 71}{ 56};
\markcell{ 71}{ 71};
\markcell{ 71}{ 73};
\markcell{ 71}{ 74};
\markcell{ 71}{ 76};
\markcell{ 71}{ 78};
\markcell{ 71}{116};
\markcell{ 72}{ 57};
\markcell{ 72}{ 72};
\markcell{ 72}{ 73};
\markcell{ 72}{ 74};
\markcell{ 72}{ 77};
\markcell{ 72}{117};
\markcell{ 72}{118};
\markcell{ 73}{ 55};
\markcell{ 73}{ 58};
\markcell{ 73}{ 70};
\markcell{ 73}{ 71};
\markcell{ 73}{ 72};
\markcell{ 73}{ 73};
\markcell{ 73}{ 76};
\markcell{ 73}{ 78};
\markcell{ 73}{117};
\markcell{ 73}{118};
\markcell{ 74}{ 55};
\markcell{ 74}{ 58};
\markcell{ 74}{ 59};
\markcell{ 74}{ 70};
\markcell{ 74}{ 71};
\markcell{ 74}{ 72};
\markcell{ 74}{ 73};
\markcell{ 74}{ 74};
\markcell{ 74}{ 76};
\markcell{ 74}{ 78};
\markcell{ 74}{ 79};
\markcell{ 74}{117};
\markcell{ 74}{118};
\markcell{ 74}{119};
\markcell{ 75}{ 60};
\markcell{ 75}{ 63};
\markcell{ 75}{ 70};
\markcell{ 75}{ 75};
\markcell{ 75}{ 78};
\markcell{ 75}{ 79};
\markcell{ 75}{ 80};
\markcell{ 75}{120};
\markcell{ 76}{ 61};
\markcell{ 76}{ 71};
\markcell{ 76}{ 73};
\markcell{ 76}{ 76};
\markcell{ 76}{ 78};
\markcell{ 76}{ 79};
\markcell{ 76}{ 81};
\markcell{ 76}{ 83};
\markcell{ 76}{121};
\markcell{ 77}{ 62};
\markcell{ 77}{ 72};
\markcell{ 77}{ 77};
\markcell{ 77}{ 78};
\markcell{ 77}{ 79};
\markcell{ 77}{ 82};
\markcell{ 77}{122};
\markcell{ 77}{123};
\markcell{ 78}{ 60};
\markcell{ 78}{ 63};
\markcell{ 78}{ 71};
\markcell{ 78}{ 73};
\markcell{ 78}{ 75};
\markcell{ 78}{ 76};
\markcell{ 78}{ 77};
\markcell{ 78}{ 78};
\markcell{ 78}{ 81};
\markcell{ 78}{ 83};
\markcell{ 78}{122};
\markcell{ 78}{123};
\markcell{ 79}{ 60};
\markcell{ 79}{ 63};
\markcell{ 79}{ 64};
\markcell{ 79}{ 71};
\markcell{ 79}{ 73};
\markcell{ 79}{ 74};
\markcell{ 79}{ 75};
\markcell{ 79}{ 76};
\markcell{ 79}{ 77};
\markcell{ 79}{ 78};
\markcell{ 79}{ 79};
\markcell{ 79}{ 81};
\markcell{ 79}{ 83};
\markcell{ 79}{ 84};
\markcell{ 79}{122};
\markcell{ 79}{123};
\markcell{ 79}{124};
\markcell{ 80}{ 65};
\markcell{ 80}{ 68};
\markcell{ 80}{ 75};
\markcell{ 80}{ 80};
\markcell{ 80}{ 83};
\markcell{ 80}{ 84};
\markcell{ 80}{125};
\markcell{ 81}{ 66};
\markcell{ 81}{ 76};
\markcell{ 81}{ 78};
\markcell{ 81}{ 81};
\markcell{ 81}{ 83};
\markcell{ 81}{ 84};
\markcell{ 81}{126};
\markcell{ 82}{ 67};
\markcell{ 82}{ 77};
\markcell{ 82}{ 82};
\markcell{ 82}{ 83};
\markcell{ 82}{ 84};
\markcell{ 82}{127};
\markcell{ 82}{128};
\markcell{ 83}{ 65};
\markcell{ 83}{ 68};
\markcell{ 83}{ 76};
\markcell{ 83}{ 78};
\markcell{ 83}{ 80};
\markcell{ 83}{ 81};
\markcell{ 83}{ 82};
\markcell{ 83}{ 83};
\markcell{ 83}{127};
\markcell{ 83}{128};
\markcell{ 84}{ 65};
\markcell{ 84}{ 68};
\markcell{ 84}{ 69};
\markcell{ 84}{ 76};
\markcell{ 84}{ 78};
\markcell{ 84}{ 79};
\markcell{ 84}{ 80};
\markcell{ 84}{ 81};
\markcell{ 84}{ 82};
\markcell{ 84}{ 83};
\markcell{ 84}{ 84};
\markcell{ 84}{127};
\markcell{ 84}{128};
\markcell{ 84}{129};
\markcell{ 85}{ 10};
\markcell{ 85}{ 13};
\markcell{ 85}{ 30};
\markcell{ 85}{ 33};
\markcell{ 85}{ 40};
\markcell{ 85}{ 85};
\markcell{ 85}{ 88};
\markcell{ 85}{ 89};
\markcell{ 85}{ 90};
\markcell{ 85}{100};
\markcell{ 85}{103};
\markcell{ 85}{130};
\markcell{ 86}{ 11};
\markcell{ 86}{ 31};
\markcell{ 86}{ 41};
\markcell{ 86}{ 86};
\markcell{ 86}{ 88};
\markcell{ 86}{ 89};
\markcell{ 86}{ 91};
\markcell{ 86}{ 93};
\markcell{ 86}{101};
\markcell{ 86}{131};
\markcell{ 87}{ 12};
\markcell{ 87}{ 32};
\markcell{ 87}{ 42};
\markcell{ 87}{ 43};
\markcell{ 87}{ 87};
\markcell{ 87}{ 88};
\markcell{ 87}{ 89};
\markcell{ 87}{ 92};
\markcell{ 87}{102};
\markcell{ 87}{132};
\markcell{ 87}{133};
\markcell{ 88}{ 10};
\markcell{ 88}{ 13};
\markcell{ 88}{ 30};
\markcell{ 88}{ 33};
\markcell{ 88}{ 42};
\markcell{ 88}{ 43};
\markcell{ 88}{ 85};
\markcell{ 88}{ 86};
\markcell{ 88}{ 87};
\markcell{ 88}{ 88};
\markcell{ 88}{ 91};
\markcell{ 88}{ 93};
\markcell{ 88}{100};
\markcell{ 88}{103};
\markcell{ 88}{132};
\markcell{ 88}{133};
\markcell{ 89}{ 10};
\markcell{ 89}{ 13};
\markcell{ 89}{ 14};
\markcell{ 89}{ 30};
\markcell{ 89}{ 33};
\markcell{ 89}{ 34};
\markcell{ 89}{ 42};
\markcell{ 89}{ 43};
\markcell{ 89}{ 44};
\markcell{ 89}{ 85};
\markcell{ 89}{ 86};
\markcell{ 89}{ 87};
\markcell{ 89}{ 88};
\markcell{ 89}{ 89};
\markcell{ 89}{ 91};
\markcell{ 89}{ 93};
\markcell{ 89}{ 94};
\markcell{ 89}{100};
\markcell{ 89}{103};
\markcell{ 89}{104};
\markcell{ 89}{132};
\markcell{ 89}{133};
\markcell{ 89}{134};
\markcell{ 90}{ 10};
\markcell{ 90}{ 13};
\markcell{ 90}{ 15};
\markcell{ 90}{ 18};
\markcell{ 90}{ 30};
\markcell{ 90}{ 33};
\markcell{ 90}{ 35};
\markcell{ 90}{ 38};
\markcell{ 90}{ 45};
\markcell{ 90}{ 85};
\markcell{ 90}{ 90};
\markcell{ 90}{ 93};
\markcell{ 90}{ 94};
\markcell{ 90}{ 95};
\markcell{ 90}{105};
\markcell{ 90}{108};
\markcell{ 90}{135};
\markcell{ 91}{ 11};
\markcell{ 91}{ 16};
\markcell{ 91}{ 31};
\markcell{ 91}{ 36};
\markcell{ 91}{ 46};
\markcell{ 91}{ 86};
\markcell{ 91}{ 88};
\markcell{ 91}{ 91};
\markcell{ 91}{ 93};
\markcell{ 91}{ 94};
\markcell{ 91}{ 96};
\markcell{ 91}{ 98};
\markcell{ 91}{106};
\markcell{ 91}{136};
\markcell{ 92}{ 12};
\markcell{ 92}{ 17};
\markcell{ 92}{ 32};
\markcell{ 92}{ 37};
\markcell{ 92}{ 47};
\markcell{ 92}{ 48};
\markcell{ 92}{ 87};
\markcell{ 92}{ 92};
\markcell{ 92}{ 93};
\markcell{ 92}{ 94};
\markcell{ 92}{ 97};
\markcell{ 92}{107};
\markcell{ 92}{137};
\markcell{ 92}{138};
\markcell{ 93}{ 10};
\markcell{ 93}{ 13};
\markcell{ 93}{ 15};
\markcell{ 93}{ 18};
\markcell{ 93}{ 30};
\markcell{ 93}{ 33};
\markcell{ 93}{ 35};
\markcell{ 93}{ 38};
\markcell{ 93}{ 47};
\markcell{ 93}{ 48};
\markcell{ 93}{ 86};
\markcell{ 93}{ 88};
\markcell{ 93}{ 90};
\markcell{ 93}{ 91};
\markcell{ 93}{ 92};
\markcell{ 93}{ 93};
\markcell{ 93}{ 96};
\markcell{ 93}{ 98};
\markcell{ 93}{105};
\markcell{ 93}{108};
\markcell{ 93}{137};
\markcell{ 93}{138};
\markcell{ 94}{ 10};
\markcell{ 94}{ 13};
\markcell{ 94}{ 14};
\markcell{ 94}{ 15};
\markcell{ 94}{ 18};
\markcell{ 94}{ 19};
\markcell{ 94}{ 30};
\markcell{ 94}{ 33};
\markcell{ 94}{ 34};
\markcell{ 94}{ 35};
\markcell{ 94}{ 38};
\markcell{ 94}{ 39};
\markcell{ 94}{ 47};
\markcell{ 94}{ 48};
\markcell{ 94}{ 49};
\markcell{ 94}{ 86};
\markcell{ 94}{ 88};
\markcell{ 94}{ 89};
\markcell{ 94}{ 90};
\markcell{ 94}{ 91};
\markcell{ 94}{ 92};
\markcell{ 94}{ 93};
\markcell{ 94}{ 94};
\markcell{ 94}{ 96};
\markcell{ 94}{ 98};
\markcell{ 94}{ 99};
\markcell{ 94}{105};
\markcell{ 94}{108};
\markcell{ 94}{109};
\markcell{ 94}{137};
\markcell{ 94}{138};
\markcell{ 94}{139};
\markcell{ 95}{ 15};
\markcell{ 95}{ 18};
\markcell{ 95}{ 35};
\markcell{ 95}{ 38};
\markcell{ 95}{ 50};
\markcell{ 95}{ 90};
\markcell{ 95}{ 95};
\markcell{ 95}{ 98};
\markcell{ 95}{ 99};
\markcell{ 95}{110};
\markcell{ 95}{113};
\markcell{ 95}{140};
\markcell{ 96}{ 16};
\markcell{ 96}{ 36};
\markcell{ 96}{ 51};
\markcell{ 96}{ 91};
\markcell{ 96}{ 93};
\markcell{ 96}{ 96};
\markcell{ 96}{ 98};
\markcell{ 96}{ 99};
\markcell{ 96}{111};
\markcell{ 96}{141};
\markcell{ 97}{ 17};
\markcell{ 97}{ 37};
\markcell{ 97}{ 52};
\markcell{ 97}{ 53};
\markcell{ 97}{ 92};
\markcell{ 97}{ 97};
\markcell{ 97}{ 98};
\markcell{ 97}{ 99};
\markcell{ 97}{112};
\markcell{ 97}{142};
\markcell{ 97}{143};
\markcell{ 98}{ 15};
\markcell{ 98}{ 18};
\markcell{ 98}{ 35};
\markcell{ 98}{ 38};
\markcell{ 98}{ 52};
\markcell{ 98}{ 53};
\markcell{ 98}{ 91};
\markcell{ 98}{ 93};
\markcell{ 98}{ 95};
\markcell{ 98}{ 96};
\markcell{ 98}{ 97};
\markcell{ 98}{ 98};
\markcell{ 98}{110};
\markcell{ 98}{113};
\markcell{ 98}{142};
\markcell{ 98}{143};
\markcell{ 99}{ 15};
\markcell{ 99}{ 18};
\markcell{ 99}{ 19};
\markcell{ 99}{ 35};
\markcell{ 99}{ 38};
\markcell{ 99}{ 39};
\markcell{ 99}{ 52};
\markcell{ 99}{ 53};
\markcell{ 99}{ 54};
\markcell{ 99}{ 91};
\markcell{ 99}{ 93};
\markcell{ 99}{ 94};
\markcell{ 99}{ 95};
\markcell{ 99}{ 96};
\markcell{ 99}{ 97};
\markcell{ 99}{ 98};
\markcell{ 99}{ 99};
\markcell{ 99}{110};
\markcell{ 99}{113};
\markcell{ 99}{114};
\markcell{ 99}{142};
\markcell{ 99}{143};
\markcell{ 99}{144};
\markcell{100}{ 55};
\markcell{100}{ 85};
\markcell{100}{ 88};
\markcell{100}{100};
\markcell{100}{103};
\markcell{100}{104};
\markcell{100}{105};
\markcell{100}{115};
\markcell{100}{118};
\markcell{100}{145};
\markcell{101}{ 56};
\markcell{101}{ 86};
\markcell{101}{101};
\markcell{101}{103};
\markcell{101}{104};
\markcell{101}{106};
\markcell{101}{108};
\markcell{101}{116};
\markcell{101}{146};
\markcell{102}{ 57};
\markcell{102}{ 58};
\markcell{102}{ 87};
\markcell{102}{102};
\markcell{102}{104};
\markcell{102}{107};
\markcell{102}{117};
\markcell{102}{147};
\markcell{102}{148};
\markcell{103}{ 57};
\markcell{103}{ 58};
\markcell{103}{ 85};
\markcell{103}{ 88};
\markcell{103}{100};
\markcell{103}{101};
\markcell{103}{103};
\markcell{103}{106};
\markcell{103}{108};
\markcell{103}{115};
\markcell{103}{118};
\markcell{103}{147};
\markcell{103}{148};
\markcell{104}{ 57};
\markcell{104}{ 58};
\markcell{104}{ 59};
\markcell{104}{ 85};
\markcell{104}{ 88};
\markcell{104}{ 89};
\markcell{104}{100};
\markcell{104}{101};
\markcell{104}{102};
\markcell{104}{103};
\markcell{104}{104};
\markcell{104}{106};
\markcell{104}{108};
\markcell{104}{109};
\markcell{104}{115};
\markcell{104}{118};
\markcell{104}{119};
\markcell{104}{147};
\markcell{104}{148};
\markcell{104}{149};
\markcell{105}{ 60};
\markcell{105}{ 90};
\markcell{105}{ 93};
\markcell{105}{100};
\markcell{105}{105};
\markcell{105}{109};
\markcell{105}{110};
\markcell{105}{120};
\markcell{105}{123};
\markcell{105}{150};
\markcell{106}{ 61};
\markcell{106}{ 91};
\markcell{106}{101};
\markcell{106}{103};
\markcell{106}{106};
\markcell{106}{108};
\markcell{106}{109};
\markcell{106}{111};
\markcell{106}{113};
\markcell{106}{121};
\markcell{106}{151};
\markcell{107}{ 62};
\markcell{107}{ 63};
\markcell{107}{ 92};
\markcell{107}{102};
\markcell{107}{107};
\markcell{107}{108};
\markcell{107}{109};
\markcell{107}{112};
\markcell{107}{122};
\markcell{107}{152};
\markcell{107}{153};
\markcell{108}{ 62};
\markcell{108}{ 63};
\markcell{108}{ 90};
\markcell{108}{ 93};
\markcell{108}{101};
\markcell{108}{103};
\markcell{108}{106};
\markcell{108}{107};
\markcell{108}{108};
\markcell{108}{111};
\markcell{108}{113};
\markcell{108}{120};
\markcell{108}{123};
\markcell{108}{152};
\markcell{108}{153};
\markcell{109}{ 62};
\markcell{109}{ 63};
\markcell{109}{ 64};
\markcell{109}{ 90};
\markcell{109}{ 93};
\markcell{109}{ 94};
\markcell{109}{101};
\markcell{109}{103};
\markcell{109}{104};
\markcell{109}{105};
\markcell{109}{106};
\markcell{109}{107};
\markcell{109}{108};
\markcell{109}{109};
\markcell{109}{111};
\markcell{109}{113};
\markcell{109}{114};
\markcell{109}{120};
\markcell{109}{123};
\markcell{109}{124};
\markcell{109}{152};
\markcell{109}{153};
\markcell{109}{154};
\markcell{110}{ 65};
\markcell{110}{ 95};
\markcell{110}{ 98};
\markcell{110}{105};
\markcell{110}{110};
\markcell{110}{113};
\markcell{110}{114};
\markcell{110}{125};
\markcell{110}{128};
\markcell{110}{155};
\markcell{111}{ 66};
\markcell{111}{ 96};
\markcell{111}{106};
\markcell{111}{108};
\markcell{111}{111};
\markcell{111}{113};
\markcell{111}{114};
\markcell{111}{126};
\markcell{111}{156};
\markcell{112}{ 67};
\markcell{112}{ 68};
\markcell{112}{ 97};
\markcell{112}{107};
\markcell{112}{112};
\markcell{112}{114};
\markcell{112}{127};
\markcell{112}{157};
\markcell{112}{158};
\markcell{113}{ 67};
\markcell{113}{ 68};
\markcell{113}{ 95};
\markcell{113}{ 98};
\markcell{113}{106};
\markcell{113}{108};
\markcell{113}{110};
\markcell{113}{111};
\markcell{113}{113};
\markcell{113}{125};
\markcell{113}{128};
\markcell{113}{157};
\markcell{113}{158};
\markcell{114}{ 67};
\markcell{114}{ 68};
\markcell{114}{ 69};
\markcell{114}{ 95};
\markcell{114}{ 98};
\markcell{114}{ 99};
\markcell{114}{106};
\markcell{114}{108};
\markcell{114}{109};
\markcell{114}{110};
\markcell{114}{111};
\markcell{114}{112};
\markcell{114}{113};
\markcell{114}{114};
\markcell{114}{125};
\markcell{114}{128};
\markcell{114}{129};
\markcell{114}{157};
\markcell{114}{158};
\markcell{114}{159};
\markcell{115}{ 70};
\markcell{115}{100};
\markcell{115}{103};
\markcell{115}{115};
\markcell{115}{118};
\markcell{115}{119};
\markcell{115}{120};
\markcell{115}{160};
\markcell{116}{ 71};
\markcell{116}{101};
\markcell{116}{116};
\markcell{116}{118};
\markcell{116}{119};
\markcell{116}{121};
\markcell{116}{123};
\markcell{116}{161};
\markcell{117}{ 72};
\markcell{117}{ 73};
\markcell{117}{102};
\markcell{117}{117};
\markcell{117}{118};
\markcell{117}{119};
\markcell{117}{122};
\markcell{117}{162};
\markcell{117}{163};
\markcell{118}{ 72};
\markcell{118}{ 73};
\markcell{118}{100};
\markcell{118}{103};
\markcell{118}{115};
\markcell{118}{116};
\markcell{118}{117};
\markcell{118}{118};
\markcell{118}{121};
\markcell{118}{123};
\markcell{118}{162};
\markcell{118}{163};
\markcell{119}{ 72};
\markcell{119}{ 73};
\markcell{119}{ 74};
\markcell{119}{100};
\markcell{119}{103};
\markcell{119}{104};
\markcell{119}{115};
\markcell{119}{116};
\markcell{119}{117};
\markcell{119}{118};
\markcell{119}{119};
\markcell{119}{121};
\markcell{119}{123};
\markcell{119}{124};
\markcell{119}{162};
\markcell{119}{163};
\markcell{119}{164};
\markcell{120}{ 75};
\markcell{120}{105};
\markcell{120}{108};
\markcell{120}{115};
\markcell{120}{120};
\markcell{120}{123};
\markcell{120}{124};
\markcell{120}{125};
\markcell{120}{165};
\markcell{121}{ 76};
\markcell{121}{106};
\markcell{121}{116};
\markcell{121}{118};
\markcell{121}{121};
\markcell{121}{123};
\markcell{121}{124};
\markcell{121}{126};
\markcell{121}{128};
\markcell{121}{166};
\markcell{122}{ 77};
\markcell{122}{ 78};
\markcell{122}{107};
\markcell{122}{117};
\markcell{122}{122};
\markcell{122}{123};
\markcell{122}{124};
\markcell{122}{127};
\markcell{122}{167};
\markcell{122}{168};
\markcell{123}{ 77};
\markcell{123}{ 78};
\markcell{123}{105};
\markcell{123}{108};
\markcell{123}{116};
\markcell{123}{118};
\markcell{123}{120};
\markcell{123}{121};
\markcell{123}{122};
\markcell{123}{123};
\markcell{123}{126};
\markcell{123}{128};
\markcell{123}{167};
\markcell{123}{168};
\markcell{124}{ 77};
\markcell{124}{ 78};
\markcell{124}{ 79};
\markcell{124}{105};
\markcell{124}{108};
\markcell{124}{109};
\markcell{124}{116};
\markcell{124}{118};
\markcell{124}{119};
\markcell{124}{120};
\markcell{124}{121};
\markcell{124}{122};
\markcell{124}{123};
\markcell{124}{124};
\markcell{124}{126};
\markcell{124}{128};
\markcell{124}{129};
\markcell{124}{167};
\markcell{124}{168};
\markcell{124}{169};
\markcell{125}{ 80};
\markcell{125}{110};
\markcell{125}{113};
\markcell{125}{120};
\markcell{125}{125};
\markcell{125}{128};
\markcell{125}{129};
\markcell{125}{170};
\markcell{126}{ 81};
\markcell{126}{111};
\markcell{126}{121};
\markcell{126}{123};
\markcell{126}{126};
\markcell{126}{128};
\markcell{126}{129};
\markcell{126}{171};
\markcell{127}{ 82};
\markcell{127}{ 83};
\markcell{127}{112};
\markcell{127}{122};
\markcell{127}{127};
\markcell{127}{128};
\markcell{127}{129};
\markcell{127}{172};
\markcell{127}{173};
\markcell{128}{ 82};
\markcell{128}{ 83};
\markcell{128}{110};
\markcell{128}{113};
\markcell{128}{121};
\markcell{128}{123};
\markcell{128}{125};
\markcell{128}{126};
\markcell{128}{127};
\markcell{128}{128};
\markcell{128}{172};
\markcell{128}{173};
\markcell{129}{ 82};
\markcell{129}{ 83};
\markcell{129}{ 84};
\markcell{129}{110};
\markcell{129}{113};
\markcell{129}{114};
\markcell{129}{121};
\markcell{129}{123};
\markcell{129}{124};
\markcell{129}{125};
\markcell{129}{126};
\markcell{129}{127};
\markcell{129}{128};
\markcell{129}{129};
\markcell{129}{172};
\markcell{129}{173};
\markcell{129}{174};
\markcell{130}{ 30};
\markcell{130}{ 33};
\markcell{130}{ 85};
\markcell{130}{130};
\markcell{130}{133};
\markcell{130}{134};
\markcell{130}{135};
\markcell{130}{145};
\markcell{130}{148};
\markcell{131}{ 31};
\markcell{131}{ 86};
\markcell{131}{131};
\markcell{131}{133};
\markcell{131}{134};
\markcell{131}{136};
\markcell{131}{138};
\markcell{131}{146};
\markcell{132}{ 32};
\markcell{132}{ 87};
\markcell{132}{ 88};
\markcell{132}{132};
\markcell{132}{133};
\markcell{132}{134};
\markcell{132}{137};
\markcell{132}{147};
\markcell{133}{ 30};
\markcell{133}{ 33};
\markcell{133}{ 87};
\markcell{133}{ 88};
\markcell{133}{130};
\markcell{133}{131};
\markcell{133}{132};
\markcell{133}{133};
\markcell{133}{136};
\markcell{133}{138};
\markcell{133}{145};
\markcell{133}{148};
\markcell{134}{ 30};
\markcell{134}{ 33};
\markcell{134}{ 34};
\markcell{134}{ 87};
\markcell{134}{ 88};
\markcell{134}{ 89};
\markcell{134}{130};
\markcell{134}{131};
\markcell{134}{132};
\markcell{134}{133};
\markcell{134}{134};
\markcell{134}{136};
\markcell{134}{138};
\markcell{134}{139};
\markcell{134}{145};
\markcell{134}{148};
\markcell{134}{149};
\markcell{135}{ 30};
\markcell{135}{ 33};
\markcell{135}{ 35};
\markcell{135}{ 38};
\markcell{135}{ 90};
\markcell{135}{130};
\markcell{135}{135};
\markcell{135}{138};
\markcell{135}{139};
\markcell{135}{140};
\markcell{135}{150};
\markcell{135}{153};
\markcell{136}{ 31};
\markcell{136}{ 36};
\markcell{136}{ 91};
\markcell{136}{131};
\markcell{136}{133};
\markcell{136}{136};
\markcell{136}{138};
\markcell{136}{139};
\markcell{136}{141};
\markcell{136}{143};
\markcell{136}{151};
\markcell{137}{ 32};
\markcell{137}{ 37};
\markcell{137}{ 92};
\markcell{137}{ 93};
\markcell{137}{132};
\markcell{137}{137};
\markcell{137}{138};
\markcell{137}{139};
\markcell{137}{142};
\markcell{137}{152};
\markcell{138}{ 30};
\markcell{138}{ 33};
\markcell{138}{ 35};
\markcell{138}{ 38};
\markcell{138}{ 92};
\markcell{138}{ 93};
\markcell{138}{131};
\markcell{138}{133};
\markcell{138}{135};
\markcell{138}{136};
\markcell{138}{137};
\markcell{138}{138};
\markcell{138}{141};
\markcell{138}{143};
\markcell{138}{150};
\markcell{138}{153};
\markcell{139}{ 30};
\markcell{139}{ 33};
\markcell{139}{ 34};
\markcell{139}{ 35};
\markcell{139}{ 38};
\markcell{139}{ 39};
\markcell{139}{ 92};
\markcell{139}{ 93};
\markcell{139}{ 94};
\markcell{139}{131};
\markcell{139}{133};
\markcell{139}{134};
\markcell{139}{135};
\markcell{139}{136};
\markcell{139}{137};
\markcell{139}{138};
\markcell{139}{139};
\markcell{139}{141};
\markcell{139}{143};
\markcell{139}{144};
\markcell{139}{150};
\markcell{139}{153};
\markcell{139}{154};
\markcell{140}{ 35};
\markcell{140}{ 38};
\markcell{140}{ 95};
\markcell{140}{135};
\markcell{140}{140};
\markcell{140}{143};
\markcell{140}{144};
\markcell{140}{155};
\markcell{140}{158};
\markcell{141}{ 36};
\markcell{141}{ 96};
\markcell{141}{136};
\markcell{141}{138};
\markcell{141}{141};
\markcell{141}{143};
\markcell{141}{144};
\markcell{141}{156};
\markcell{142}{ 37};
\markcell{142}{ 97};
\markcell{142}{ 98};
\markcell{142}{137};
\markcell{142}{142};
\markcell{142}{143};
\markcell{142}{144};
\markcell{142}{157};
\markcell{143}{ 35};
\markcell{143}{ 38};
\markcell{143}{ 97};
\markcell{143}{ 98};
\markcell{143}{136};
\markcell{143}{138};
\markcell{143}{140};
\markcell{143}{141};
\markcell{143}{142};
\markcell{143}{143};
\markcell{143}{155};
\markcell{143}{158};
\markcell{144}{ 35};
\markcell{144}{ 38};
\markcell{144}{ 39};
\markcell{144}{ 97};
\markcell{144}{ 98};
\markcell{144}{ 99};
\markcell{144}{136};
\markcell{144}{138};
\markcell{144}{139};
\markcell{144}{140};
\markcell{144}{141};
\markcell{144}{142};
\markcell{144}{143};
\markcell{144}{144};
\markcell{144}{155};
\markcell{144}{158};
\markcell{144}{159};
\markcell{145}{100};
\markcell{145}{130};
\markcell{145}{133};
\markcell{145}{145};
\markcell{145}{148};
\markcell{145}{149};
\markcell{145}{150};
\markcell{145}{160};
\markcell{145}{163};
\markcell{146}{101};
\markcell{146}{131};
\markcell{146}{146};
\markcell{146}{148};
\markcell{146}{149};
\markcell{146}{151};
\markcell{146}{153};
\markcell{146}{161};
\markcell{147}{102};
\markcell{147}{103};
\markcell{147}{132};
\markcell{147}{147};
\markcell{147}{148};
\markcell{147}{149};
\markcell{147}{152};
\markcell{147}{162};
\markcell{148}{102};
\markcell{148}{103};
\markcell{148}{130};
\markcell{148}{133};
\markcell{148}{145};
\markcell{148}{146};
\markcell{148}{147};
\markcell{148}{148};
\markcell{148}{151};
\markcell{148}{153};
\markcell{148}{160};
\markcell{148}{163};
\markcell{149}{102};
\markcell{149}{103};
\markcell{149}{104};
\markcell{149}{130};
\markcell{149}{133};
\markcell{149}{134};
\markcell{149}{145};
\markcell{149}{146};
\markcell{149}{147};
\markcell{149}{148};
\markcell{149}{149};
\markcell{149}{151};
\markcell{149}{153};
\markcell{149}{154};
\markcell{149}{160};
\markcell{149}{163};
\markcell{149}{164};
\markcell{150}{105};
\markcell{150}{135};
\markcell{150}{138};
\markcell{150}{145};
\markcell{150}{150};
\markcell{150}{153};
\markcell{150}{154};
\markcell{150}{155};
\markcell{150}{165};
\markcell{150}{168};
\markcell{151}{106};
\markcell{151}{136};
\markcell{151}{146};
\markcell{151}{148};
\markcell{151}{151};
\markcell{151}{154};
\markcell{151}{156};
\markcell{151}{158};
\markcell{151}{166};
\markcell{152}{107};
\markcell{152}{108};
\markcell{152}{137};
\markcell{152}{147};
\markcell{152}{152};
\markcell{152}{153};
\markcell{152}{154};
\markcell{152}{157};
\markcell{152}{167};
\markcell{153}{107};
\markcell{153}{108};
\markcell{153}{135};
\markcell{153}{138};
\markcell{153}{146};
\markcell{153}{148};
\markcell{153}{150};
\markcell{153}{152};
\markcell{153}{153};
\markcell{153}{156};
\markcell{153}{158};
\markcell{153}{165};
\markcell{153}{168};
\markcell{154}{107};
\markcell{154}{108};
\markcell{154}{109};
\markcell{154}{135};
\markcell{154}{138};
\markcell{154}{139};
\markcell{154}{146};
\markcell{154}{148};
\markcell{154}{149};
\markcell{154}{150};
\markcell{154}{151};
\markcell{154}{152};
\markcell{154}{153};
\markcell{154}{154};
\markcell{154}{156};
\markcell{154}{158};
\markcell{154}{159};
\markcell{154}{165};
\markcell{154}{168};
\markcell{154}{169};
\markcell{155}{110};
\markcell{155}{140};
\markcell{155}{143};
\markcell{155}{150};
\markcell{155}{155};
\markcell{155}{158};
\markcell{155}{159};
\markcell{155}{170};
\markcell{155}{173};
\markcell{156}{111};
\markcell{156}{141};
\markcell{156}{151};
\markcell{156}{153};
\markcell{156}{156};
\markcell{156}{158};
\markcell{156}{159};
\markcell{156}{171};
\markcell{157}{112};
\markcell{157}{113};
\markcell{157}{142};
\markcell{157}{152};
\markcell{157}{157};
\markcell{157}{158};
\markcell{157}{159};
\markcell{157}{172};
\markcell{158}{112};
\markcell{158}{113};
\markcell{158}{140};
\markcell{158}{143};
\markcell{158}{151};
\markcell{158}{153};
\markcell{158}{155};
\markcell{158}{156};
\markcell{158}{157};
\markcell{158}{158};
\markcell{158}{170};
\markcell{158}{173};
\markcell{159}{112};
\markcell{159}{113};
\markcell{159}{114};
\markcell{159}{140};
\markcell{159}{143};
\markcell{159}{144};
\markcell{159}{151};
\markcell{159}{153};
\markcell{159}{154};
\markcell{159}{155};
\markcell{159}{156};
\markcell{159}{157};
\markcell{159}{158};
\markcell{159}{159};
\markcell{159}{170};
\markcell{159}{173};
\markcell{159}{174};
\markcell{160}{115};
\markcell{160}{145};
\markcell{160}{148};
\markcell{160}{160};
\markcell{160}{163};
\markcell{160}{164};
\markcell{160}{165};
\markcell{161}{116};
\markcell{161}{146};
\markcell{161}{161};
\markcell{161}{163};
\markcell{161}{164};
\markcell{161}{166};
\markcell{161}{168};
\markcell{162}{117};
\markcell{162}{118};
\markcell{162}{147};
\markcell{162}{162};
\markcell{162}{163};
\markcell{162}{164};
\markcell{162}{167};
\markcell{163}{117};
\markcell{163}{118};
\markcell{163}{145};
\markcell{163}{148};
\markcell{163}{160};
\markcell{163}{161};
\markcell{163}{162};
\markcell{163}{163};
\markcell{163}{166};
\markcell{163}{168};
\markcell{164}{117};
\markcell{164}{118};
\markcell{164}{119};
\markcell{164}{145};
\markcell{164}{148};
\markcell{164}{149};
\markcell{164}{160};
\markcell{164}{161};
\markcell{164}{162};
\markcell{164}{163};
\markcell{164}{164};
\markcell{164}{166};
\markcell{164}{168};
\markcell{164}{169};
\markcell{165}{120};
\markcell{165}{150};
\markcell{165}{153};
\markcell{165}{160};
\markcell{165}{165};
\markcell{165}{168};
\markcell{165}{169};
\markcell{165}{170};
\markcell{166}{121};
\markcell{166}{151};
\markcell{166}{161};
\markcell{166}{163};
\markcell{166}{166};
\markcell{166}{168};
\markcell{166}{169};
\markcell{166}{171};
\markcell{166}{173};
\markcell{167}{122};
\markcell{167}{123};
\markcell{167}{152};
\markcell{167}{162};
\markcell{167}{167};
\markcell{167}{168};
\markcell{167}{169};
\markcell{167}{172};
\markcell{168}{122};
\markcell{168}{123};
\markcell{168}{150};
\markcell{168}{153};
\markcell{168}{161};
\markcell{168}{163};
\markcell{168}{165};
\markcell{168}{166};
\markcell{168}{167};
\markcell{168}{168};
\markcell{168}{171};
\markcell{168}{173};
\markcell{169}{122};
\markcell{169}{123};
\markcell{169}{124};
\markcell{169}{150};
\markcell{169}{153};
\markcell{169}{154};
\markcell{169}{161};
\markcell{169}{163};
\markcell{169}{164};
\markcell{169}{165};
\markcell{169}{166};
\markcell{169}{167};
\markcell{169}{168};
\markcell{169}{169};
\markcell{169}{171};
\markcell{169}{173};
\markcell{169}{174};
\markcell{170}{125};
\markcell{170}{155};
\markcell{170}{158};
\markcell{170}{165};
\markcell{170}{170};
\markcell{170}{173};
\markcell{170}{174};
\markcell{171}{126};
\markcell{171}{156};
\markcell{171}{166};
\markcell{171}{168};
\markcell{171}{171};
\markcell{171}{173};
\markcell{171}{174};
\markcell{172}{127};
\markcell{172}{128};
\markcell{172}{157};
\markcell{172}{167};
\markcell{172}{172};
\markcell{172}{173};
\markcell{172}{174};
\markcell{173}{127};
\markcell{173}{128};
\markcell{173}{155};
\markcell{173}{158};
\markcell{173}{166};
\markcell{173}{168};
\markcell{173}{170};
\markcell{173}{171};
\markcell{173}{172};
\markcell{173}{173};
\markcell{174}{127};
\markcell{174}{128};
\markcell{174}{129};
\markcell{174}{155};
\markcell{174}{158};
\markcell{174}{159};
\markcell{174}{166};
\markcell{174}{168};
\markcell{174}{169};
\markcell{174}{170};
\markcell{174}{171};
\markcell{174}{172};
\markcell{174}{173};
\markcell{174}{174};
    %BLOCK 1

%   \draw[ultra thick] (0,\xStep*\nStep) -- (8*\xStep,\xStep*\nStep) -- (8*\xStep,\xStep*\nStep-8*\xStep) -- (0,\xStep*\nStep-8*\xStep) -- cycle;
%   \draw[ultra thick] (8*\xStep,\xStep*\nStep) -- (35*\xStep,\xStep*\nStep) -- (35*\xStep,\xStep*\nStep-8*\xStep) -- (8*\xStep,\xStep*\nStep-8*\xStep) -- cycle;
%   \draw[ultra thick] (8*\xStep,\xStep*\nStep-8*\xStep) -- (35*\xStep,\xStep*\nStep-8*\xStep) -- (35*\xStep,0) -- (8*\xStep,0) -- cycle;
%   \draw[ultra thick] (0,\xStep*\nStep-8*\xStep) -- (8*\xStep,\xStep*\nStep-8*\xStep) -- (8*\xStep,0) -- (0*\xStep,0) -- cycle;

\end{tikzpicture}

  \caption{Non-zero structure of the linear system used in the coupled solution algorithm for a block-structured grid consisting of one $2\times2\times2$ cell block and one $3\times3\times3$ cell block. The variables have been interlaced and the matrix consists of blocks as shown in figure \ref{fig:cpldassemble}.}
  \label{fig:interlacemat}
\end{figure}

\begin{figure}
  \centering
   \newcommand*{\xMin}{0}%
\newcommand*{\xMax}{12}%
\newcommand*{\xStep}{0.075}%
\newcommand*{\nStep}{175}
\newcommand*{\yMin}{0}%
\newcommand*{\yMax}{1}%
\newcommand\markcell[2] {
  \fill[gray,draw=black] (#2*\xStep,\xStep*\nStep-#1*\xStep) -- (#2*\xStep+\xStep,\xStep*\nStep-#1*\xStep) -- (#2*\xStep+\xStep,\xStep*\nStep-#1*\xStep-\xStep) -- (#2*\xStep,\xStep*\nStep-#1*\xStep-\xStep) -- cycle;
  }

\newcommand\markcellblue[2] {
  \fill[orange,draw=black] (#2*\xStep,\xStep*\nStep-#1*\xStep) -- (#2*\xStep+\xStep,\xStep*\nStep-#1*\xStep) -- (#2*\xStep+\xStep,\xStep*\nStep-#1*\xStep-\xStep) -- (#2*\xStep,\xStep*\nStep-#1*\xStep-\xStep) -- cycle;
  }

\begin{tikzpicture}

    %BACKGROUND COLORING
    %\fill[gray!40] (0,\xStep*\nStep-8*\xStep) -- (8*\xStep,\xStep*\nStep-8*\xStep) -- (8*\xStep,0) -- (0*\xStep,0) -- cycle;
    %\fill[gray!40] (8*\xStep,\xStep*\nStep) -- (35*\xStep,\xStep*\nStep) -- (35*\xStep,\xStep*\nStep-8*\xStep) -- (8*\xStep,\xStep*\nStep-8*\xStep) -- cycle;

    \foreach \j in {0,...,5} {
    \draw [very thin,gray] (0,\j*\xStep*35) -- (\xStep*\nStep,\j*\xStep*35)  ;
    \draw [very thin,gray] (\j*\xStep*35,0) -- (\j*\xStep*35,\xStep*\nStep)  ;
    }

\markcell{  0}{  0};
\markcell{  0}{  1};
\markcell{  0}{  2};
\markcell{  0}{  4};
\markcell{  0}{105};
\markcell{  0}{107};
\markcell{  0}{140};
\markcell{  1}{  0};
\markcell{  1}{  1};
\markcell{  1}{  3};
\markcell{  1}{  5};
\markcell{  1}{106};
\markcell{  1}{108};
\markcell{  1}{141};
\markcell{  2}{  0};
\markcell{  2}{  2};
\markcell{  2}{  3};
\markcell{  2}{  6};
\markcell{  2}{  8};
\markcell{  2}{  9};
\markcell{  2}{ 17};
\markcell{  2}{ 18};
\markcell{  2}{105};
\markcell{  2}{107};
\markcell{  2}{113};
\markcell{  2}{114};
\markcell{  2}{122};
\markcell{  2}{123};
\markcell{  2}{142};
\markcell{  3}{  1};
\markcell{  3}{  2};
\markcell{  3}{  3};
\markcell{  3}{  7};
\markcell{  3}{  9};
\markcell{  3}{ 10};
\markcell{  3}{ 18};
\markcell{  3}{ 19};
\markcell{  3}{106};
\markcell{  3}{108};
\markcell{  3}{114};
\markcell{  3}{115};
\markcell{  3}{123};
\markcell{  3}{124};
\markcell{  3}{143};
\markcell{  4}{  0};
\markcell{  4}{  4};
\markcell{  4}{  5};
\markcell{  4}{  6};
\markcell{  4}{109};
\markcell{  4}{111};
\markcell{  4}{144};
\markcell{  5}{  1};
\markcell{  5}{  4};
\markcell{  5}{  5};
\markcell{  5}{  7};
\markcell{  5}{110};
\markcell{  5}{112};
\markcell{  5}{145};
\markcell{  6}{  2};
\markcell{  6}{  4};
\markcell{  6}{  6};
\markcell{  6}{  7};
\markcell{  6}{ 17};
\markcell{  6}{ 18};
\markcell{  6}{ 26};
\markcell{  6}{ 27};
\markcell{  6}{109};
\markcell{  6}{111};
\markcell{  6}{122};
\markcell{  6}{123};
\markcell{  6}{131};
\markcell{  6}{132};
\markcell{  6}{146};
\markcell{  7}{  3};
\markcell{  7}{  5};
\markcell{  7}{  6};
\markcell{  7}{  7};
\markcell{  7}{ 18};
\markcell{  7}{ 19};
\markcell{  7}{ 27};
\markcell{  7}{ 28};
\markcell{  7}{110};
\markcell{  7}{112};
\markcell{  7}{123};
\markcell{  7}{124};
\markcell{  7}{132};
\markcell{  7}{133};
\markcell{  7}{147};
\markcell{  8}{  2};
\markcell{  8}{  8};
\markcell{  8}{  9};
\markcell{  8}{ 11};
\markcell{  8}{ 17};
\markcell{  8}{107};
\markcell{  8}{113};
\markcell{  8}{116};
\markcell{  8}{148};
\markcell{  9}{  2};
\markcell{  9}{  3};
\markcell{  9}{  8};
\markcell{  9}{  9};
\markcell{  9}{ 10};
\markcell{  9}{ 12};
\markcell{  9}{ 18};
\markcell{  9}{107};
\markcell{  9}{108};
\markcell{  9}{114};
\markcell{  9}{117};
\markcell{  9}{149};
\markcell{ 10}{  3};
\markcell{ 10}{  9};
\markcell{ 10}{ 10};
\markcell{ 10}{ 13};
\markcell{ 10}{ 19};
\markcell{ 10}{108};
\markcell{ 10}{115};
\markcell{ 10}{118};
\markcell{ 10}{150};
\markcell{ 11}{  8};
\markcell{ 11}{ 11};
\markcell{ 11}{ 12};
\markcell{ 11}{ 14};
\markcell{ 11}{ 20};
\markcell{ 11}{113};
\markcell{ 11}{116};
\markcell{ 11}{119};
\markcell{ 11}{151};
\markcell{ 12}{  9};
\markcell{ 12}{ 11};
\markcell{ 12}{ 12};
\markcell{ 12}{ 13};
\markcell{ 12}{ 15};
\markcell{ 12}{ 21};
\markcell{ 12}{114};
\markcell{ 12}{117};
\markcell{ 12}{120};
\markcell{ 12}{152};
\markcell{ 13}{ 10};
\markcell{ 13}{ 12};
\markcell{ 13}{ 13};
\markcell{ 13}{ 16};
\markcell{ 13}{ 22};
\markcell{ 13}{115};
\markcell{ 13}{118};
\markcell{ 13}{121};
\markcell{ 13}{153};
\markcell{ 14}{ 11};
\markcell{ 14}{ 14};
\markcell{ 14}{ 15};
\markcell{ 14}{ 23};
\markcell{ 14}{116};
\markcell{ 14}{119};
\markcell{ 14}{154};
\markcell{ 15}{ 12};
\markcell{ 15}{ 14};
\markcell{ 15}{ 15};
\markcell{ 15}{ 16};
\markcell{ 15}{ 24};
\markcell{ 15}{117};
\markcell{ 15}{120};
\markcell{ 15}{155};
\markcell{ 16}{ 13};
\markcell{ 16}{ 15};
\markcell{ 16}{ 16};
\markcell{ 16}{ 25};
\markcell{ 16}{118};
\markcell{ 16}{121};
\markcell{ 16}{156};
\markcell{ 17}{  2};
\markcell{ 17}{  6};
\markcell{ 17}{  8};
\markcell{ 17}{ 17};
\markcell{ 17}{ 18};
\markcell{ 17}{ 20};
\markcell{ 17}{ 26};
\markcell{ 17}{107};
\markcell{ 17}{111};
\markcell{ 17}{122};
\markcell{ 17}{125};
\markcell{ 17}{157};
\markcell{ 18}{  2};
\markcell{ 18}{  3};
\markcell{ 18}{  6};
\markcell{ 18}{  7};
\markcell{ 18}{  9};
\markcell{ 18}{ 17};
\markcell{ 18}{ 18};
\markcell{ 18}{ 19};
\markcell{ 18}{ 21};
\markcell{ 18}{ 27};
\markcell{ 18}{107};
\markcell{ 18}{108};
\markcell{ 18}{111};
\markcell{ 18}{112};
\markcell{ 18}{123};
\markcell{ 18}{126};
\markcell{ 18}{158};
\markcell{ 19}{  3};
\markcell{ 19}{  7};
\markcell{ 19}{ 10};
\markcell{ 19}{ 18};
\markcell{ 19}{ 19};
\markcell{ 19}{ 22};
\markcell{ 19}{ 28};
\markcell{ 19}{108};
\markcell{ 19}{112};
\markcell{ 19}{124};
\markcell{ 19}{127};
\markcell{ 19}{159};
\markcell{ 20}{ 11};
\markcell{ 20}{ 17};
\markcell{ 20}{ 20};
\markcell{ 20}{ 21};
\markcell{ 20}{ 23};
\markcell{ 20}{ 29};
\markcell{ 20}{122};
\markcell{ 20}{125};
\markcell{ 20}{128};
\markcell{ 20}{160};
\markcell{ 21}{ 12};
\markcell{ 21}{ 18};
\markcell{ 21}{ 20};
\markcell{ 21}{ 21};
\markcell{ 21}{ 22};
\markcell{ 21}{ 24};
\markcell{ 21}{ 30};
\markcell{ 21}{123};
\markcell{ 21}{129};
\markcell{ 21}{161};
\markcell{ 22}{ 13};
\markcell{ 22}{ 19};
\markcell{ 22}{ 21};
\markcell{ 22}{ 22};
\markcell{ 22}{ 25};
\markcell{ 22}{ 31};
\markcell{ 22}{124};
\markcell{ 22}{127};
\markcell{ 22}{130};
\markcell{ 22}{162};
\markcell{ 23}{ 14};
\markcell{ 23}{ 20};
\markcell{ 23}{ 23};
\markcell{ 23}{ 24};
\markcell{ 23}{ 32};
\markcell{ 23}{125};
\markcell{ 23}{128};
\markcell{ 23}{163};
\markcell{ 24}{ 15};
\markcell{ 24}{ 21};
\markcell{ 24}{ 23};
\markcell{ 24}{ 24};
\markcell{ 24}{ 25};
\markcell{ 24}{ 33};
\markcell{ 24}{126};
\markcell{ 24}{129};
\markcell{ 24}{164};
\markcell{ 25}{ 16};
\markcell{ 25}{ 22};
\markcell{ 25}{ 24};
\markcell{ 25}{ 25};
\markcell{ 25}{ 34};
\markcell{ 25}{127};
\markcell{ 25}{130};
\markcell{ 25}{165};
\markcell{ 26}{  6};
\markcell{ 26}{ 17};
\markcell{ 26}{ 26};
\markcell{ 26}{ 27};
\markcell{ 26}{ 29};
\markcell{ 26}{111};
\markcell{ 26}{131};
\markcell{ 26}{134};
\markcell{ 26}{166};
\markcell{ 27}{  6};
\markcell{ 27}{  7};
\markcell{ 27}{ 18};
\markcell{ 27}{ 26};
\markcell{ 27}{ 27};
\markcell{ 27}{ 28};
\markcell{ 27}{ 30};
\markcell{ 27}{111};
\markcell{ 27}{112};
\markcell{ 27}{132};
\markcell{ 27}{135};
\markcell{ 27}{167};
\markcell{ 28}{  7};
\markcell{ 28}{ 19};
\markcell{ 28}{ 27};
\markcell{ 28}{ 28};
\markcell{ 28}{ 31};
\markcell{ 28}{112};
\markcell{ 28}{133};
\markcell{ 28}{136};
\markcell{ 28}{168};
\markcell{ 29}{ 20};
\markcell{ 29}{ 26};
\markcell{ 29}{ 29};
\markcell{ 29}{ 30};
\markcell{ 29}{ 32};
\markcell{ 29}{131};
\markcell{ 29}{134};
\markcell{ 29}{137};
\markcell{ 29}{169};
\markcell{ 30}{ 21};
\markcell{ 30}{ 27};
\markcell{ 30}{ 29};
\markcell{ 30}{ 30};
\markcell{ 30}{ 31};
\markcell{ 30}{ 33};
\markcell{ 30}{132};
\markcell{ 30}{135};
\markcell{ 30}{138};
\markcell{ 30}{170};
\markcell{ 31}{ 22};
\markcell{ 31}{ 28};
\markcell{ 31}{ 30};
\markcell{ 31}{ 31};
\markcell{ 31}{ 34};
\markcell{ 31}{133};
\markcell{ 31}{136};
\markcell{ 31}{139};
\markcell{ 31}{171};
\markcell{ 32}{ 23};
\markcell{ 32}{ 29};
\markcell{ 32}{ 32};
\markcell{ 32}{ 33};
\markcell{ 32}{134};
\markcell{ 32}{137};
\markcell{ 32}{172};
\markcell{ 33}{ 24};
\markcell{ 33}{ 30};
\markcell{ 33}{ 32};
\markcell{ 33}{ 33};
\markcell{ 33}{ 34};
\markcell{ 33}{135};
\markcell{ 33}{138};
\markcell{ 33}{173};
\markcell{ 34}{ 25};
\markcell{ 34}{ 31};
\markcell{ 34}{ 33};
\markcell{ 34}{ 34};
\markcell{ 34}{136};
\markcell{ 34}{139};
\markcell{ 34}{174};
\markcell{ 35}{ 35};
\markcell{ 35}{ 36};
\markcell{ 35}{ 37};
\markcell{ 35}{ 39};
\markcell{ 35}{105};
\markcell{ 35}{106};
\markcell{ 35}{140};
\markcell{ 36}{ 35};
\markcell{ 36}{ 36};
\markcell{ 36}{ 38};
\markcell{ 36}{ 40};
\markcell{ 36}{105};
\markcell{ 36}{106};
\markcell{ 36}{141};
\markcell{ 37}{ 35};
\markcell{ 37}{ 37};
\markcell{ 37}{ 38};
\markcell{ 37}{ 41};
\markcell{ 37}{ 43};
\markcell{ 37}{ 44};
\markcell{ 37}{ 52};
\markcell{ 37}{ 53};
\markcell{ 37}{107};
\markcell{ 37}{108};
\markcell{ 37}{142};
\markcell{ 38}{ 36};
\markcell{ 38}{ 37};
\markcell{ 38}{ 38};
\markcell{ 38}{ 42};
\markcell{ 38}{ 44};
\markcell{ 38}{ 45};
\markcell{ 38}{ 53};
\markcell{ 38}{ 54};
\markcell{ 38}{107};
\markcell{ 38}{108};
\markcell{ 38}{143};
\markcell{ 39}{ 35};
\markcell{ 39}{ 39};
\markcell{ 39}{ 40};
\markcell{ 39}{ 41};
\markcell{ 39}{109};
\markcell{ 39}{110};
\markcell{ 39}{144};
\markcell{ 40}{ 36};
\markcell{ 40}{ 39};
\markcell{ 40}{ 40};
\markcell{ 40}{ 42};
\markcell{ 40}{109};
\markcell{ 40}{110};
\markcell{ 40}{145};
\markcell{ 41}{ 37};
\markcell{ 41}{ 39};
\markcell{ 41}{ 41};
\markcell{ 41}{ 42};
\markcell{ 41}{ 52};
\markcell{ 41}{ 53};
\markcell{ 41}{ 61};
\markcell{ 41}{ 62};
\markcell{ 41}{111};
\markcell{ 41}{112};
\markcell{ 41}{146};
\markcell{ 42}{ 38};
\markcell{ 42}{ 40};
\markcell{ 42}{ 41};
\markcell{ 42}{ 42};
\markcell{ 42}{ 53};
\markcell{ 42}{ 54};
\markcell{ 42}{ 62};
\markcell{ 42}{ 63};
\markcell{ 42}{111};
\markcell{ 42}{112};
\markcell{ 42}{147};
\markcell{ 43}{ 37};
\markcell{ 43}{ 43};
\markcell{ 43}{ 44};
\markcell{ 43}{ 46};
\markcell{ 43}{ 52};
\markcell{ 43}{113};
\markcell{ 43}{114};
\markcell{ 43}{148};
\markcell{ 44}{ 37};
\markcell{ 44}{ 38};
\markcell{ 44}{ 43};
\markcell{ 44}{ 44};
\markcell{ 44}{ 45};
\markcell{ 44}{ 47};
\markcell{ 44}{ 53};
\markcell{ 44}{113};
\markcell{ 44}{114};
\markcell{ 44}{115};
\markcell{ 44}{149};
\markcell{ 45}{ 38};
\markcell{ 45}{ 44};
\markcell{ 45}{ 45};
\markcell{ 45}{ 48};
\markcell{ 45}{ 54};
\markcell{ 45}{114};
\markcell{ 45}{115};
\markcell{ 45}{150};
\markcell{ 46}{ 43};
\markcell{ 46}{ 46};
\markcell{ 46}{ 47};
\markcell{ 46}{ 49};
\markcell{ 46}{ 55};
\markcell{ 46}{116};
\markcell{ 46}{117};
\markcell{ 46}{151};
\markcell{ 47}{ 44};
\markcell{ 47}{ 46};
\markcell{ 47}{ 47};
\markcell{ 47}{ 48};
\markcell{ 47}{ 50};
\markcell{ 47}{ 56};
\markcell{ 47}{116};
\markcell{ 47}{118};
\markcell{ 47}{152};
\markcell{ 48}{ 45};
\markcell{ 48}{ 47};
\markcell{ 48}{ 48};
\markcell{ 48}{ 51};
\markcell{ 48}{ 57};
\markcell{ 48}{117};
\markcell{ 48}{118};
\markcell{ 48}{153};
\markcell{ 49}{ 46};
\markcell{ 49}{ 49};
\markcell{ 49}{ 50};
\markcell{ 49}{ 58};
\markcell{ 49}{119};
\markcell{ 49}{120};
\markcell{ 49}{154};
\markcell{ 50}{ 47};
\markcell{ 50}{ 49};
\markcell{ 50}{ 50};
\markcell{ 50}{ 51};
\markcell{ 50}{ 59};
\markcell{ 50}{119};
\markcell{ 50}{120};
\markcell{ 50}{121};
\markcell{ 50}{155};
\markcell{ 51}{ 48};
\markcell{ 51}{ 50};
\markcell{ 51}{ 51};
\markcell{ 51}{ 60};
\markcell{ 51}{120};
\markcell{ 51}{121};
\markcell{ 51}{156};
\markcell{ 52}{ 37};
\markcell{ 52}{ 41};
\markcell{ 52}{ 43};
\markcell{ 52}{ 52};
\markcell{ 52}{ 53};
\markcell{ 52}{ 55};
\markcell{ 52}{ 61};
\markcell{ 52}{122};
\markcell{ 52}{123};
\markcell{ 52}{157};
\markcell{ 53}{ 37};
\markcell{ 53}{ 38};
\markcell{ 53}{ 41};
\markcell{ 53}{ 42};
\markcell{ 53}{ 44};
\markcell{ 53}{ 52};
\markcell{ 53}{ 53};
\markcell{ 53}{ 54};
\markcell{ 53}{ 56};
\markcell{ 53}{ 62};
\markcell{ 53}{122};
\markcell{ 53}{123};
\markcell{ 53}{124};
\markcell{ 53}{158};
\markcell{ 54}{ 38};
\markcell{ 54}{ 42};
\markcell{ 54}{ 45};
\markcell{ 54}{ 53};
\markcell{ 54}{ 54};
\markcell{ 54}{ 57};
\markcell{ 54}{ 63};
\markcell{ 54}{123};
\markcell{ 54}{124};
\markcell{ 54}{159};
\markcell{ 55}{ 46};
\markcell{ 55}{ 52};
\markcell{ 55}{ 55};
\markcell{ 55}{ 56};
\markcell{ 55}{ 58};
\markcell{ 55}{ 64};
\markcell{ 55}{125};
\markcell{ 55}{126};
\markcell{ 55}{160};
\markcell{ 56}{ 47};
\markcell{ 56}{ 53};
\markcell{ 56}{ 55};
\markcell{ 56}{ 56};
\markcell{ 56}{ 57};
\markcell{ 56}{ 59};
\markcell{ 56}{ 65};
\markcell{ 56}{125};
\markcell{ 56}{126};
\markcell{ 56}{127};
\markcell{ 56}{161};
\markcell{ 57}{ 48};
\markcell{ 57}{ 54};
\markcell{ 57}{ 56};
\markcell{ 57}{ 57};
\markcell{ 57}{ 60};
\markcell{ 57}{ 66};
\markcell{ 57}{126};
\markcell{ 57}{127};
\markcell{ 57}{162};
\markcell{ 58}{ 49};
\markcell{ 58}{ 55};
\markcell{ 58}{ 58};
\markcell{ 58}{ 59};
\markcell{ 58}{ 67};
\markcell{ 58}{128};
\markcell{ 58}{129};
\markcell{ 58}{163};
\markcell{ 59}{ 50};
\markcell{ 59}{ 56};
\markcell{ 59}{ 58};
\markcell{ 59}{ 59};
\markcell{ 59}{ 60};
\markcell{ 59}{ 68};
\markcell{ 59}{128};
\markcell{ 59}{129};
\markcell{ 59}{130};
\markcell{ 59}{164};
\markcell{ 60}{ 51};
\markcell{ 60}{ 57};
\markcell{ 60}{ 59};
\markcell{ 60}{ 60};
\markcell{ 60}{ 69};
\markcell{ 60}{129};
\markcell{ 60}{130};
\markcell{ 60}{165};
\markcell{ 61}{ 41};
\markcell{ 61}{ 52};
\markcell{ 61}{ 61};
\markcell{ 61}{ 62};
\markcell{ 61}{ 64};
\markcell{ 61}{131};
\markcell{ 61}{132};
\markcell{ 61}{166};
\markcell{ 62}{ 41};
\markcell{ 62}{ 42};
\markcell{ 62}{ 53};
\markcell{ 62}{ 61};
\markcell{ 62}{ 62};
\markcell{ 62}{ 63};
\markcell{ 62}{ 65};
\markcell{ 62}{131};
\markcell{ 62}{132};
\markcell{ 62}{133};
\markcell{ 62}{167};
\markcell{ 63}{ 42};
\markcell{ 63}{ 54};
\markcell{ 63}{ 62};
\markcell{ 63}{ 63};
\markcell{ 63}{ 66};
\markcell{ 63}{132};
\markcell{ 63}{133};
\markcell{ 63}{168};
\markcell{ 64}{ 55};
\markcell{ 64}{ 61};
\markcell{ 64}{ 64};
\markcell{ 64}{ 65};
\markcell{ 64}{ 67};
\markcell{ 64}{134};
\markcell{ 64}{135};
\markcell{ 64}{169};
\markcell{ 65}{ 56};
\markcell{ 65}{ 62};
\markcell{ 65}{ 64};
\markcell{ 65}{ 65};
\markcell{ 65}{ 66};
\markcell{ 65}{ 68};
\markcell{ 65}{134};
\markcell{ 65}{136};
\markcell{ 65}{170};
\markcell{ 66}{ 57};
\markcell{ 66}{ 63};
\markcell{ 66}{ 65};
\markcell{ 66}{ 66};
\markcell{ 66}{ 69};
\markcell{ 66}{135};
\markcell{ 66}{136};
\markcell{ 66}{171};
\markcell{ 67}{ 58};
\markcell{ 67}{ 64};
\markcell{ 67}{ 67};
\markcell{ 67}{ 68};
\markcell{ 67}{137};
\markcell{ 67}{138};
\markcell{ 67}{172};
\markcell{ 68}{ 59};
\markcell{ 68}{ 65};
\markcell{ 68}{ 67};
\markcell{ 68}{ 68};
\markcell{ 68}{ 69};
\markcell{ 68}{137};
\markcell{ 68}{138};
\markcell{ 68}{139};
\markcell{ 68}{173};
\markcell{ 69}{ 60};
\markcell{ 69}{ 66};
\markcell{ 69}{ 68};
\markcell{ 69}{ 69};
\markcell{ 69}{138};
\markcell{ 69}{139};
\markcell{ 69}{174};
\markcell{ 70}{ 70};
\markcell{ 70}{ 71};
\markcell{ 70}{ 72};
\markcell{ 70}{ 74};
\markcell{ 70}{105};
\markcell{ 70}{109};
\markcell{ 70}{140};
\markcell{ 71}{ 70};
\markcell{ 71}{ 71};
\markcell{ 71}{ 73};
\markcell{ 71}{ 75};
\markcell{ 71}{106};
\markcell{ 71}{110};
\markcell{ 71}{141};
\markcell{ 72}{ 70};
\markcell{ 72}{ 72};
\markcell{ 72}{ 73};
\markcell{ 72}{ 76};
\markcell{ 72}{ 78};
\markcell{ 72}{ 79};
\markcell{ 72}{ 87};
\markcell{ 72}{ 88};
\markcell{ 72}{107};
\markcell{ 72}{111};
\markcell{ 72}{142};
\markcell{ 73}{ 71};
\markcell{ 73}{ 72};
\markcell{ 73}{ 73};
\markcell{ 73}{ 77};
\markcell{ 73}{ 79};
\markcell{ 73}{ 80};
\markcell{ 73}{ 88};
\markcell{ 73}{ 89};
\markcell{ 73}{108};
\markcell{ 73}{112};
\markcell{ 73}{143};
\markcell{ 74}{ 70};
\markcell{ 74}{ 74};
\markcell{ 74}{ 75};
\markcell{ 74}{ 76};
\markcell{ 74}{105};
\markcell{ 74}{109};
\markcell{ 74}{144};
\markcell{ 75}{ 71};
\markcell{ 75}{ 74};
\markcell{ 75}{ 75};
\markcell{ 75}{ 77};
\markcell{ 75}{106};
\markcell{ 75}{110};
\markcell{ 75}{145};
\markcell{ 76}{ 72};
\markcell{ 76}{ 74};
\markcell{ 76}{ 76};
\markcell{ 76}{ 77};
\markcell{ 76}{ 87};
\markcell{ 76}{ 88};
\markcell{ 76}{ 96};
\markcell{ 76}{ 97};
\markcell{ 76}{107};
\markcell{ 76}{111};
\markcell{ 76}{146};
\markcell{ 77}{ 73};
\markcell{ 77}{ 75};
\markcell{ 77}{ 76};
\markcell{ 77}{ 77};
\markcell{ 77}{ 88};
\markcell{ 77}{ 89};
\markcell{ 77}{ 97};
\markcell{ 77}{ 98};
\markcell{ 77}{108};
\markcell{ 77}{112};
\markcell{ 77}{147};
\markcell{ 78}{ 72};
\markcell{ 78}{ 78};
\markcell{ 78}{ 79};
\markcell{ 78}{ 81};
\markcell{ 78}{ 87};
\markcell{ 78}{113};
\markcell{ 78}{122};
\markcell{ 78}{148};
\markcell{ 79}{ 72};
\markcell{ 79}{ 73};
\markcell{ 79}{ 78};
\markcell{ 79}{ 79};
\markcell{ 79}{ 80};
\markcell{ 79}{ 82};
\markcell{ 79}{ 88};
\markcell{ 79}{114};
\markcell{ 79}{123};
\markcell{ 79}{149};
\markcell{ 80}{ 73};
\markcell{ 80}{ 79};
\markcell{ 80}{ 80};
\markcell{ 80}{ 83};
\markcell{ 80}{ 89};
\markcell{ 80}{115};
\markcell{ 80}{124};
\markcell{ 80}{150};
\markcell{ 81}{ 78};
\markcell{ 81}{ 81};
\markcell{ 81}{ 82};
\markcell{ 81}{ 84};
\markcell{ 81}{ 90};
\markcell{ 81}{116};
\markcell{ 81}{125};
\markcell{ 81}{151};
\markcell{ 82}{ 79};
\markcell{ 82}{ 81};
\markcell{ 82}{ 82};
\markcell{ 82}{ 83};
\markcell{ 82}{ 85};
\markcell{ 82}{ 91};
\markcell{ 82}{117};
\markcell{ 82}{126};
\markcell{ 82}{152};
\markcell{ 83}{ 80};
\markcell{ 83}{ 82};
\markcell{ 83}{ 83};
\markcell{ 83}{ 86};
\markcell{ 83}{ 92};
\markcell{ 83}{118};
\markcell{ 83}{127};
\markcell{ 83}{153};
\markcell{ 84}{ 81};
\markcell{ 84}{ 84};
\markcell{ 84}{ 85};
\markcell{ 84}{ 93};
\markcell{ 84}{119};
\markcell{ 84}{128};
\markcell{ 84}{154};
\markcell{ 85}{ 82};
\markcell{ 85}{ 84};
\markcell{ 85}{ 85};
\markcell{ 85}{ 86};
\markcell{ 85}{ 94};
\markcell{ 85}{120};
\markcell{ 85}{129};
\markcell{ 85}{155};
\markcell{ 86}{ 83};
\markcell{ 86}{ 85};
\markcell{ 86}{ 86};
\markcell{ 86}{ 95};
\markcell{ 86}{121};
\markcell{ 86}{130};
\markcell{ 86}{156};
\markcell{ 87}{ 72};
\markcell{ 87}{ 76};
\markcell{ 87}{ 78};
\markcell{ 87}{ 87};
\markcell{ 87}{ 88};
\markcell{ 87}{ 90};
\markcell{ 87}{ 96};
\markcell{ 87}{113};
\markcell{ 87}{122};
\markcell{ 87}{131};
\markcell{ 87}{157};
\markcell{ 88}{ 72};
\markcell{ 88}{ 73};
\markcell{ 88}{ 76};
\markcell{ 88}{ 77};
\markcell{ 88}{ 79};
\markcell{ 88}{ 87};
\markcell{ 88}{ 88};
\markcell{ 88}{ 89};
\markcell{ 88}{ 91};
\markcell{ 88}{ 97};
\markcell{ 88}{114};
\markcell{ 88}{123};
\markcell{ 88}{132};
\markcell{ 88}{158};
\markcell{ 89}{ 73};
\markcell{ 89}{ 77};
\markcell{ 89}{ 80};
\markcell{ 89}{ 88};
\markcell{ 89}{ 89};
\markcell{ 89}{ 92};
\markcell{ 89}{ 98};
\markcell{ 89}{115};
\markcell{ 89}{124};
\markcell{ 89}{133};
\markcell{ 89}{159};
\markcell{ 90}{ 81};
\markcell{ 90}{ 87};
\markcell{ 90}{ 90};
\markcell{ 90}{ 91};
\markcell{ 90}{ 93};
\markcell{ 90}{ 99};
\markcell{ 90}{116};
\markcell{ 90}{134};
\markcell{ 90}{160};
\markcell{ 91}{ 82};
\markcell{ 91}{ 88};
\markcell{ 91}{ 90};
\markcell{ 91}{ 91};
\markcell{ 91}{ 92};
\markcell{ 91}{ 94};
\markcell{ 91}{100};
\markcell{ 91}{117};
\markcell{ 91}{126};
\markcell{ 91}{135};
\markcell{ 91}{161};
\markcell{ 92}{ 83};
\markcell{ 92}{ 89};
\markcell{ 92}{ 91};
\markcell{ 92}{ 92};
\markcell{ 92}{ 95};
\markcell{ 92}{101};
\markcell{ 92}{118};
\markcell{ 92}{136};
\markcell{ 92}{162};
\markcell{ 93}{ 84};
\markcell{ 93}{ 90};
\markcell{ 93}{ 93};
\markcell{ 93}{ 94};
\markcell{ 93}{102};
\markcell{ 93}{119};
\markcell{ 93}{128};
\markcell{ 93}{137};
\markcell{ 93}{163};
\markcell{ 94}{ 85};
\markcell{ 94}{ 91};
\markcell{ 94}{ 93};
\markcell{ 94}{ 94};
\markcell{ 94}{ 95};
\markcell{ 94}{103};
\markcell{ 94}{120};
\markcell{ 94}{129};
\markcell{ 94}{138};
\markcell{ 94}{164};
\markcell{ 95}{ 86};
\markcell{ 95}{ 92};
\markcell{ 95}{ 94};
\markcell{ 95}{ 95};
\markcell{ 95}{104};
\markcell{ 95}{121};
\markcell{ 95}{130};
\markcell{ 95}{139};
\markcell{ 95}{165};
\markcell{ 96}{ 76};
\markcell{ 96}{ 87};
\markcell{ 96}{ 96};
\markcell{ 96}{ 97};
\markcell{ 96}{ 99};
\markcell{ 96}{122};
\markcell{ 96}{131};
\markcell{ 96}{166};
\markcell{ 97}{ 76};
\markcell{ 97}{ 77};
\markcell{ 97}{ 88};
\markcell{ 97}{ 96};
\markcell{ 97}{ 97};
\markcell{ 97}{ 98};
\markcell{ 97}{100};
\markcell{ 97}{123};
\markcell{ 97}{132};
\markcell{ 97}{167};
\markcell{ 98}{ 77};
\markcell{ 98}{ 89};
\markcell{ 98}{ 97};
\markcell{ 98}{ 98};
\markcell{ 98}{101};
\markcell{ 98}{124};
\markcell{ 98}{133};
\markcell{ 98}{168};
\markcell{ 99}{ 90};
\markcell{ 99}{ 96};
\markcell{ 99}{ 99};
\markcell{ 99}{100};
\markcell{ 99}{102};
\markcell{ 99}{125};
\markcell{ 99}{134};
\markcell{ 99}{169};
\markcell{100}{ 91};
\markcell{100}{ 97};
\markcell{100}{ 99};
\markcell{100}{100};
\markcell{100}{101};
\markcell{100}{103};
\markcell{100}{126};
\markcell{100}{135};
\markcell{100}{170};
\markcell{101}{ 92};
\markcell{101}{ 98};
\markcell{101}{100};
\markcell{101}{101};
\markcell{101}{104};
\markcell{101}{127};
\markcell{101}{136};
\markcell{101}{171};
\markcell{102}{ 93};
\markcell{102}{ 99};
\markcell{102}{102};
\markcell{102}{103};
\markcell{102}{128};
\markcell{102}{137};
\markcell{102}{172};
\markcell{103}{ 94};
\markcell{103}{100};
\markcell{103}{102};
\markcell{103}{103};
\markcell{103}{104};
\markcell{103}{129};
\markcell{103}{138};
\markcell{103}{173};
\markcell{104}{ 95};
\markcell{104}{101};
\markcell{104}{103};
\markcell{104}{104};
\markcell{104}{130};
\markcell{104}{139};
\markcell{104}{174};
\markcell{105}{  0};
\markcell{105}{  2};
\markcell{105}{ 35};
\markcell{105}{ 36};
\markcell{105}{ 70};
\markcell{105}{ 74};
\markcell{105}{105};
\markcell{105}{106};
\markcell{105}{107};
\markcell{105}{109};
\markcell{106}{  1};
\markcell{106}{  3};
\markcell{106}{ 35};
\markcell{106}{ 36};
\markcell{106}{ 71};
\markcell{106}{ 75};
\markcell{106}{105};
\markcell{106}{106};
\markcell{106}{108};
\markcell{106}{110};
\markcell{107}{  0};
\markcell{107}{  2};
\markcell{107}{  8};
\markcell{107}{  9};
\markcell{107}{ 17};
\markcell{107}{ 18};
\markcell{107}{ 37};
\markcell{107}{ 38};
\markcell{107}{ 72};
\markcell{107}{ 76};
\markcell{107}{105};
\markcell{107}{107};
\markcell{107}{108};
\markcell{107}{111};
\markcell{107}{113};
\markcell{107}{114};
\markcell{107}{122};
\markcell{107}{123};
\markcell{108}{  1};
\markcell{108}{  3};
\markcell{108}{  9};
\markcell{108}{ 10};
\markcell{108}{ 18};
\markcell{108}{ 19};
\markcell{108}{ 37};
\markcell{108}{ 38};
\markcell{108}{ 73};
\markcell{108}{ 77};
\markcell{108}{106};
\markcell{108}{107};
\markcell{108}{108};
\markcell{108}{112};
\markcell{108}{114};
\markcell{108}{115};
\markcell{108}{123};
\markcell{108}{124};
\markcell{109}{  4};
\markcell{109}{  6};
\markcell{109}{ 39};
\markcell{109}{ 40};
\markcell{109}{ 70};
\markcell{109}{ 74};
\markcell{109}{105};
\markcell{109}{109};
\markcell{109}{110};
\markcell{109}{111};
\markcell{110}{  5};
\markcell{110}{  7};
\markcell{110}{ 39};
\markcell{110}{ 40};
\markcell{110}{ 71};
\markcell{110}{ 75};
\markcell{110}{106};
\markcell{110}{109};
\markcell{110}{110};
\markcell{110}{112};
\markcell{111}{  4};
\markcell{111}{  6};
\markcell{111}{ 17};
\markcell{111}{ 18};
\markcell{111}{ 26};
\markcell{111}{ 27};
\markcell{111}{ 41};
\markcell{111}{ 42};
\markcell{111}{ 72};
\markcell{111}{ 76};
\markcell{111}{107};
\markcell{111}{109};
\markcell{111}{111};
\markcell{111}{112};
\markcell{111}{122};
\markcell{111}{123};
\markcell{111}{131};
\markcell{111}{132};
\markcell{112}{  5};
\markcell{112}{  7};
\markcell{112}{ 18};
\markcell{112}{ 19};
\markcell{112}{ 27};
\markcell{112}{ 28};
\markcell{112}{ 41};
\markcell{112}{ 42};
\markcell{112}{ 73};
\markcell{112}{ 77};
\markcell{112}{108};
\markcell{112}{110};
\markcell{112}{111};
\markcell{112}{112};
\markcell{112}{123};
\markcell{112}{124};
\markcell{112}{132};
\markcell{112}{133};
\markcell{113}{  2};
\markcell{113}{  8};
\markcell{113}{ 11};
\markcell{113}{ 43};
\markcell{113}{ 44};
\markcell{113}{ 78};
\markcell{113}{ 87};
\markcell{113}{107};
\markcell{113}{113};
\markcell{113}{114};
\markcell{113}{116};
\markcell{113}{122};
\markcell{114}{  2};
\markcell{114}{  3};
\markcell{114}{  9};
\markcell{114}{ 12};
\markcell{114}{ 43};
\markcell{114}{ 44};
\markcell{114}{ 45};
\markcell{114}{ 79};
\markcell{114}{ 88};
\markcell{114}{107};
\markcell{114}{108};
\markcell{114}{113};
\markcell{114}{114};
\markcell{114}{115};
\markcell{114}{117};
\markcell{114}{123};
\markcell{115}{  3};
\markcell{115}{ 10};
\markcell{115}{ 13};
\markcell{115}{ 44};
\markcell{115}{ 45};
\markcell{115}{ 80};
\markcell{115}{ 89};
\markcell{115}{108};
\markcell{115}{114};
\markcell{115}{115};
\markcell{115}{118};
\markcell{115}{124};
\markcell{116}{  8};
\markcell{116}{ 11};
\markcell{116}{ 14};
\markcell{116}{ 46};
\markcell{116}{ 47};
\markcell{116}{ 81};
\markcell{116}{ 90};
\markcell{116}{113};
\markcell{116}{116};
\markcell{116}{117};
\markcell{116}{119};
\markcell{116}{125};
\markcell{117}{  9};
\markcell{117}{ 12};
\markcell{117}{ 15};
\markcell{117}{ 46};
\markcell{117}{ 48};
\markcell{117}{ 82};
\markcell{117}{ 91};
\markcell{117}{114};
\markcell{117}{116};
\markcell{117}{117};
\markcell{117}{118};
\markcell{117}{120};
\markcell{117}{126};
\markcell{118}{ 10};
\markcell{118}{ 13};
\markcell{118}{ 16};
\markcell{118}{ 47};
\markcell{118}{ 48};
\markcell{118}{ 83};
\markcell{118}{ 92};
\markcell{118}{115};
\markcell{118}{117};
\markcell{118}{118};
\markcell{118}{121};
\markcell{118}{127};
\markcell{119}{ 11};
\markcell{119}{ 14};
\markcell{119}{ 49};
\markcell{119}{ 50};
\markcell{119}{ 84};
\markcell{119}{ 93};
\markcell{119}{116};
\markcell{119}{119};
\markcell{119}{120};
\markcell{119}{128};
\markcell{120}{ 12};
\markcell{120}{ 15};
\markcell{120}{ 49};
\markcell{120}{ 50};
\markcell{120}{ 51};
\markcell{120}{ 85};
\markcell{120}{ 94};
\markcell{120}{117};
\markcell{120}{119};
\markcell{120}{120};
\markcell{120}{121};
\markcell{120}{129};
\markcell{121}{ 13};
\markcell{121}{ 16};
\markcell{121}{ 50};
\markcell{121}{ 51};
\markcell{121}{ 86};
\markcell{121}{ 95};
\markcell{121}{118};
\markcell{121}{120};
\markcell{121}{121};
\markcell{121}{130};
\markcell{122}{  2};
\markcell{122}{  6};
\markcell{122}{ 17};
\markcell{122}{ 20};
\markcell{122}{ 52};
\markcell{122}{ 53};
\markcell{122}{ 78};
\markcell{122}{ 87};
\markcell{122}{ 96};
\markcell{122}{107};
\markcell{122}{111};
\markcell{122}{113};
\markcell{122}{122};
\markcell{122}{123};
\markcell{122}{125};
\markcell{122}{131};
\markcell{123}{  2};
\markcell{123}{  3};
\markcell{123}{  6};
\markcell{123}{  7};
\markcell{123}{ 18};
\markcell{123}{ 21};
\markcell{123}{ 52};
\markcell{123}{ 53};
\markcell{123}{ 54};
\markcell{123}{ 79};
\markcell{123}{ 88};
\markcell{123}{ 97};
\markcell{123}{107};
\markcell{123}{108};
\markcell{123}{111};
\markcell{123}{112};
\markcell{123}{114};
\markcell{123}{122};
\markcell{123}{123};
\markcell{123}{124};
\markcell{123}{126};
\markcell{123}{132};
\markcell{124}{  3};
\markcell{124}{  7};
\markcell{124}{ 19};
\markcell{124}{ 22};
\markcell{124}{ 53};
\markcell{124}{ 54};
\markcell{124}{ 80};
\markcell{124}{ 89};
\markcell{124}{ 98};
\markcell{124}{108};
\markcell{124}{112};
\markcell{124}{115};
\markcell{124}{123};
\markcell{124}{124};
\markcell{124}{127};
\markcell{124}{133};
\markcell{125}{ 17};
\markcell{125}{ 20};
\markcell{125}{ 23};
\markcell{125}{ 55};
\markcell{125}{ 56};
\markcell{125}{ 81};
\markcell{125}{ 99};
\markcell{125}{116};
\markcell{125}{122};
\markcell{125}{125};
\markcell{125}{126};
\markcell{125}{128};
\markcell{125}{134};
\markcell{126}{ 18};
\markcell{126}{ 24};
\markcell{126}{ 55};
\markcell{126}{ 56};
\markcell{126}{ 57};
\markcell{126}{ 82};
\markcell{126}{ 91};
\markcell{126}{100};
\markcell{126}{117};
\markcell{126}{123};
\markcell{126}{125};
\markcell{126}{126};
\markcell{126}{127};
\markcell{126}{129};
\markcell{126}{135};
\markcell{127}{ 19};
\markcell{127}{ 22};
\markcell{127}{ 25};
\markcell{127}{ 56};
\markcell{127}{ 57};
\markcell{127}{ 83};
\markcell{127}{101};
\markcell{127}{118};
\markcell{127}{124};
\markcell{127}{126};
\markcell{127}{127};
\markcell{127}{130};
\markcell{127}{136};
\markcell{128}{ 20};
\markcell{128}{ 23};
\markcell{128}{ 58};
\markcell{128}{ 59};
\markcell{128}{ 84};
\markcell{128}{ 93};
\markcell{128}{102};
\markcell{128}{119};
\markcell{128}{125};
\markcell{128}{128};
\markcell{128}{129};
\markcell{128}{137};
\markcell{129}{ 21};
\markcell{129}{ 24};
\markcell{129}{ 58};
\markcell{129}{ 59};
\markcell{129}{ 60};
\markcell{129}{ 85};
\markcell{129}{ 94};
\markcell{129}{103};
\markcell{129}{120};
\markcell{129}{126};
\markcell{129}{128};
\markcell{129}{129};
\markcell{129}{130};
\markcell{129}{138};
\markcell{130}{ 22};
\markcell{130}{ 25};
\markcell{130}{ 59};
\markcell{130}{ 60};
\markcell{130}{ 86};
\markcell{130}{ 95};
\markcell{130}{104};
\markcell{130}{121};
\markcell{130}{127};
\markcell{130}{129};
\markcell{130}{130};
\markcell{130}{139};
\markcell{131}{  6};
\markcell{131}{ 26};
\markcell{131}{ 29};
\markcell{131}{ 61};
\markcell{131}{ 62};
\markcell{131}{ 87};
\markcell{131}{ 96};
\markcell{131}{111};
\markcell{131}{122};
\markcell{131}{131};
\markcell{131}{132};
\markcell{131}{134};
\markcell{132}{  6};
\markcell{132}{  7};
\markcell{132}{ 27};
\markcell{132}{ 30};
\markcell{132}{ 61};
\markcell{132}{ 62};
\markcell{132}{ 63};
\markcell{132}{ 88};
\markcell{132}{ 97};
\markcell{132}{111};
\markcell{132}{112};
\markcell{132}{123};
\markcell{132}{131};
\markcell{132}{132};
\markcell{132}{133};
\markcell{132}{135};
\markcell{133}{  7};
\markcell{133}{ 28};
\markcell{133}{ 31};
\markcell{133}{ 62};
\markcell{133}{ 63};
\markcell{133}{ 89};
\markcell{133}{ 98};
\markcell{133}{112};
\markcell{133}{124};
\markcell{133}{132};
\markcell{133}{133};
\markcell{133}{136};
\markcell{134}{ 26};
\markcell{134}{ 29};
\markcell{134}{ 32};
\markcell{134}{ 64};
\markcell{134}{ 65};
\markcell{134}{ 90};
\markcell{134}{ 99};
\markcell{134}{125};
\markcell{134}{131};
\markcell{134}{134};
\markcell{134}{135};
\markcell{134}{137};
\markcell{135}{ 27};
\markcell{135}{ 30};
\markcell{135}{ 33};
\markcell{135}{ 64};
\markcell{135}{ 66};
\markcell{135}{ 91};
\markcell{135}{100};
\markcell{135}{126};
\markcell{135}{132};
\markcell{135}{134};
\markcell{135}{135};
\markcell{135}{136};
\markcell{135}{138};
\markcell{136}{ 28};
\markcell{136}{ 31};
\markcell{136}{ 34};
\markcell{136}{ 65};
\markcell{136}{ 66};
\markcell{136}{ 92};
\markcell{136}{101};
\markcell{136}{127};
\markcell{136}{133};
\markcell{136}{135};
\markcell{136}{136};
\markcell{136}{139};
\markcell{137}{ 29};
\markcell{137}{ 32};
\markcell{137}{ 67};
\markcell{137}{ 68};
\markcell{137}{ 93};
\markcell{137}{102};
\markcell{137}{128};
\markcell{137}{134};
\markcell{137}{137};
\markcell{137}{138};
\markcell{138}{ 30};
\markcell{138}{ 33};
\markcell{138}{ 67};
\markcell{138}{ 68};
\markcell{138}{ 69};
\markcell{138}{ 94};
\markcell{138}{103};
\markcell{138}{129};
\markcell{138}{135};
\markcell{138}{137};
\markcell{138}{138};
\markcell{138}{139};
\markcell{139}{ 31};
\markcell{139}{ 34};
\markcell{139}{ 68};
\markcell{139}{ 69};
\markcell{139}{ 95};
\markcell{139}{104};
\markcell{139}{130};
\markcell{139}{136};
\markcell{139}{138};
\markcell{139}{139};
\markcell{140}{  0};
\markcell{140}{  2};
\markcell{140}{ 35};
\markcell{140}{ 36};
\markcell{140}{ 70};
\markcell{140}{ 74};
\markcell{140}{105};
\markcell{140}{106};
\markcell{140}{107};
\markcell{140}{109};
\markcell{140}{140};
\markcell{140}{141};
\markcell{140}{142};
\markcell{140}{144};
\markcell{141}{  1};
\markcell{141}{  3};
\markcell{141}{ 35};
\markcell{141}{ 36};
\markcell{141}{ 71};
\markcell{141}{ 75};
\markcell{141}{105};
\markcell{141}{106};
\markcell{141}{108};
\markcell{141}{110};
\markcell{141}{140};
\markcell{141}{141};
\markcell{141}{143};
\markcell{141}{145};
\markcell{142}{  0};
\markcell{142}{  2};
\markcell{142}{  8};
\markcell{142}{  9};
\markcell{142}{ 17};
\markcell{142}{ 18};
\markcell{142}{ 37};
\markcell{142}{ 38};
\markcell{142}{ 72};
\markcell{142}{ 76};
\markcell{142}{105};
\markcell{142}{107};
\markcell{142}{108};
\markcell{142}{111};
\markcell{142}{113};
\markcell{142}{114};
\markcell{142}{122};
\markcell{142}{123};
\markcell{142}{140};
\markcell{142}{142};
\markcell{142}{143};
\markcell{142}{146};
\markcell{142}{148};
\markcell{142}{149};
\markcell{142}{157};
\markcell{142}{158};
\markcell{143}{  1};
\markcell{143}{  3};
\markcell{143}{  9};
\markcell{143}{ 10};
\markcell{143}{ 18};
\markcell{143}{ 19};
\markcell{143}{ 37};
\markcell{143}{ 38};
\markcell{143}{ 73};
\markcell{143}{ 77};
\markcell{143}{106};
\markcell{143}{107};
\markcell{143}{108};
\markcell{143}{112};
\markcell{143}{114};
\markcell{143}{115};
\markcell{143}{123};
\markcell{143}{124};
\markcell{143}{141};
\markcell{143}{142};
\markcell{143}{143};
\markcell{143}{147};
\markcell{143}{149};
\markcell{143}{150};
\markcell{143}{158};
\markcell{143}{159};
\markcell{144}{  4};
\markcell{144}{  6};
\markcell{144}{ 39};
\markcell{144}{ 40};
\markcell{144}{ 70};
\markcell{144}{ 74};
\markcell{144}{105};
\markcell{144}{109};
\markcell{144}{110};
\markcell{144}{111};
\markcell{144}{140};
\markcell{144}{144};
\markcell{144}{145};
\markcell{144}{146};
\markcell{145}{  5};
\markcell{145}{  7};
\markcell{145}{ 39};
\markcell{145}{ 40};
\markcell{145}{ 71};
\markcell{145}{ 75};
\markcell{145}{106};
\markcell{145}{109};
\markcell{145}{110};
\markcell{145}{112};
\markcell{145}{141};
\markcell{145}{144};
\markcell{145}{145};
\markcell{145}{147};
\markcell{146}{  4};
\markcell{146}{  6};
\markcell{146}{ 17};
\markcell{146}{ 18};
\markcell{146}{ 26};
\markcell{146}{ 27};
\markcell{146}{ 41};
\markcell{146}{ 42};
\markcell{146}{ 72};
\markcell{146}{ 76};
\markcell{146}{107};
\markcell{146}{109};
\markcell{146}{111};
\markcell{146}{112};
\markcell{146}{122};
\markcell{146}{123};
\markcell{146}{131};
\markcell{146}{132};
\markcell{146}{142};
\markcell{146}{144};
\markcell{146}{146};
\markcell{146}{147};
\markcell{146}{157};
\markcell{146}{158};
\markcell{146}{166};
\markcell{146}{167};
\markcell{147}{  5};
\markcell{147}{  7};
\markcell{147}{ 18};
\markcell{147}{ 19};
\markcell{147}{ 27};
\markcell{147}{ 28};
\markcell{147}{ 41};
\markcell{147}{ 42};
\markcell{147}{ 73};
\markcell{147}{ 77};
\markcell{147}{108};
\markcell{147}{110};
\markcell{147}{111};
\markcell{147}{112};
\markcell{147}{123};
\markcell{147}{124};
\markcell{147}{132};
\markcell{147}{133};
\markcell{147}{143};
\markcell{147}{145};
\markcell{147}{146};
\markcell{147}{147};
\markcell{147}{158};
\markcell{147}{159};
\markcell{147}{167};
\markcell{147}{168};
\markcell{148}{  2};
\markcell{148}{  8};
\markcell{148}{ 11};
\markcell{148}{ 43};
\markcell{148}{ 44};
\markcell{148}{ 78};
\markcell{148}{ 87};
\markcell{148}{107};
\markcell{148}{113};
\markcell{148}{114};
\markcell{148}{116};
\markcell{148}{122};
\markcell{148}{142};
\markcell{148}{148};
\markcell{148}{149};
\markcell{148}{151};
\markcell{148}{157};
\markcell{149}{  2};
\markcell{149}{  3};
\markcell{149}{  9};
\markcell{149}{ 12};
\markcell{149}{ 43};
\markcell{149}{ 44};
\markcell{149}{ 45};
\markcell{149}{ 79};
\markcell{149}{ 88};
\markcell{149}{107};
\markcell{149}{108};
\markcell{149}{113};
\markcell{149}{114};
\markcell{149}{115};
\markcell{149}{117};
\markcell{149}{123};
\markcell{149}{142};
\markcell{149}{143};
\markcell{149}{148};
\markcell{149}{149};
\markcell{149}{150};
\markcell{149}{152};
\markcell{149}{158};
\markcell{150}{  3};
\markcell{150}{ 10};
\markcell{150}{ 13};
\markcell{150}{ 44};
\markcell{150}{ 45};
\markcell{150}{ 80};
\markcell{150}{ 89};
\markcell{150}{108};
\markcell{150}{114};
\markcell{150}{115};
\markcell{150}{118};
\markcell{150}{124};
\markcell{150}{143};
\markcell{150}{149};
\markcell{150}{150};
\markcell{150}{153};
\markcell{150}{159};
\markcell{151}{  8};
\markcell{151}{ 11};
\markcell{151}{ 14};
\markcell{151}{ 46};
\markcell{151}{ 47};
\markcell{151}{ 81};
\markcell{151}{ 90};
\markcell{151}{113};
\markcell{151}{116};
\markcell{151}{117};
\markcell{151}{119};
\markcell{151}{125};
\markcell{151}{148};
\markcell{151}{151};
\markcell{151}{152};
\markcell{151}{154};
\markcell{151}{160};
\markcell{152}{  9};
\markcell{152}{ 12};
\markcell{152}{ 15};
\markcell{152}{ 46};
\markcell{152}{ 47};
\markcell{152}{ 48};
\markcell{152}{ 82};
\markcell{152}{ 91};
\markcell{152}{114};
\markcell{152}{116};
\markcell{152}{117};
\markcell{152}{118};
\markcell{152}{120};
\markcell{152}{126};
\markcell{152}{149};
\markcell{152}{151};
\markcell{152}{152};
\markcell{152}{153};
\markcell{152}{155};
\markcell{152}{161};
\markcell{153}{ 10};
\markcell{153}{ 13};
\markcell{153}{ 16};
\markcell{153}{ 47};
\markcell{153}{ 48};
\markcell{153}{ 83};
\markcell{153}{ 92};
\markcell{153}{115};
\markcell{153}{117};
\markcell{153}{118};
\markcell{153}{121};
\markcell{153}{127};
\markcell{153}{150};
\markcell{153}{152};
\markcell{153}{153};
\markcell{153}{156};
\markcell{153}{162};
\markcell{154}{ 11};
\markcell{154}{ 14};
\markcell{154}{ 49};
\markcell{154}{ 50};
\markcell{154}{ 84};
\markcell{154}{ 93};
\markcell{154}{116};
\markcell{154}{119};
\markcell{154}{120};
\markcell{154}{128};
\markcell{154}{151};
\markcell{154}{154};
\markcell{154}{155};
\markcell{154}{163};
\markcell{155}{ 12};
\markcell{155}{ 15};
\markcell{155}{ 49};
\markcell{155}{ 50};
\markcell{155}{ 51};
\markcell{155}{ 85};
\markcell{155}{ 94};
\markcell{155}{117};
\markcell{155}{119};
\markcell{155}{120};
\markcell{155}{121};
\markcell{155}{129};
\markcell{155}{152};
\markcell{155}{154};
\markcell{155}{155};
\markcell{155}{156};
\markcell{155}{164};
\markcell{156}{ 13};
\markcell{156}{ 16};
\markcell{156}{ 50};
\markcell{156}{ 51};
\markcell{156}{ 86};
\markcell{156}{ 95};
\markcell{156}{118};
\markcell{156}{120};
\markcell{156}{121};
\markcell{156}{130};
\markcell{156}{153};
\markcell{156}{155};
\markcell{156}{156};
\markcell{156}{165};
\markcell{157}{  2};
\markcell{157}{  6};
\markcell{157}{ 17};
\markcell{157}{ 20};
\markcell{157}{ 52};
\markcell{157}{ 53};
\markcell{157}{ 78};
\markcell{157}{ 87};
\markcell{157}{ 96};
\markcell{157}{107};
\markcell{157}{111};
\markcell{157}{113};
\markcell{157}{122};
\markcell{157}{123};
\markcell{157}{125};
\markcell{157}{131};
\markcell{157}{142};
\markcell{157}{146};
\markcell{157}{148};
\markcell{157}{157};
\markcell{157}{158};
\markcell{157}{160};
\markcell{157}{166};
\markcell{158}{  2};
\markcell{158}{  3};
\markcell{158}{  6};
\markcell{158}{  7};
\markcell{158}{ 18};
\markcell{158}{ 21};
\markcell{158}{ 52};
\markcell{158}{ 53};
\markcell{158}{ 54};
\markcell{158}{ 79};
\markcell{158}{ 88};
\markcell{158}{ 97};
\markcell{158}{107};
\markcell{158}{108};
\markcell{158}{111};
\markcell{158}{112};
\markcell{158}{114};
\markcell{158}{122};
\markcell{158}{123};
\markcell{158}{124};
\markcell{158}{126};
\markcell{158}{132};
\markcell{158}{142};
\markcell{158}{143};
\markcell{158}{146};
\markcell{158}{147};
\markcell{158}{149};
\markcell{158}{157};
\markcell{158}{158};
\markcell{158}{159};
\markcell{158}{161};
\markcell{158}{167};
\markcell{159}{  3};
\markcell{159}{  7};
\markcell{159}{ 19};
\markcell{159}{ 22};
\markcell{159}{ 53};
\markcell{159}{ 54};
\markcell{159}{ 80};
\markcell{159}{ 89};
\markcell{159}{ 98};
\markcell{159}{108};
\markcell{159}{112};
\markcell{159}{115};
\markcell{159}{123};
\markcell{159}{124};
\markcell{159}{127};
\markcell{159}{133};
\markcell{159}{143};
\markcell{159}{147};
\markcell{159}{150};
\markcell{159}{158};
\markcell{159}{159};
\markcell{159}{162};
\markcell{159}{168};
\markcell{160}{ 17};
\markcell{160}{ 20};
\markcell{160}{ 23};
\markcell{160}{ 55};
\markcell{160}{ 56};
\markcell{160}{ 81};
\markcell{160}{ 90};
\markcell{160}{ 99};
\markcell{160}{116};
\markcell{160}{122};
\markcell{160}{125};
\markcell{160}{126};
\markcell{160}{128};
\markcell{160}{134};
\markcell{160}{151};
\markcell{160}{157};
\markcell{160}{160};
\markcell{160}{161};
\markcell{160}{163};
\markcell{160}{169};
\markcell{161}{ 18};
\markcell{161}{ 21};
\markcell{161}{ 24};
\markcell{161}{ 55};
\markcell{161}{ 56};
\markcell{161}{ 57};
\markcell{161}{ 82};
\markcell{161}{ 91};
\markcell{161}{100};
\markcell{161}{117};
\markcell{161}{123};
\markcell{161}{125};
\markcell{161}{126};
\markcell{161}{127};
\markcell{161}{129};
\markcell{161}{135};
\markcell{161}{152};
\markcell{161}{158};
\markcell{161}{160};
\markcell{161}{161};
\markcell{161}{162};
\markcell{161}{164};
\markcell{161}{170};
\markcell{162}{ 19};
\markcell{162}{ 22};
\markcell{162}{ 25};
\markcell{162}{ 56};
\markcell{162}{ 57};
\markcell{162}{ 83};
\markcell{162}{ 92};
\markcell{162}{101};
\markcell{162}{118};
\markcell{162}{124};
\markcell{162}{126};
\markcell{162}{127};
\markcell{162}{130};
\markcell{162}{136};
\markcell{162}{153};
\markcell{162}{159};
\markcell{162}{161};
\markcell{162}{162};
\markcell{162}{165};
\markcell{162}{171};
\markcell{163}{ 20};
\markcell{163}{ 23};
\markcell{163}{ 58};
\markcell{163}{ 59};
\markcell{163}{ 84};
\markcell{163}{ 93};
\markcell{163}{102};
\markcell{163}{119};
\markcell{163}{125};
\markcell{163}{128};
\markcell{163}{129};
\markcell{163}{137};
\markcell{163}{154};
\markcell{163}{160};
\markcell{163}{163};
\markcell{163}{164};
\markcell{163}{172};
\markcell{164}{ 21};
\markcell{164}{ 24};
\markcell{164}{ 58};
\markcell{164}{ 59};
\markcell{164}{ 60};
\markcell{164}{ 85};
\markcell{164}{ 94};
\markcell{164}{103};
\markcell{164}{120};
\markcell{164}{126};
\markcell{164}{128};
\markcell{164}{129};
\markcell{164}{130};
\markcell{164}{138};
\markcell{164}{155};
\markcell{164}{161};
\markcell{164}{163};
\markcell{164}{164};
\markcell{164}{165};
\markcell{164}{173};
\markcell{165}{ 22};
\markcell{165}{ 25};
\markcell{165}{ 59};
\markcell{165}{ 60};
\markcell{165}{ 86};
\markcell{165}{ 95};
\markcell{165}{104};
\markcell{165}{121};
\markcell{165}{127};
\markcell{165}{129};
\markcell{165}{130};
\markcell{165}{139};
\markcell{165}{156};
\markcell{165}{162};
\markcell{165}{164};
\markcell{165}{165};
\markcell{165}{174};
\markcell{166}{  6};
\markcell{166}{ 26};
\markcell{166}{ 29};
\markcell{166}{ 61};
\markcell{166}{ 62};
\markcell{166}{ 87};
\markcell{166}{ 96};
\markcell{166}{111};
\markcell{166}{122};
\markcell{166}{131};
\markcell{166}{132};
\markcell{166}{134};
\markcell{166}{146};
\markcell{166}{157};
\markcell{166}{166};
\markcell{166}{167};
\markcell{166}{169};
\markcell{167}{  6};
\markcell{167}{  7};
\markcell{167}{ 27};
\markcell{167}{ 30};
\markcell{167}{ 61};
\markcell{167}{ 62};
\markcell{167}{ 63};
\markcell{167}{ 88};
\markcell{167}{ 97};
\markcell{167}{111};
\markcell{167}{112};
\markcell{167}{123};
\markcell{167}{131};
\markcell{167}{132};
\markcell{167}{133};
\markcell{167}{135};
\markcell{167}{146};
\markcell{167}{147};
\markcell{167}{158};
\markcell{167}{166};
\markcell{167}{167};
\markcell{167}{168};
\markcell{167}{170};
\markcell{168}{  7};
\markcell{168}{ 28};
\markcell{168}{ 31};
\markcell{168}{ 62};
\markcell{168}{ 63};
\markcell{168}{ 89};
\markcell{168}{ 98};
\markcell{168}{112};
\markcell{168}{124};
\markcell{168}{132};
\markcell{168}{133};
\markcell{168}{136};
\markcell{168}{147};
\markcell{168}{159};
\markcell{168}{167};
\markcell{168}{168};
\markcell{168}{171};
\markcell{169}{ 26};
\markcell{169}{ 29};
\markcell{169}{ 32};
\markcell{169}{ 64};
\markcell{169}{ 65};
\markcell{169}{ 90};
\markcell{169}{ 99};
\markcell{169}{125};
\markcell{169}{131};
\markcell{169}{134};
\markcell{169}{135};
\markcell{169}{137};
\markcell{169}{160};
\markcell{169}{166};
\markcell{169}{169};
\markcell{169}{170};
\markcell{169}{172};
\markcell{170}{ 27};
\markcell{170}{ 30};
\markcell{170}{ 33};
\markcell{170}{ 64};
\markcell{170}{ 65};
\markcell{170}{ 66};
\markcell{170}{ 91};
\markcell{170}{100};
\markcell{170}{126};
\markcell{170}{132};
\markcell{170}{134};
\markcell{170}{135};
\markcell{170}{136};
\markcell{170}{138};
\markcell{170}{161};
\markcell{170}{167};
\markcell{170}{169};
\markcell{170}{170};
\markcell{170}{171};
\markcell{170}{173};
\markcell{171}{ 28};
\markcell{171}{ 31};
\markcell{171}{ 34};
\markcell{171}{ 65};
\markcell{171}{ 66};
\markcell{171}{ 92};
\markcell{171}{101};
\markcell{171}{127};
\markcell{171}{133};
\markcell{171}{135};
\markcell{171}{136};
\markcell{171}{139};
\markcell{171}{162};
\markcell{171}{168};
\markcell{171}{170};
\markcell{171}{171};
\markcell{171}{174};
\markcell{172}{ 29};
\markcell{172}{ 32};
\markcell{172}{ 67};
\markcell{172}{ 68};
\markcell{172}{ 93};
\markcell{172}{102};
\markcell{172}{128};
\markcell{172}{134};
\markcell{172}{137};
\markcell{172}{138};
\markcell{172}{163};
\markcell{172}{169};
\markcell{172}{172};
\markcell{172}{173};
\markcell{173}{ 30};
\markcell{173}{ 33};
\markcell{173}{ 67};
\markcell{173}{ 68};
\markcell{173}{ 69};
\markcell{173}{ 94};
\markcell{173}{103};
\markcell{173}{129};
\markcell{173}{135};
\markcell{173}{137};
\markcell{173}{138};
\markcell{173}{139};
\markcell{173}{164};
\markcell{173}{170};
\markcell{173}{172};
\markcell{173}{173};
\markcell{173}{174};
\markcell{174}{ 31};
\markcell{174}{ 34};
\markcell{174}{ 68};
\markcell{174}{ 69};
\markcell{174}{ 95};
\markcell{174}{104};
\markcell{174}{130};
\markcell{174}{136};
\markcell{174}{138};
\markcell{174}{139};
\markcell{174}{165};
\markcell{174}{171};
\markcell{174}{173};
\markcell{174}{174};

    %BLOCK 1

%   \draw[ultra thick] (0,\xStep*\nStep) -- (8*\xStep,\xStep*\nStep) -- (8*\xStep,\xStep*\nStep-8*\xStep) -- (0,\xStep*\nStep-8*\xStep) -- cycle;
%   \draw[ultra thick] (8*\xStep,\xStep*\nStep) -- (35*\xStep,\xStep*\nStep) -- (35*\xStep,\xStep*\nStep-8*\xStep) -- (8*\xStep,\xStep*\nStep-8*\xStep) -- cycle;
%   \draw[ultra thick] (8*\xStep,\xStep*\nStep-8*\xStep) -- (35*\xStep,\xStep*\nStep-8*\xStep) -- (35*\xStep,0) -- (8*\xStep,0) -- cycle;
%   \draw[ultra thick] (0,\xStep*\nStep-8*\xStep) -- (8*\xStep,\xStep*\nStep-8*\xStep) -- (8*\xStep,0) -- (0*\xStep,0) -- cycle;

\end{tikzpicture}

  \caption{Non-zero structure of the linear system used in the coupled solution algorithm for a block-structured grid consisting of one $2\times2\times2$ cell block and one $3\times3\times3$ cell block. The variables have been ordered in a way that each major matrix block refers to only one variable or the coupling between exactly two variables as it is denoted by the corresponding subindices.}
  \label{fig:nointerlacemat}
\end{figure}
