  \section{Finite Volume Method for Incompressible Flows -- Segregated Approach}

  The purpose of this section is to present the discretization applied to the set of equations (\ref{eq:completeset}). The applied discretization techniques depend on the different terms of each equation, thus at first every equation will be discretized individually. The finite volume method relies on the discretization of integral equations, which will be derived at the beginning of each subsection that relies on them. Since the system of partial differential equations to be solved always exhibits coupling at least between the dependent variables pressure and velocity a first solution algorithm, namely the \textit{SIMPLE} algorithm addressed to resolve the pressure velocity coupling is introduced. The efficient coupling of the Navier-Stokes equations to the temperature equation is one part of the present thesis and will be addressed in a separate subsection. Furthermore every problem modelled by partial differential equations needs to provide valid boundary conditions. The discretization of those boundary conditions, that are relevant for the present thesis will be presented in their own subsection.

    \subsection{Discretization of the Mass Balance}

    Integration of equation (\ref{eq:contidiff}) over the integration domain of a single control volume \(P\) yields after the application of Gauss' integration theorem and the additivity of the Riemann integral
    \begin{displaymath}
      \iint\limits_S  u_i n_i \mathrm{d}S = \sum_{f \in \{w,s,b,t,n,e\}} \iint\limits_{S_f}  u_i n_{i} \mathrm{d}S = 0.
    \end{displaymath}
    In the present work the mass balance is discretized using the midpoint rule for the surface integrals and linear interpolation of the velocity to to center of mass of the surface. This leads to the following form of the mass balance: 
    \begin{align*}
      \sum_{f \in \{w,s,b,t,n,e\}} u_{i_f} n_{f_i} S_f 
      &= u_{i_w} n_{w_i} S_w + u_{i_e} n_{e_i} S_e 
       + u_{i_s} n_{s_i} S_s + u_{i_n} n_{n_i} S_n 
       + u_{i_b} n_{b_i} S_b + u_{i_t} n_{t_i} S_t  \\
      &= ( \gamma_w u_{i_W} + (1 - \gamma_w ) u_{i_P} ) n_{w_i} S_w + ( \gamma_s u_{i_S} + (1 - \gamma_s ) u_{i_P} ) n_{s_i} S_s \\
      &\quad + ( \gamma_b u_{i_B} + (1 - \gamma_b ) u_{i_P} ) n_{b_i} S_b + ( \gamma_t u_{i_T} + (1 - \gamma_t ) u_{i_P} ) n_{t_i} S_t \\
      &\quad + ( \gamma_n u_{i_N} + (1 - \gamma_n ) u_{i_P} ) n_{n_i} S_n + ( \gamma_e u_{i_E} + (1 - \gamma_e ) u_{i_P} ) n_{e_i} S_e \\
      & =  0,
    \end{align*}
    where \( \gamma_f \) for \( f \in \{w,e,s,n,b,t\} \) is the geometrical interpolation factor.

    \subsection{Discretization of the Momentum Balance}

      The stationary momentum balance integrated over a single control volume \(P\) reads as
      \begin{equation}
        \label{eq:semidiscrete}
        \underbrace{\iint\limits_S (\rho u_i u_j)n_j \mathrm{d}S}_{\text{convective term}}
        - \underbrace{\iint\limits_S \left(\mu \left( \frac{\partial u_i}{\partial x_j} + \frac{\partial u_j}{\partial x_i}\right)\right)n_j \mathrm{d}S}_{\text{diffusive term}}
        = - \underbrace{\iiint\limits_V \frac{\partial p}{\partial x_i} \mathrm{d}V}_{\text{sourceterm pressure}}
        - \underbrace{\iiint\limits_V \rho \beta \left(T - T_0\right) \mathrm{d}V}_{\text{sourceterm temperature}}
      \end{equation}
      where the different terms to be addressed individually in the following sections are indicated. Note that the form of this equation has been modified by using Gauss' integration theorem. The terms residing on the left will be treated in an implicit and due to deferred corrections in an explicit manner whereas the terms on the right will be treated exclusively in an  explicit manner.

      \subsubsection{Calculation of Mass Flux -- Rhie-Chow Interpolation}

      \subsubsection{Linearization and Discretization of the Convective Term}

      The convective term \(\rho u_i u_j\) of the Navier-Stokes equations is the reason for the non-linearity of the equations. In order to deduce a set of linear algebraic equations from the Navier-Stokes equations this term has to be linearized. As introduced in section (\ref{sec:nonlinear}), the non linearity will be dealt with by means of an iterative process, the Picard iteration. The part dependent on the non dominant dependent variable therefore will be approximated by its value from the previous iteration as \( \rho u_i^{(n)} u_j^{(n)} \approx \rho u_i^{(n)} u_j^{(n-1)} \). However this linearization will not be directly visible because it will be covered by the mass flux \(\dot{m}_f = \iint\limits_{S_f} \rho u_j^{(n-1)} n_j \mathrm{d}S \). Using the additivity of the Riemann integral the first step is to decompose the surface integral into individual contributions from each boundary face of the control volume \(P\)
      \begin{displaymath}
      \iint\limits_S \rho u_i u_jn_j \mathrm{d}S
      = \sum_{f \in \{w,s,b,t,n,e\}} \iint\limits_{S_f}\rho u_{i} u_{j} n_{j} \mathrm{d}S
      = \sum_{f \in \{w,s,b,t,n,e\}} F_{i,f}^{c}
      \end{displaymath}
      where \(F_{i,f}^c := \iint\limits_{S_f} \rho u_{i}^{(n)} u_{j}^{(n-1)} n_{j} \mathrm{d}S \) is the convective flux of the velocity \(u_i\) through the face \(S_f\). 
      
      To improve diagonal dominance of the resulting linear system while maintaining the smaller discretization error of a higher order discretization, a blended discretization scheme is applied using a deferred correction. Since due to the non-linearity of the equations to be solved an iterative solution process is needed by all means, the overall convergence doesn't degrade noticeably when using a deferred correction. Blending and deferred correction result in a decomposition of the convective flux into a lower order approximation that is treated implicitly and the explicit difference between the higher and lower order approximation for the same convective flux. Since for coarse grid resolutions the use of higher order approximations may lead to oscillations of the solution which may degrade or even impede convergence, the schemes can be blended by a control factor \( \eta \in [0,1]\). To show the generality of this approach all further derivations are presented for the generic boundary face \(S_f\) that separates control volume \(P\) from its neighbour \(F \in NB(P)\). This decomposition then leads to
      \begin{displaymath}
        F_{i,f}^c \approx  \underbrace{F_{i,f}^{c,l}}_{\text{implicit}} + \eta \, \bigl[\underbrace{ F_{i,f}^{c,h} - F_{i,f}^{c,l} }_{\text{explicit}}\bigr]^{(n-1)}.
      \end{displaymath}
      Note that the convective fluxes carrying a \(l\) or \(h\) as exponent, already have been linearized and discretized. The discretization applied to the convective flux in the present work is using the midpoint integration rule and blends the upwind interpolation scheme with the linear interpolation scheme. Applied to above decomposition one can derive the following approximations
      \begin{align*}
        F_{i,f}^{c,l} &= u_{i,F} \min(\dot{m}_f ,0) + u_{i,P} \max(0,\dot{m}_f) \\
        F_{i,f}^{c,h} &= u_{i,F} \, \gamma_f + u_{i,P} \, (1 - \gamma_f),
      \end{align*}
      where the variable values have to be taken from the previous iteration step \((n-1)\) as necessary and the mass flux \(\dot{m}_f\) has been used as result of the linearization process. The results can now be summarized by presenting the convective contribution to the matrix coefficients \(a_{F,u_i}\) and \(a_{P,u_i}\) and the right hand side \(b_{P,u_i}\) which are calculated as
      \begin{subequations}
      \begin{align}
        a_{F,u_i}^c &= \min(\dot{m}_f ,0), \quad \quad a_{P,u_i}^c = \sum_{F \in NB(P)} \max(0,\dot{m}_f) \\[1em]
        b_{P,u_i}^c &= \sum_{F \in NB(P)} \eta  \left(u_{i,F}^{(n-1)} \left( \min(\dot{m}_f,0) - \gamma_f \right)\right) \nonumber \\
                    &\quad \quad \quad  \quad+ \eta \left( u_{i,P}^{(n-1)} \left( \max(0,\dot{m}_f) - \left(1 - \gamma_f\right) \right)\right)
      \end{align}
    \end{subequations}

      \subsubsection{Discretization of the Diffusive Term}

      The diffusive term contains the first partial derivatives of the velocity as result of the material constitutive equation that characterizes the behaviour of Newtonian fluids. As pointed out in section \ref{sec:nonorth} directional derivatives can be discretized using central differences on orthogonal grids or in the more general case of non-orthogonal grids using central differences implicitly and a explicit deferred correction comprising the non-orthogonality of the grid. As seen in equation (\ref{eq:navierstokes}) the diffusive term of the Navier-Stokes equations can be simplified using the mass balance in the case of an incompressible flow with constant viscosity \(\mu\). To sustain the generality of the presented approach this simplification will be omitted.

      As before, by using the additivity and furthermore linearity of the Riemann integral, the integration of the diffusive term will be divided into integration over individual boundary faces \(S_f\) 
      \begin{displaymath}
      \iint\limits_S \left(\mu \left( \frac{\partial u_i}{\partial x_j} + \frac{\partial u_j}{\partial x_i}\right)\right)n_j \mathrm{d}S \mathrm{d}S
      = \sum_{f \in \{w,s,b,t,n,e\}} \left[
        \iint\limits_{S_f} \mu \underline{\frac{\partial u_i}{\partial x_j}n_j \mathrm{d}S}
    + \iint\limits_{S_f} \mu \frac{\partial u_j}{\partial x_i}n_j \mathrm{d}S \right]
       = \sum_{f \in \{w,s,b,t,n,e\}} F_{i,f}^{d},
      \end{displaymath}
      where \(F_{i,f}^{d}\) denotes the diffusive flux through an individual boundary face. Section \ref{sec:nonorth} only covered the non-orthogonal corrector for directional derivatives. Since the velocity is a vector field and not a scalar field, the results of section \ref{sec:nonorth} may only be applied to the underlined term. The other term will be treated explicitly since it is considerably smaller than the underlined term and does not cause oscillations and thus will not derogate convergence. To begin with all present integrals will be approximated using the midpoint rule of integration. The diffusive flux \(F_{i,f}^d\) for a generic face \(S_f\) then reads 
      \begin{displaymath}
        F_{i,f}^d \approx \mu \underline{\left(\frac{\partial u_i}{\partial x_j}\right)_f n_j S_f} + \mu \left(\frac{\partial u_j}{\partial x_i}\right)_f n_j S_f.
      \end{displaymath}

      Using central differences for the implicit discretization of the directional derivative and furthermore using the \textit{orthogonal correction} approach from \ref{seq:orthcorrapproach} the approximation can be derived as
      \begin{align*}
        F_{i,f}^d 
        &\approx 
        \mu \left( \underline{||\vecg{\vecg{\Delta}_f}||_2 \frac{u_{P_i} - u_{F_i}}{ || \vec{x}_P - \vec{x}_F ||_2 }  
        -  \left(\nabla u_i \right)_f^{(n-1)} \cdot \left(\vecg{\Delta}_f - \vec{S}_f\right)  }  \right)
        + \mu \left( \frac{\partial u_j}{\partial x_i} \right)_f^{(n-1)} n_{f_i} \\[1em]
        &= \mu \left(\underline{  S_f \frac{u_{P_i} - u_{F_i}}{ || \vec{x}_P - \vec{x}_F ||_2 }  
    - \left( \frac{\partial u_i}{\partial x_j}\right)_f^{(n-1)} \left(\xi_{f_i} - n_{f_i}\right)S_f  } \right)
      + \mu \left( \frac{\partial u_j}{\partial x_i} \right)_f^{(n-1)} n_{f_i},
    \end{align*}
      where the unit vector pointing in direction of the straight line connecting control volume \(P\) and control volume \(F\) is denoted as
      \begin{displaymath}
        \vecg{\xi}_f = \frac{\vec{x}_P - \vec{x}_F}{|| \vec{x}_p - \vec{x}_F ||_2}.
      \end{displaymath}
      The interpolation of the cell center gradients to the boundary faces is performed as in (\ref{eq:interpolgrad}). Now the contribution of the diffusive part to the matrix coefficients and the right hand side can be calculated as
      \begin{subequations}
        \begin{align}
          a_{F,u_i}^d &= - \frac{\mu S_f}{||\vec{x}_P - \vec{x}_F||_2}, 
          \quad \quad a_{P,u_i}^d = \sum_{F \in NB(P)} \frac{\mu S_f}{|| \vec{x}_P - \vec{x}_F ||} \\[1em]
          b_{F,u_i}^d &=  \sum_{F \in NB(P)} \left( \frac{\partial u_i}{\partial x_j}\right)_f^{(n-1)} \left(\xi_{f_i} - n_{f_i}\right)S_f  
          - \mu \left( \frac{\partial u_j}{\partial x_i} \right)_f^{(n-1)} n_{f_i} S_f   \nonumber \\[0.5em]
          &=   \left( \frac{\partial u_i}{\partial x_j}\right)_f^{(n-1)} \xi_{f_i} S_f
          - \mu \left( \left( \frac{\partial u_i}{\partial x_j} \right)_f^{(n-1)}
          - \left( \frac{\partial u_j}{\partial x_i} \right)_f^{(n-1)} \right) n_{f_i} S_f.
        \end{align}
      \end{subequations}

      \subsubsection{Discretization of the Source Terms}

      Since in the segregated solution approach in every equation all other variables but the dominant one are treated as constants and furthermore the source terms in equation (\ref{eq:semidiscrete}) do not depend on the dominant variable the discretization is straightforward. The source terms of the momentum balance are discretized using the midpoint rule of integration, which leads to the source term
      \begin{equation}
        - \iiint\limits_V \frac{\partial p}{\partial x_i} \mathrm{d}V
        - \iiint\limits_V \rho \beta \left(T - T_0\right) \mathrm{d}V
        \approx
        - \left(\frac{\partial p}{\partial x_i}\right)_P^{(n-1)} V_P
        - \rho \beta \left(T_P^{(n-1)} - T_0\right) V_P
        = b_{P,u_i}^{sc}
      \end{equation}

    \subsection{Assembly of Linear Systems -- Final Form of Equations with under-relaxation}

    The objective of a finite volume method is to create a set of linear algebraic equations by discretizing partial differential equations. For the momentum balance all necessary components have been calculated. Taking all contributions together leads to the following linear algebraic equation for each control volume \(P\)
    \begin{displaymath}
      a_{P,u_i} u_{P_i} + \sum_{F \in NB(P)} a_F u_{F_i} = b_{P,u_i},
    \end{displaymath}
    where the coefficients are composed as
    \begin{align}
      a_{P,u_i} &= a_{P,u_i}^c - a_{P,u_i}^d \\
      a_{F,u_i} &= a_{F,u_i}^c - a_{F,u_i}^d \\
      b_{P,u_i} &= b_{P,u_i}^c - b_{P,u_i}^d + b_{P,u_i}^{sc}.
    \end{align}
    In the case of control volumes located at boundaries some of the coefficients will be calculated in a different manner. This aspect will be addressed in a later section. For the decoupled iterative solution process of the Navier-Stokes equations it is necessary to reduce the change of each dependent variable in each iteration. Normally this is done by a \textit{under-relaxation} technique, a convex combination of the solution of the linear system present iteration \((n)\) and from the previous iteration \((n-1)\) with the under-relaxation parameter \(\alpha_{u_i}\). Generally speaking this parameter can be chosen individually for each equation. Since there are no rules for choosing this parameters in a general setting the under-relaxation parameter for the velocities is chosen to be equal for all three velocities, \(\alpha_{u_i} = \alpha_{\vec{u}}\). This has the further advantage that, in case the boundary conditions are implemented with the same intention, the linear system for each of the velocities remains unchanged except for the right hand side. This helps to increase memory efficiency.

    Let  the solution for the linear system without under-relaxation be denoted as
    \begin{displaymath}
      \tilde{u}_{P_i}^{(n)} := \frac{b_{P,u_i} - \sum_{F \in NB(P)} a_F u_{F_i}}{a_{P,u_i}},
    \end{displaymath}
    Which is only a formal expression. A convex combination as described yields
    \begin{align*}
      u_{P_i}^{(n)} :&= \alpha_{\vec{u}} \tilde{u}_{P_i}^{(n)} + (1 - \alpha_{\vec{u}} )\, u_{P_i}^{(n-1)} \\[0.5em]
                     &= \alpha_{\vec{u}} \frac{b_{P,u_i} - \sum_{F \in NB(P)} a_F u_{F_i}}{a_{P,u_i}} + (1 - \alpha_{\vec{u}} )\, u_{P_i}^{(n-1)},
    \end{align*}
    an expression that can be modified to derive a linear system whose solution is the under-relaxed velocity
    \begin{displaymath}
      \frac{a_{P,u_i}}{\alpha_{\vec{u}}} u_{P_i} + \sum_{F \in NB(P)} a_F u_{F_i} = b_{P,u_i} + (1 - \alpha_{\vec{u}} )\, u_{P_i}^{(n-1)}. 
    \end{displaymath}
    It must be noted that under-relaxation has not been accounted for in the derivation of the Rhie-Chow interpolation method (REFERENCE). As it has been shown the results depend on the choice of the under-relaxation factor. However under-relaxation is necessary for overall convergence, so in order to be able to compare the results of the different solver algorithms this dependency has to be eliminated. Subsection (REFERENCE) will present a common approach to resolve this dependency.

    \subsection{Discretization of the Temperature Equation}

    The discretization of the temperature equation is performed by the same means as for the momentum balance. The only difference is a simpler diffusion term. The integral form of the temperature equation after applying the Gauss' theorem of integration is
    \begin{displaymath}
    \underbrace{ \iint\limits_S \rho u_j T n_j \mathrm{d}S }_{\text{advective term}}
    - \underbrace{ \iint\limits_S \kappa \frac{\partial T}{\partial x_j} n_j \mathrm{d}V }_{\text{diffusive term}}
    = \underbrace{\iiint\limits_V \vphantom{\frac{ T}{ x_j} } q_T \mathrm{d}V }_{\text{source term}}.
    \end{displaymath}
    Proceeding as in the previous subsections one can now discretize the advective, the diffusive term and the source term. Since this process does not provide further insight, just the final results will be presented. The discretization yields the matrix coefficients as
    \begin{subequations}
      \begin{align}
        a_{F,T} &= \min(\dot{m}_f,0) + \frac{\kappa S_f}{||\vec{x}_P - \vec{x}_F||_2} \\[1em]
        a_{P,T} &= \sum_{F \in NB(P)}\max(0,\dot{m}_f) - \frac{\kappa S_f}{||\vec{x}_P - \vec{x}_F||_2} \\[1em]
        b_{P,T} &= \sum_{F \in NB(P)} \eta  \left(T_F^{(n-1)} \left( \min(\dot{m}_f,0) - \gamma_f \right)\right) \nonumber \\
                &\quad \quad \quad  \quad+ \eta \left( T_{P}^{(n-1)} \left( \max(0,\dot{m}_f) - \left(1 - \gamma_f\right) \right)\right) \nonumber \\[0.5em]
                &\quad + \sum_{F \in NB(P)} \left( \frac{\partial T}{\partial x_j}\right)_f^{(n-1)} \left(\xi_{f_j} - n_{f_j}\right)S_f \nonumber \\[0.5em]
                &\quad + q_{T_P} V_P.
      \end{align}
    \end{subequations}
    Again it is possible though not always necessary, as in the case of the velocities, to under-relax the solution of the resulting linear system with a factor \(\alpha_T\). This can be accomplished as shown in the previous sections.

    \subsection{The SIMPLE-Algorithm}
      
      \subsubsection{Pressure Correction Equation}

      \subsubsection{Characteristic Properties of Projection Methods}

        Under-relaxation, slow convergence, inner iterations outer iterations, relative tolerances, also talk about staggered and collocated variable positioning

      \subsubsection{Dependency on Under-Relaxation -- The Pressure-Weighted Interpolation Method}

        Present an approach for Under-Relaxation independent converged solution. Conduct the proof to show it really works. Present the results for different under-relaxation factors

      \subsubsection{Coupling of Temperature Equation}

        Explicit coupling through source term in momentum balances (Boussinesq-Approximation)

    \subsection{Boundary Conditions on Domain and Block Boundaries}

        Introduce chapter by talking about the nature of partial differential equations (Hackbusch). Always start with a simple implementation for the generic transport equation, then specialize to Navier-Stokes equation.

      \subsubsection{Dirichlet Boundary Condition}

        Only talk about dirichlet for velocities not for pressure.

      %\subsubsection{Neumann Boundary Condition}

        Problematics of outlet boundary conditions

      %\subsubsection{Symmetry Boundary Condition}

      \subsubsection{Wall Boundary Condition}

      Note that there are different approaches. Explain which approach is used and why (memory efficiency)

      \subsubsection{Block Boundary Condition}

      Relevant for block structured grids as for the validity of the domain composition.

      
