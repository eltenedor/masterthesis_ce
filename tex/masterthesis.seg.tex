  \section{Finite Volume Method for Incompressible Flows -- Segregated Approach}

  The purpose of this section is to present the discretization applied to the set of equations (\ref{eq:setofeq}). The applied discretization techniques depend on the different terms of each equation, thus at first every equation will be discretized individually. The finite volume method relies on the discretization of integral equations, which will be derived at the beginning of each subsection that relies on them. Since the system of partial differential equations to be solved always exhibits coupling at least between the dependent variables pressure and velocity a first solution algorithm, namely the \textit{SIMPLE} algorithm addressed to resolve the pressure velocity coupling is introduced. The efficient coupling of the Navier-Stokes equations to the temperature equation is one part of the present thesis and will be addressed in a separate subsection. Furthermore every problem modelled by partial differential equations needs to provide valid boundary conditions. The discretization of those boundary conditions, that are relevant for the present thesis will be presented in their own subsection.

    \subsection{Discretization of the Mass Balance}

    \subsection{Discretization of the Momentum Balance}
      
      \subsubsection{Semi Discretized Linearized Form of the Navier-Stokes Equations}

      \subsubsection{Calculation of Mass Flux -- Rhie-Chow Interpolation}

      \subsubsection{Discretization of the Convective Term}

      \subsubsection{Discretization of the Diffusive Term}

      \subsubsection{Discretization of the Source Term}

      \subsubsection{Assembly of Linear Systems -- Final Form of Equations}
        Coefficients of matrices for momentum are identical except in case of different factors for under-relaxation (underrelaxation (Andersson) )(when does this happen) for the main diagonal coefficient. Small example in code, then show image of assembled system.

    \subsection{Discretization of the Generic Transport Equation}

    \subsection{The SIMPLE-Algorithm}
      
      \subsubsection{Pressure Correction Equation}

      \subsubsection{Characteristic Properties of Projection Methods}

        Under-relaxation, slow convergence, inner iterations outer iterations, relative tolerances, also talk about staggered and collocated variable positioning

      \subsubsection{Dependence on Under-Relaxation -- The Pressure-Weighted Interpolation Method}

        Present an approach for Under-Relaxation independent converged solution. Conduct the proof to show it really works. Present the results for different under-relaxation factors

      \subsubsection{Coupling of Temperature Equation}

        Explicit coupling through source term in momentum balances (Boussinesq-Approximation)

    \subsection{Boundary Conditions on Domain and Block Boundaries}

        Introduce chapter by talking about the nature of partial differential equations (Hackbusch). Always start with a simple implementation for the generic transport equation, then specialize to Navier-Stokes equation.

      \subsubsection{Dirichlet Boundary Condition}

        Only talk about dirichlet for velocities not for pressure.

      %\subsubsection{Neumann Boundary Condition}

        Problematics of outlet boundary conditions

      %\subsubsection{Symmetry Boundary Condition}

      \subsubsection{Wall Boundary Condition}

      Note that there are different approaches. Explain which approach is used and why (memory efficiency)

      \subsubsection{Block Boundary Condition}

      Relevant for block structured grids as for the validity of the domain composition.
      
