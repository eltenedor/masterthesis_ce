  \section{Finite Volume Method for Incompressible Flows -- Segregated Approach}

  The purpose of this section is to present the discretization applied to the set of equations (\ref{eq:setofeq}). The applied discretization techniques depend on the different terms of each equation, thus at first every equation will be discretized individually. The finite volume method relies on the discretization of integral equations, which will be derived at the beginning of each subsection that relies on them. Since the system of partial differential equations to be solved always exhibits coupling at least between the dependent variables pressure and velocity a first solution algorithm, namely the \textit{SIMPLE} algorithm addressed to resolve the pressure velocity coupling is introduced. The efficient coupling of the Navier-Stokes equations to the temperature equation is one part of the present thesis and will be addressed in a separate subsection. Furthermore every problem modelled by partial differential equations needs to provide valid boundary conditions. The discretization of those boundary conditions, that are relevant for the present thesis will be presented in their own subsection.

    \subsection{Discretization of the Mass Balance}

    Integration of equation (\ref{eq:contidiff}) over the integration domain of a single control volume \(P\) yields after the application of Gauss' integration theorem and the additivity of the Riemann integral

    \begin{displaymath}
      \iint\limits_S  u_i n_i \mathrm{d}S = \sum_{f \in \{w,s,b,t,n,e\}} \iint\limits_{S_f}  u_i n_{i} \mathrm{d}S = 0
    \end{displaymath}

    In the present work the mass balance is discretized using the midpoint rule for the surface integrals and linear interpolation of the velocity to to center of mass of the surface. This leads to the following form of the mass balance: 

    \begin{align*}
      \sum_{f \in \{w,s,b,t,n,e\}} u_{i_f} n_{f_i} S_f 
      &= u_{i_w} n_{w_i} S_w + u_{i_e} n_{e_i} S_e 
       + u_{i_s} n_{s_i} S_s + u_{i_n} n_{n_i} S_n 
       + u_{i_b} n_{b_i} S_b + u_{i_t} n_{t_i} S_t  \\
      &= ( \gamma_w u_{i_W} + (1 - \gamma_w ) u_{i_P} ) n_{w_i} S_w + ( \gamma_s u_{i_S} + (1 - \gamma_s ) u_{i_P} ) n_{s_i} S_s \\
      &\quad + ( \gamma_b u_{i_B} + (1 - \gamma_b ) u_{i_P} ) n_{b_i} S_b + ( \gamma_t u_{i_T} + (1 - \gamma_t ) u_{i_P} ) n_{t_i} S_t \\
      &\quad + ( \gamma_n u_{i_N} + (1 - \gamma_n ) u_{i_P} ) n_{n_i} S_n + ( \gamma_e u_{i_E} + (1 - \gamma_e ) u_{i_P} ) n_{e_i} S_e \\
      & =  0,
    \end{align*}

    where \( \gamma_f \) for \( f \in \{w,e,s,n,b,t\} \) is the geometrical interpolation factor.

    \subsection{Discretization of the Momentum Balance}

      The stationary momentum balance integrated over a single control volume \(P\) reads as

      \begin{displaymath}
        \underbrace{\iint\limits_S (\rho u_i u_j)n_j \mathrm{d}S}_{\text{convection term}}
        - \underbrace{\iint\limits_S \left(\mu \left( \frac{\partial u_i}{\partial x_j} + \frac{\partial u_j}{\partial x_i}\right)\right)n_j \mathrm{d}S}_{\text{diffusion term}}
        = - \overbrace{\iiint_V \frac{\partial p}{\partial x_i} \mathrm{d}V}^{\text{sourceterm pressure}}
        - \overbrace{\iiint_V \rho \beta \left(T - T_0\right) \mathrm{d}V}^{\text{sourceterm temperature}}
      \end{displaymath}

      where the different terms to be addressed individually in the following sections are indicated. Note that the form of this equation has been modified by using Gauss' integration theorem The terms residing on the left will be treated in an implicit manner whereas the terms on the right will be treated explicitly. 

      \subsubsection{Calculation of Mass Flux -- Rhie-Chow Interpolation}

      \subsubsection{Linearization and Discretization of the Convective Term}

      The convective term \(\rho u_i u_j\) of the Navier-Stokes equations is the reason for the non-linearity of the equations. In order to deduce a set of linear algebraic equations from the Navier-Stokes equations this term has to be linearized. As introduced in section (REFERENCE), the non linearity will be dealt with by means of an iterative process, the Picard iteration. The part dependent of the non dominant dependent variable therefore will be approximated by its value from the previous iteration as \( \rho u_i^{(n)} u_j^{(n)} \approx \rho u_i^{(n)} u_j^{(n-1)} \). Using the additivity of the Riemann integral the first step is to decompose the surface integral into individual contributions from each boundary face of the control volume \(P\)
      
      \begin{displaymath}
      \iint\limits_S (\rho u_i u_j)n_j \mathrm{d}S
      = \sum_{f \in \{w,s,b,t,n,e\}} \iint\limits_{S_f}\rho u_{i} u_{j} n_{j} \mathrm{d}S
      \approx \sum_{f \in \{w,s,b,t,n,e\}} \iint\limits_{S_f} \rho u_{i}^{(n)} u_{j}^{(n-1)} n_{j} \mathrm{d}S,
      \end{displaymath}

      where the last approximation is due to the linearization. The convective term is then discretized by using the midpoint integration rule and a blended interpolation scheme, that combines upwind and linear interpolation. The blending combines one discretization of lower order (Upwind) with one of higher order (Linear Interpolation). Since this term is treated explicitly the computational stencil is kept small and the diagonal dominance of the resulting system matrix is increased which benefits the convergence of a linear solver. For simplicity all further derivations are presented for the boundary face \(S_e\). The discretization leads to

      \begin{displaymath}
        \iint\limits_{S_f} \rho u_{i}^{(n)} u_{j}^{(n-1)} n_{j} \mathrm{d}S 
        \approx \rho u_{i_e}^{(*)} u_{j_e}^{(n-1)} n_{e_j} S_e,  
      \end{displaymath}

      where the interpolated velocity \(u_{i_e}^{(*)}\) is calculated as

      \begin{displaymath}
        u_{i_e}^{(*)} =  \left(\max(0,u_{i_P}^{(n)}) + \min(u_{i_E}^{(n)},0)\right) 
        - \eta \left[ \left(\max(0,u_{i_P}^{(n-1)}) + \min(u_{i_E}^{(n-1)},0)\right) - \left( \gamma_e u_{j_E}^{(n-1)} + \left(1 - \gamma_e \right) u_{j_P}^{(n-1)}\right)
         \right] .
      \end{displaymath}

      \vspace{0.3cm} The blending parameter \(\eta \in [0,1]\) is chosen to be \(1\), which leads to an approximation of second order in a fully converged solution, since the implicit and explicit upwind interpolation coincide in case of a converged solution.

      \subsubsection{Discretization of the Diffusive Term}

      \subsubsection{Discretization of the Source Term}

      \subsubsection{Assembly of Linear Systems -- Final Form of Equations}
        Coefficients of matrices for momentum are identical except in case of different factors for under-relaxation (underrelaxation (Andersson) )(when does this happen) for the main diagonal coefficient. Small example in code, then show image of assembled system.

    \subsection{Discretization of the Generic Transport Equation}

    \subsection{The SIMPLE-Algorithm}
      
      \subsubsection{Pressure Correction Equation}

      \subsubsection{Characteristic Properties of Projection Methods}

        Under-relaxation, slow convergence, inner iterations outer iterations, relative tolerances, also talk about staggered and collocated variable positioning

      \subsubsection{Dependence on Under-Relaxation -- The Pressure-Weighted Interpolation Method}

        Present an approach for Under-Relaxation independent converged solution. Conduct the proof to show it really works. Present the results for different under-relaxation factors

      \subsubsection{Coupling of Temperature Equation}

        Explicit coupling through source term in momentum balances (Boussinesq-Approximation)

    \subsection{Boundary Conditions on Domain and Block Boundaries}

        Introduce chapter by talking about the nature of partial differential equations (Hackbusch). Always start with a simple implementation for the generic transport equation, then specialize to Navier-Stokes equation.

      \subsubsection{Dirichlet Boundary Condition}

        Only talk about dirichlet for velocities not for pressure.

      %\subsubsection{Neumann Boundary Condition}

        Problematics of outlet boundary conditions

      %\subsubsection{Symmetry Boundary Condition}

      \subsubsection{Wall Boundary Condition}

      Note that there are different approaches. Explain which approach is used and why (memory efficiency)

      \subsubsection{Block Boundary Condition}

      Relevant for block structured grids as for the validity of the domain composition.
      
