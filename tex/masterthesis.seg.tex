  \section{Finite Volume Method for Incompressible Flows -- Segregated Approach}
to be solved  Since this thesis focusses on the analysis of a fully coupled numerical solution algorithm, special emphasis is put on the comparison between segregated and coupled solution methods for solving the set of partial differential equations. Concretely this is achieved by presenting the discretization technique used for the application of the SIMPLE algorithm and later on modifying those results to be applicable to use in a fully coupled solution algorithm. Since on top of the Navier-Stokes equations a scalar transport equation for the temperature is solved, different methods to realize velocity-temperature-coupling and vice versa are discussed.

    \subsection{Discretization of the Mass Balance}

    \subsection{Discretization of the Momentum Balance}
      
      \subsubsection{Semi Discretized Linearized Form of the Navier-Stokes Equations}

      \subsubsection{Calculation of Mass Flux -- Rhie-Chow Interpolation}

      \subsubsection{Discretization of the Convective Term}

      \subsubsection{Discretization of the Diffusive Term}

      \subsubsection{Discretization of the Source Term}

      \subsubsection{Assembly of Linear Systems -- Final Form of Equations}
        Coefficients of matrices for momentum are identical except in case of different factors for under-relaxation (underrelaxation (Andersson) )(when does this happen) for the main diagonal coefficient. Small example in code, then show image of assembled system.

    \subsection{Discretization of the Generic Transport Equation}

    \subsection{The SIMPLE-Algorithm}
      
      \subsubsection{Pressure Correction Equation}

      \subsubsection{Characteristic Properties of Projection Methods}

        Under-relaxation, slow convergence, inner iterations outer iterations, relative tolerances, also talk about staggered and collocated variable positioning

      \subsubsection{Dependence on Under-Relaxation -- The Pressure-Weighted Interpolation Method}

        Present an approach for Under-Relaxation independent converged solution. Conduct the proof to show it really works. Present the results for different under-relaxation factors

      \subsubsection{Coupling of Temperature Equation}

        Explicit coupling through source term in momentum balances (Boussinesq-Approximation)

    \subsection{Boundary Conditions on Domain and Block Boundaries}

        Introduce chapter by talking about the nature of partial differential equations (Hackbusch). Always start with a simple implementation for the generic transport equation, then specialize to Navier-Stokes equation.

      \subsubsection{Dirichlet Boundary Condition}

        Only talk about dirichlet for velocities not for pressure.

      \subsubsection{Neumann Boundary Condition}

        Problematics of outlet boundary conditions

      \subsubsection{Symmetry Boundary Condition}

      \subsubsection{Wall Boundary Condition}

      Note that there are different approaches. Explain which approach is used and why (memory efficiency)

      \subsubsection{Block Boundary Condition}
      
