  \section{Finite Volume Method for Incompressible Flows -- Segregated Approach}

    \subsection{Discretization of the Mass Balance}

    \subsection{Discretization of the Momentum Balance}
      
      \subsubsection{Semi Discretized Linearized Form of the Navier-Stokes Equations}

      \subsubsection{Calculation of Mass Flux -- Rhie-Chow Interpolation}

      \subsubsection{Discretization of the Convective Term}

      \subsubsection{Discretization of the Diffusive Term}

      \subsubsection{Discretization of the Source Term}

      \subsubsection{Assembly of Linear Systems -- Final Form of Equations}
        Coefficients of matrices for momentum are identical except in case of different factors for under-relaxation (underrelaxation (Andersson) )(when does this happen) for the main diagonal coefficient. Small example in code, then show image of assembled system.

    \subsection{Discretization of the Generic Transport Equation}

    \subsection{The SIMPLE-Algorithm}
      
      \subsubsection{Pressure Correction Equation}

      \subsubsection{Characteristic Properties of Projection Methods}

      \subsubsection{Dependence on Under-Relaxation -- The Pressure-Weighted Interpolation Method}

        Under-relaxation, slow convergence, inner iterations outer iterations, relative tolerances, also talk about staggered and collocated variable positioning

      \subsubsection{Coupling of Temperature Equation}

    \subsection{Boundary Conditions on Domain and Block Boundaries}

        Introduce chapter by talking about the nature of partial differential equations (Hackbusch). Always start with a simple implementation for the generic transport equation, then specialize to Navier-Stokes equation.

      \subsubsection{Dirichlet Boundary Condition}

        Only talk about dirichlet for velocities not for pressure.

      \subsubsection{Neumann Boundary Condition}

        Problematics of outlet boundary conditions

      \subsubsection{Symmetry Boundary Condition}

      \subsubsection{Wall Boundary Condition}

      \subsubsection{Block Boundary Condition}
      
