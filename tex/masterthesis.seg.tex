\section{Implicit Finite Volume Method for Incompressible Flows -- Segregated Approach}

  The purpose of this section is to present the discretization applied to the set of equations (\ref{eq:completeset}). Since the system of partial differential equations to be solved always exhibits a coupling at least between the dependent variables pressure and velocity a first solution algorithm, namely the \emph{SIMPLE} algorithm addressed to resolve the pressure velocity coupling is introduced. Methods of calculating mass fluxes and the detailed derivation of all coefficients that result from the discretization process is presented. The discretization of those boundary conditions, that are relevant for the present thesis will be presented in their own subsection.

\subsection{Discretization of the Mass Balance}
Integration of equation (\ref{eq:contidiff}) over the integration domain of a single control volume \(P\) yields after the application of Gauss' integration theorem and the additivity of the Riemann integral
\begin{displaymath}
\iint\limits_S u_i n_i \mathrm{d}S = \sum_{f \in \{w,s,b,t,n,e\}} \iint\limits_{S_f} u_i n_{i} \mathrm{d}S = 0.
\end{displaymath}
In the present work the mass balance is discretized using the midpoint rule for the surface integrals and linear interpolation of the velocity to to center of mass of the surface. This leads to the following form of the mass balance:
\begin{align*}
\sum_{f \in \{w,s,b,t,n,e\}} u_{i_f} n_{f_i} S_f = 0,
\end{align*}
where no interpolation to attain the values of \(u_i\) at the face \(S_f\) is performed yet, since the straightforward linear interpolation will lead to undesired oscillations in the solution fields. An interpolation method to circumvent this so called \emph{checker boarding} effect is presented in section \ref{sec:massflux}.

  \subsection{Calculation of Mass Fluxes and the Pressure-Weighted Interpolation Method}
  \label{sec:massflux}

  The advantages of using a cell-centered variable arrangement are evident: The treatment of non-orthogonality is simplified and the conservation property of finite volume methods is retained \cite{choi99,majumdar88,miller88,zhang14}. A major drawback if SIMPLE-Type algorithms, to be introduced in section \ref{sec:simple}, are used is that pressure field may delink which will then lead to unphysical oscillations in both the pressure and the velocity results. This . If the oscillations are severe enough the solution algorithm might get unstable and diverge. The described decoupling occurs, when the pressure gradient in the momentum balances and the mass fluxes in the continuity equation are discretized using central differences. A common practice to eliminate this behaviour is the use of a momentum interpolation technique, also known as \emph{Rhie-Chow Interpolation} \cite{rhie82}. The original interpolation scheme however doesn't guarantee a unique solution, independent of the amount of under-relaxation. The performance of one of the algorithms that are used in the present thesis heavily relies on the under-relaxation of variables to accomplish stability. Furthermore the original method as proposed by \cite{rhie82} does not account for large body forces which also may lead to unphysical results. This issues will be addressed in this subsection which at the end will present an interpolation method that assures an under-relaxation independent solution, the \emph{pressure-weighted interpolation method} \cite{miller88}.

  Starting point of the pressure-weighted interpolation method are the discretized momentum balances at node \(P\) and the neighbouring node \(Q\). The discretization for a finite volume methods and details including the incorporation of under-relaxation factors will be handled in subsection \ref{sec:segdiscretization}. The semi-discrete implicit momentum balances read if one solves for the velocity at node \(P\) or \(Q\)
  \begin{subequations}
  \begin{align}
    u_{i,P}^{(n)} 
    &= 
    - \frac{\alpha_{\vec{u}_P}}{a_{P,u_i}} \left(\sum_{F \in NB(P)} a_{F,u_i} u_{i,F}^{(n)}
    +                                     b_{P,u_i}^{(n-1)} 
    -                                     V_P\left(\frac{\partial p}{\partial x_i}\right)_P^{(n-1)} \right)
    + \left(1 - \alpha_{\vec{u}}\right) u_{i,P}^{(n-1)}  \\[1em]
    \text{and} \quad
    u_{i,Q}^{(n)} 
    &= 
    - \frac{\alpha_{\vec{u}_Q}}{a_{Q,u_i}} \left(\sum_{F \in NB(Q)} a_{F,u_i} u_{i,F}^{(n)}
    +                                     b_{Q,u_i}^{(n-1)} 
    -                                     V_Q\left(\frac{\partial p}{\partial x_i}\right)_Q^{(n-1)}   \right)
    + \left(1 - \alpha_{\vec{u}}\right) u_{i,Q}^{(n-1)}
  \end{align}
  \end{subequations}
  where the superscript \((n-1)\) denotes the previous outer iteration number. The reader should note, that the pressure gradient has not been discretized yet. This has the advantage that the selective interpolation technique \cite{schaefer99} can be applied, which is crucial for the elimination of the mentioned oscillations. In almost the same manner a semi-discrete implicit momentum can be formulated for a virtual control volume located between nodes \(P\) and \(Q\) (PICTURE)
  \begin{equation}
    \label{eq:virtualu}
    u_{i,f}^{(n)} 
    = 
    - \frac{\alpha_{\vec{u}_f}}{a_{f,u_i}} \left(\sum_{F \in NB(f)} a_{F,u_i} u_{i,F}^{(n)} 
    +                                     b_{f,u_i}^{(n-1)} 
    -                                     V_f\left(\frac{\partial p}{\partial x_i}\right)_f^{(n-1)}  \right)
    + \left(1 - \alpha_{\vec{u}}\right) u_{i,f}^{(n-1)}.
  \end{equation}
  To guarantee convergence of this expression of \(u_{i,f}\) under-relaxation is necessary \cite{majumdar88}. To eliminate the undefined artifacts surging form the virtualization of a control volume the following assumptions have to be made to derive a closed expression for the velocity on the boundary face \(S_f\)
  \begin{subequations}
    \label{eq:approxpwim}
  \begin{align}
    \frac{\alpha_{\vec{u}_f}}{a_{f,u_i}} \left(\sum_{F \in NB(f)} a_{F,u_i} u_{i,F}^{(n)} \right)
    &\approx
    \left(1-\gamma_f\right) \frac{\alpha_{\vec{u}_P}}{a_{P,u_i}} \left(\sum_{F \in NB(P)} a_{F,u_i} u_{i,F}^{(n)} \right)
    +
    \gamma_f \frac{\alpha_{\vec{u}_Q}}{a_{Q,u_i}} \left(\sum_{F \in NB(Q)} a_{F,u_i} u_{i,F}^{(n)} \right) \\[1em]
    \text{and} \quad
    \frac{\alpha_{\vec{u}_f}}{a_{f,u_i}} 
    &\approx
    \left(1-\gamma_f\right) \frac{\alpha_{\vec{u}_P}}{a_{P,u_i}} 
    +
    \gamma_f \frac{\alpha_{\vec{u}_Q}}{a_{Q,u_i}},
  \end{align}
  \end{subequations}
  where \(\gamma_f\) is a geometric interpolation factor. 

  Using the assumptions made in equation (\ref{eq:approxpwim}) the expression in equation (\ref{eq:virtualu}) can be closed in a way that it only depends on the variable values in node \(P\) and \(Q\)
  \begin{align}
    \label{eq:closepwim}
    u_{i,f}^{(n)} 
    &\approx 
    \left(1-\gamma_f\right)  \left( -\frac{\alpha_\vec{u}}{a_{P,u_i}} \sum_{F \in NB(P)} a_{F,u_i} u_{i,F}^{(n)} \right)
    +\gamma_f  \left( -\frac{\alpha_\vec{u}}{a_{Q,u_i}} \sum_{F \in NB(Q)} a_{F,u_i} u_{i,F}^{(n)}  \right) \nonumber \\[1em]
    &\quad\quad+ \frac{\alpha_\vec{u}}{a_{f,u_i}}b_{f,u_i}^{(n-1)} 
    - \frac{\alpha_{\vec{u}_f}}{a_{f,u_i}}V_f\left(\frac{\partial p}{\partial x_i}\right)_f^{(n-1)} 
    + \left(1 - \alpha_{\vec{u}}\right) u_{i,f}^{(n-1)} \nonumber \\[1em]
    &=
    \left(1-\gamma_f\right) u_{i,P}^{(n)} - \left(1 - \gamma_f\right) \left(  b_{Q,u_i}^{(n-1)} - V_Q \left(\frac{\partial p}{\partial x_i}\right)_Q^{(n-1)} \right) \nonumber \\[1em]
    &\quad\quad+\gamma_f  u_{i,Q}^{(n)} - \gamma_f\left( b_{Q,u_i}^{(n-1)} - V_Q \left(\frac{\partial p}{\partial x_i}\right)_Q^{(n-1)}  \right) \nonumber \\[1em]
    &\quad\quad+ \frac{\alpha_{\vec{u}_f}}{a_{f,u_i}}b_{f,u_i}^{(n-1)} 
    - \frac{\alpha_{\vec{u}_f}}{a_{f,u_i}}V_f\left(\frac{\partial p^{(n-1)}}{\partial x_i}\right)_f 
    + \left(1 - \alpha_{\vec{u}}\right) u_{i,f}^{(n-1)} \nonumber\\[1em]
    &=
    \left[\left(1 - \gamma_f\right) u_{i,P}^{(n)} + \gamma_f u_{i,Q}^{(n)} \right] \nonumber\\[1em]
    &\quad\quad - 
    \left[ 
    \left(\left(1 - \gamma_f\right) \frac{\alpha_\vec{u} V_P}{a_{P,u_i}} + \gamma_f \frac{\alpha_\vec{u} V_Q}{a_{Q,u_i}}\right)
    \left(\frac{\partial p}{\partial x_i}\right)_f^{(n-1)} 
    - \left(1 - \gamma_f \right) \frac{\alpha_\vec{u} V_P}{a_{P,u_i}}\left( \frac{\partial p}{\partial x_i} \right)_P^{(n-1)} 
    - \gamma_f \frac{\alpha_\vec{u} V_Q}{a_{Q,u_i}}\left(\frac{\partial p}{\partial x_i}\right)_Q^{(n-1)}
    \right] \nonumber \\[1em]
    &\quad\quad + \left(1 - \alpha\right) \left[ u_{i,f}^{(n-1)} - \left(1 - \gamma_f\right) u_{i,P}^{(n-1)} - \gamma_f \, u_{i,Q}^{(n-1)} \right] \nonumber \\[1em]
    &\approx
    \left[\left(1 - \gamma_f\right) u_{i,P}^{(n)} + \gamma_f u_{i,Q}^{(n)} \right] \nonumber\\[1em]
    &\quad\quad - 
    \left(\left(1 - \gamma_f\right) \frac{\alpha_\vec{u} V_P}{a_{P,u_i}} + \gamma_f \frac{\alpha_\vec{u} V_Q}{a_{Q,u_i}}\right)
    \left[ 
    \left(\frac{\partial p}{\partial x_i}\right)_f^{(n-1)} 
    - \left(1 - \gamma_f \right) \left( \frac{\partial p}{\partial x_i} \right)_P^{(n-1)} 
    - \gamma_f \left(\frac{\partial p}{\partial x_i}\right)_Q^{(n-1)}
    \right] \nonumber \\[1em]
    &\quad\quad + \left(1 - \alpha_\vec{u}\right) \left[ u_{i,f}^{(n-1)} - \left(1 - \gamma_f\right) u_{i,P}^{(n-1)} - \gamma_f \, u_{i,Q}^{(n-1)} \right].
  \end{align}
  It should be noted that the argumentation that led to the last expression is that the task of the pressure gradient corrector is to suppress oscillations. 
  
  If there are no oscillations this part does not come into consideration. This is however true on equidistant grids, where \(\gamma_f = 1/2\) and central differences are used to calculate the gradients. On arbitrary orthogonal grids another modification has to be performed which is based on the special case of the mean value theorem of differential calculus and the following 
\begin{prop}
  Let \(x_1,x_2 \in \mathbb{R}\) with \(x_1 \neq x_2\) and \(p(x) = a_0 + a_1 x + a_2 x^2\) a real polynomial function. Then 
  \begin{displaymath}
    \frac{dp}{dx}\left(\frac{x_1+x_2}{2}\right) = \frac{p(x_2) - p(x_1)}{x_2 - x_1},
  \end{displaymath}
  i.e. the slope of the secant equals the value of the first derivative of \(p\) exactly half the way between \(x_1\) and \(x_2\).
\end{prop}

\begin{proof}
  Evaluation of the derivative yields
  \begin{displaymath}
      \frac{dp}{dx}\left(\frac{x_1+x_2}{2}\right) = a_1 + 2 a_2 \frac{x_1 + x_2}{2} = a_1 + a_2(x_1 + x_2).
  \end{displaymath}
  On the other hand the slope of the secant, using the third binomial rule can be expressed as
  \begin{displaymath}
    \begin{array}{ll}
    \frac{p(x_2) - p(x_1)}{x_2 - x_1} 
  &= \frac{a_0 + a_1 x_2 + a_2 x_2^2 - \left(a_0 + a_1 x_1 + a_2 x_1 ^2\right)}{x_2 - x_1} \\[1.0em]
    \quad &= \frac{a_1 (x_2 - x_1) + a_2 \left(x_2^2 - x_1^2\right)}{x_2 - x_1} \\[1.0em]
    \quad &= a_1 + a_2 (x_2 + x_1).
  \end{array}
  \end{displaymath}
  The comparison of both expressions completes the proof.
\end{proof}
  It is desirable for the pressure corrector to vanish independent of the grid spacing if the profile of the pressure is quadratic and hence does not exhibit oscillations. According to the preceding proposition this can be accomplished by modifying equation (\ref{eq:closepwim}) to average the pressure gradients from node \(P\) and \(Q\) instead of interpolating linearly
  \begin{align}
    u_{i,f}^{(n)} 
    &=
    \left[\left(1 - \gamma_f\right) u_{i,P}^{(n-1)} + \gamma_f u_{i,Q}^{(n-1)} \right] \nonumber\\[1em]
    &\quad\quad - 
    \left(\left(1 - \gamma_f\right) \frac{\alpha_\vec{u} V_P}{a_{P,u_i}} + \gamma_f \frac{\alpha_\vec{u} V_Q}{a_{Q,u_i}}\right)
    \left[ 
    \left(\frac{\partial p}{\partial x_i}\right)_f^{(n-1)} 
    - \frac{1}{2} \left( \left( \frac{\partial p}{\partial x_i} \right)_P^{(n-1)} 
    + \left(\frac{\partial p}{\partial x_i}\right)_Q^{(n-1)} \right)
    \right] \nonumber \\[1em]
    &\quad\quad + \underline{\left(1 - \alpha_\vec{u}\right) \left[ u_{i,f}^{(n-1)} - \left(1 - \gamma_f\right) u_{i,P}^{(n-1)} - \gamma_f \, u_{i,Q}^{(n-1)} \right]}.
  \end{align}
  Comparing this final expression with the standard interpolation scheme it is evident that the underlined term is not taken into consideration \cite{ferziger02} normally. However section \ref{sec:independence} shows that neglecting this term creates under-relaxation factor dependent results indeed.

  PROOF THAT THIS GUARANTEES INDEPENDENT RESULTS MATHEMATICALLY

  \subsection{Implicit Pressure Correction and the SIMPLE Algorithm}
  \label{sec:simple}
  
  The goal of finite volume methods is to deduce a system of linear algebraic equations from a partial differential equation. In the case of the momentum balances the general structure of this linear equations is
  \begin{equation}
    \label{eq:linfinal}
    u_{i,P}^{(n)} 
    = 
    - \frac{\alpha_{\vec{u}_P}}{a_{P,u_i}} \left(\sum_{F \in NB(P)} a_{F,u_i} u_{i,F}^{(n)}
    +                                     b_{P,u_i}^{(n-1)} 
    -                                     V_P\left(\frac{\partial p}{\partial x_i}\right)_P^{(n-1)} \right)
    + \left(1 - \alpha_{\vec{u}}\right) u_{i,P}^{(n-1)}  
  \end{equation}
  where the pressure gradient has been discretized only symbolically and \(b_{P,u_i}\) denotes the source term. At this stage the equations are still coupled and non-linear. As described in section \ref{sec:nonlinear} the Picard iteration process has been used to linearize the equations. Every momentum balance equation only depends on the one dominant variable \(u_i\). Furthermore the coupling of the momentum balances through the convective term \((u_i u_j)\) has been resolved in the process of linearization. The decoupled momentum balances can then be solved sequentially for the dominant variable \(u_i\). All coefficients \(a_{\{P,F\},u_i}\), the source term and the pressure gradient will be evaluated explicitly by using results of the preceding outer iteration \((n-1)\) until the non-linear equations are fulfilled up to a certain tolerance (\ref{sec:convergence}). This leads to the following linear equation 
  \begin{equation}
    \label{eq:nodeinter}
    u_{i,P}^{(n*)} 
    = 
    - \frac{\alpha_{\vec{u}_P}}{a_{P,u_i}} \left(\sum_{F \in NB(P)} a_{F,u_i} u_{i,F}^{(n*)}
    +                                     b_{P,u_i}^{(n-1)} 
    -                                     V_P\left(\frac{\partial p}{\partial x_i}\right)_P^{(n-1)} \right)
    + \left(1 - \alpha_{\vec{u}}\right) u_{i,P}^{(n-1)}  
\end{equation}
Here \((*)\) indicates that the solution of this equation still needs to be corrected to also fulfill the discretized mass balance
  \begin{equation}
    \label{eq:contisemi}
    \sum_{F \in NB(P)} \left(u_i\right)^{(n)}_f n_i S_f = 0.
  \end{equation}

  Applying the same procedure as in section \ref{sec:massflux} to equation (\ref{eq:lininter}) results in the following expression for the face velocities after solving the discretized momentum balances
  \begin{align}
    \label{eq:faceinter}
    u_{i,f}^{(n*)} 
    &=
    \left[\left(1 - \gamma_f\right) u_{i,P}^{(n*)} + \gamma_f u_{i,Q}^{(n*)} \right] \nonumber \\[1em]
    &\quad\quad - 
    \left(\left(1 - \gamma_f\right) \frac{\alpha_\vec{u} V_P}{a_{P,u_i}} + \gamma_f \frac{\alpha_\vec{u} V_Q}{a_{Q,u_i}}\right)
    \left[ 
    \left(\frac{\partial p}{\partial x_i}\right)_f^{(n-1)} 
    - \left( 1 - \gamma_f \right) \left( \frac{\partial p}{\partial x_i} \right)_P^{(n-1)} 
    - \gamma_f \left(\frac{\partial p}{\partial x_i}\right)_Q^{(n-1)}
    \right] \nonumber \\[1em]
    &\quad\quad + \left(1 - \alpha_\vec{u}\right) \left[ u_{i,f}^{(n-1)} - \left(1 - \gamma_f\right) u_{i,P}^{(n-1)} - \gamma_f \, u_{i,Q}^{(n-1)} \right].
  \end{align}

  The lack of an equation with the pressure as dominant variable leads to the necessity to alter the mass balance as the only equation left. Methods of this type are called projection methods. A common class of algorithms of this family of methods uses an equation for the additive pressure correction \(p'\) instead of the pressure itself and enforces continuity by correcting the velocities with an additive corrector \(u_i'\) as in
  \begin{displaymath}
    u_{i,P}^{(n)} =  u_{i,P}^{(n*)}  + u_{i,P}',\quad u_{i,f}^{(n)} =  u_{i,f}^{(n*)}  + u_{i,f}' \quad \text{and} \quad   p_P^{(n)} =  p_P^{(n-1)}  + p_P'.
  \end{displaymath}
  It is now possible to formulate the discretized momentum balance for the corrected velocities and the corrected pressure as
  \begin{equation}
    \label{eq:nodecorr}
    u_{i,P}^{(n)} 
    = 
    - \frac{\alpha_{\vec{u}_P}}{a_{P,u_i}} \left(\sum_{F \in NB(P)} a_{F,u_i} u_{i,F}^{(n)}
    +                                     b_{P,u_i}^{(n-1)} 
    -                                     V_P\left(\frac{\partial p}{\partial x_i}\right)_P^{(n)} \right)
    + \left(1 - \alpha_{\vec{u}}\right) u_{i,P}^{(n-1)}  .
  \end{equation}
  It should be noted that the only difference to the equation that will be solved in the next outer iteration is that the source term \(b_{P,u_i}\) has not been updated yet. The same applies to the equation for the face velocity \(u_{i,f}\)
  \begin{align}
    \label{eq:facecorr}
    u_{i,f}^{(n)} 
    &=
    \left[\left(1 - \gamma_f\right) u_{i,P}^{(n)} + \gamma_f u_{i,Q}^{(n)} \right] \nonumber\\[1em]
    &\quad\quad - 
    \left(\left(1 - \gamma_f\right) \frac{\alpha_\vec{u} V_P}{a_{P,u_i}} + \gamma_f \frac{\alpha_\vec{u} V_Q}{a_{Q,u_i}}\right)
    \left[ 
    \left(\frac{\partial p}{\partial x_i}\right)_f^{(n)} 
    -  \left(1 - \gamma_f\right) \left( \frac{\partial p}{\partial x_i} \right)_P^{(n)} 
    - \gamma_f \left(\frac{\partial p}{\partial x_i}\right)_Q^{(n)} 
    \right] \nonumber \\[1em]
    &\quad\quad + \left(1 - \alpha_\vec{u}\right) \left[ u_{i,f}^{(n-1)} - \left(1 - \gamma_f\right) u_{i,P}^{(n-1)} - \gamma_f \, u_{i,Q}^{(n-1)} \right].
  \end{align}
  To couple velocity and pressure correctors one can subtract equations (\ref{eq:nodeinter}) from (\ref{eq:nodecorr}), (\ref{eq:faceinter}) from (\ref{eq:facecorr}) to get
  \begin{align}
    \label{eq:nodeprime}
    u_{i,P}' 
    &=  
    - \frac{\alpha_{\vec{u}_P}}{a_{P,u_i}} \left(\underline{\sum_{F \in NB(P)} a_{F,u_i} u_{i,F}'}
    - V_P\left(\frac{\partial p'}{\partial x_i}\right)_P^{(n)} \right) \\[1em]
    \label{eq:faceprime}
    u_{i,f}' 
    &= 
    \left[\left(1 - \gamma_f\right) u_{i,P}' + \gamma_f u_{i,Q}' \right] 
    - 
    \left(\left(1 - \gamma_f\right) \frac{\alpha_\vec{u} V_P}{a_{P,u_i}} + \gamma_f \frac{\alpha_\vec{u} V_Q}{a_{Q,u_i}}\right)
    \left[ 
    \left(\frac{\partial p}{\partial x_i}\right)_f' 
    - \left( 1 - \gamma_f \right) \left( \frac{\partial p}{\partial x_i} \right)_P' 
    - \gamma_f \left(\frac{\partial p}{\partial x_i}\right)_Q' 
    \right],
  \end{align}
  The majority of the class of pressure correction algorithms has this equations as a common basis. Each algorithm then introduces special distinguishable approximations of the velocity corrections that are, at the moment of solving the pressure equation, still unknown. The method used in the present work is the SIMPLE Algorithm (Semi-Implicit Method for Pressure-Linked Equations \cite{spalding}). The approximation this algorithm performs is severe since the term containing the unknown velocity corrections is dropped entirely. The respective term has been underlined in equation \ref{eq:nodeprime} accordingly. Since the global purpose of the presented method is to enforce continuity by implicitly calculating a pressure correction, the velocity correction has to be expressed solely in terms of the pressure correction. This can be accomplished by substituting equation (\ref{eq:nodeprime}) in to equation (\ref{eq:faceprime}). This gives
  \begin{align}
    u_{i,f}' 
    &= 
    - 
    \left(\left(1 - \gamma_f\right) \frac{\alpha_\vec{u} V_P}{a_{P,u_i}} + \gamma_f \frac{\alpha_\vec{u} V_Q}{a_{Q,u_i}}\right)
    \left(\frac{\partial p}{\partial x_i}\right)_f' 
  \end{align}
  which is then, together with (\ref{eq:faceinter}) substituted into the discretized continuity equation (\ref{eq:contisemi}) to obtain
  \begin{equation}
    \label{eq:presscorr}
    \sum_{F \in NB(P)} \left(\left(1 - \gamma_f\right) \frac{\alpha_\vec{u} V_P}{a_{P,u_i}} + \gamma_f \frac{\alpha_\vec{u} V_F}{a_{F,u_i}}\right)
    \left(\frac{\partial p}{\partial x_i}\right)_f' 
    = b_{P,p}
    \quad,
  \end{equation}
  where the right hand side \(b_{P,p}\) is defined as
  \begin{equation}
    b_{P,p} := \sum_{F \in NB(P)} u_{i,f}^{(n*)} n_i S_f
  \end{equation}
  
  The approximation performed in the SIMPLE algorithm affects convergence in a way that the pressure correction has to be under-relaxed with a parameter \(\alpha_p \in [0,1]\)
  \begin{equation}
    \label{eq:pressupdate}
    p_P^{(n)} = p_P^{(n-1)} + \alpha_{\vec{p}} p_P'
  \end{equation}

  Generally the SIMPLE algorithm can be represented by the following iterative procedure as in Algorithm \ref{al:simple}.
\alglanguage{pseudocode}
\begin{algorithm}
\label{al:simple}
\caption{SIMPLE Algorithm}
\begin{algorithmic}
%\Procedure{B-TREE-DELETE}{$x,k$}
\State{\textit{INITIALIZE} variables}
\While{(convergence criterion not accomplished)}
\State{\textit{SOLVE} linearized momentum balances, equation \textbf{(\ref{eq:nodeinter})}}
\State{\textit{CALCULATE} mass fluxes using \textbf{(\ref{eq:facecorr})} or \textbf{(\ref{eq:facecorr2}})}
\State{\textit{SOLVE} pressure correction equation to assure continuity, equation \textbf{(\ref{eq:presscorr})}}
\State{\textit{UPDATE} pressure using \textbf{(\ref{eq:pressupdate})}}
\State{\textit{UPDATE} velocities and mass fluxes using \textbf{(\ref{eq:nodeprime})}}
\If{(Coupled scalar equation)}
  \State{\textit{SOLVE} scalar equation as described in \textbf{(\ref{sec:discretetemperature})}}
  \EndIf
\EndWhile
\end{algorithmic}
\end{algorithm}

As shown an section \ref{sec:massflux} the behaviour of the pressure weighted interpolation method on non-equidistant grids can be improved by substituting the linear interpolation of pressure gradients by simple averaging in equation (\ref{eq:faceinter}). This leads to the following equation for calculating mass fluxes
  \begin{align}
    \label{eq:facecorr2}
    u_{i,f}^{(n)} 
    &=
    \left[\left(1 - \gamma_f\right) u_{i,P}^{(n)} + \gamma_f u_{i,Q}^{(n)} \right] \nonumber\\[1em]
    &\quad\quad - 
    \left(\left(1 - \gamma_f\right) \frac{\alpha_\vec{u} V_P}{a_{P,u_i}} + \gamma_f \frac{\alpha_\vec{u} V_Q}{a_{Q,u_i}}\right)
    \left[ 
    \left(\frac{\partial p}{\partial x_i}\right)_f^{(n)} 
    -  \frac{1}{2} \left( \left( \frac{\partial p}{\partial x_i} \right)_P^{(n)} 
    +  \left(\frac{\partial p}{\partial x_i}\right)_Q^{(n)} \right)
    \right] \nonumber \\[1em]
    &\quad\quad + \left(1 - \alpha_\vec{u}\right) \left[ u_{i,f}^{(n-1)} - \left(1 - \gamma_f\right) u_{i,P}^{(n-1)} - \gamma_f \, u_{i,Q}^{(n-1)} \right].
  \end{align}



  \subsection{Discretization of the Momentum Balance}
  \label{sec:segdiscretization}

      The stationary momentum balance integrated over a single control volume \(P\) reads as
      \begin{equation}
        \label{eq:semidiscrete}
        \underbrace{\iint\limits_S (\rho u_i u_j)n_j \mathrm{d}S}_{\text{convective term}}
        - \underbrace{\iint\limits_S \left(\mu \left( \frac{\partial u_i}{\partial x_j} + \frac{\partial u_j}{\partial x_i}\right)\right)n_j \mathrm{d}S}_{\text{diffusive term}}
        = - \underbrace{\iiint\limits_V \frac{\partial p}{\partial x_i} \mathrm{d}V}_{\text{sourceterm pressure}}
        - \underbrace{\iiint\limits_V \rho \beta \left(T - T_0\right) \mathrm{d}V}_{\text{sourceterm temperature}}
      \end{equation}
      where the different terms to be addressed individually in the following sections are indicated. Note that the form of this equation has been modified by using Gauss' integration theorem. The terms residing on the left will be treated in an implicit and due to deferred corrections in an explicit manner whereas the terms on the right will be treated exclusively in an  explicit manner.

      \subsubsection{Linearization and Discretization of the Convective Term}

      The convective term \(\rho u_i u_j\) of the Navier-Stokes equations is the reason for the non-linearity of the equations. In order to deduce a set of linear algebraic equations from the Navier-Stokes equations this term has to be linearized. As introduced in section (\ref{sec:nonlinear}), the non linearity will be dealt with by means of an iterative process, the Picard iteration. The part dependent on the non dominant dependent variable therefore will be approximated by its value from the previous iteration as \( \rho u_i^{(n)} u_j^{(n)} \approx \rho u_i^{(n)} u_j^{(n-1)} \). However this linearization will not be directly visible because it will be covered by the mass flux \(\dot{m}_f = \iint\limits_{S_f} \rho u_j^{(n-1)} n_j \mathrm{d}S \). Using the additivity of the Riemann integral the first step is to decompose the surface integral into individual contributions from each boundary face of the control volume \(P\)
      \begin{displaymath}
      \iint\limits_S \rho u_i u_jn_j \mathrm{d}S
      = \sum_{f \in \{w,s,b,t,n,e\}} \iint\limits_{S_f}\rho u_{i} u_{j} n_{j} \mathrm{d}S
      = \sum_{f \in \{w,s,b,t,n,e\}} F_{i,f}^{c}
      \end{displaymath}
      where \(F_{i,f}^c := \iint\limits_{S_f} \rho u_{i}^{(n)} u_{j}^{(n-1)} n_{j} \mathrm{d}S \) is the convective flux of the velocity \(u_i\) through the face \(S_f\). 
      
      To improve diagonal dominance of the resulting linear system while maintaining the smaller discretization error of a higher order discretization, a blended discretization scheme is applied using a deferred correction. Since due to the non-linearity of the equations to be solved an iterative solution process is needed by all means, the overall convergence doesn't degrade noticeably when using a deferred correction. Blending and deferred correction result in a decomposition of the convective flux into a lower order approximation that is treated implicitly and the explicit difference between the higher and lower order approximation for the same convective flux. Since for coarse grid resolutions the use of higher order approximations may lead to oscillations of the solution which may degrade or even impede convergence, the schemes can be blended by a control factor \( \eta \in [0,1]\). To show the generality of this approach all further derivations are presented for the generic boundary face \(S_f\) that separates control volume \(P\) from its neighbour \(F \in NB(P)\). This decomposition then leads to
      \begin{displaymath}
        F_{i,f}^c \approx  \underbrace{F_{i,f}^{c,l}}_{\text{implicit}} + \eta \, \bigl[\underbrace{ F_{i,f}^{c,h} - F_{i,f}^{c,l} }_{\text{explicit}}\bigr]^{(n-1)}.
      \end{displaymath}
      Note that the convective fluxes carrying a \(l\) or \(h\) as exponent, already have been linearized and discretized. The discretization applied to the convective flux in the present work is using the midpoint integration rule and blends the upwind interpolation scheme with the linear interpolation scheme. Applied to above decomposition one can derive the following approximations
      \begin{align*}
        F_{i,f}^{c,l} &= u_{i,F} \min(\dot{m}_f ,0) + u_{i,P} \max(0,\dot{m}_f) \\
        F_{i,f}^{c,h} &= u_{i,F} \, \gamma_f + u_{i,P} \, (1 - \gamma_f),
      \end{align*}
      where the variable values have to be taken from the previous iteration step \((n-1)\) as necessary and the mass flux \(\dot{m}_f\) has been used as result of the linearization process. The results can now be summarized by presenting the convective contribution to the matrix coefficients \(a_{F,u_i}\) and \(a_{P,u_i}\) and the right hand side \(b_{P,u_i}\) which are calculated as
      \begin{subequations}
      \begin{align}
        a_{F,u_i}^c &= \min(\dot{m}_f ,0), \quad \quad a_{P,u_i}^c = \sum_{F \in NB(P)} \max(0,\dot{m}_f) \\[1em]
        b_{P,u_i}^c &= \sum_{F \in NB(P)} \eta  \left(u_{i,F}^{(n-1)} \left( \min(\dot{m}_f,0) - \gamma_f \right)\right) \nonumber \\
                    &\quad \quad \quad  \quad+ \eta \left( u_{i,P}^{(n-1)} \left( \max(0,\dot{m}_f) - \left(1 - \gamma_f\right) \right)\right)
      \end{align}
    \end{subequations}

      \subsubsection{Discretization of the Diffusive Term}

      The diffusive term contains the first partial derivatives of the velocity as result of the material constitutive equation that characterizes the behaviour of Newtonian fluids. As pointed out in section \ref{sec:nonorth} directional derivatives can be discretized using central differences on orthogonal grids or in the more general case of non-orthogonal grids using central differences implicitly and a explicit deferred correction comprising the non-orthogonality of the grid. As seen in equation (\ref{eq:navierstokes}) the diffusive term of the Navier-Stokes equations can be simplified using the mass balance in the case of an incompressible flow with constant viscosity \(\mu\). To sustain the generality of the presented approach this simplification will be omitted.

      As before, by using the additivity and furthermore linearity of the Riemann integral, the integration of the diffusive term will be divided into integration over individual boundary faces \(S_f\) 
      \begin{displaymath}
      \iint\limits_S \left(\mu \left( \frac{\partial u_i}{\partial x_j} + \frac{\partial u_j}{\partial x_i}\right)\right)n_j \mathrm{d}S \mathrm{d}S
      = \sum_{f \in \{w,s,b,t,n,e\}} \left[
        \iint\limits_{S_f} \mu \underline{\frac{\partial u_i}{\partial x_j}n_j \mathrm{d}S}
    + \iint\limits_{S_f} \mu \frac{\partial u_j}{\partial x_i}n_j \mathrm{d}S \right]
       = \sum_{f \in \{w,s,b,t,n,e\}} F_{i,f}^{d},
      \end{displaymath}
      where \(F_{i,f}^{d}\) denotes the diffusive flux through an individual boundary face. Section \ref{sec:nonorth} only covered the non-orthogonal corrector for directional derivatives. Since the velocity is a vector field and not a scalar field, the results of section \ref{sec:nonorth} may only be applied to the underlined term. The other term will be treated explicitly since it is considerably smaller than the underlined term and does not cause oscillations and thus will not derogate convergence. To begin with all present integrals will be approximated using the midpoint rule of integration. The diffusive flux \(F_{i,f}^d\) for a generic face \(S_f\) then reads 
      \begin{displaymath}
        F_{i,f}^d \approx \mu \underline{\left(\frac{\partial u_i}{\partial x_j}\right)_f n_j S_f} + \mu \left(\frac{\partial u_j}{\partial x_i}\right)_f n_j S_f.
      \end{displaymath}

      Using central differences for the implicit discretization of the directional derivative and furthermore using the \emph{orthogonal correction} approach from \ref{seq:orthcorrapproach} the approximation can be derived as
      \begin{align*}
        F_{i,f}^d 
        &\approx 
        \mu \left( \underline{||\vecg{\vecg{\Delta}_f}||_2 \frac{u_{P_i} - u_{F_i}}{ || \vec{x}_P - \vec{x}_F ||_2 }  
        -  \left(\nabla u_i \right)_f^{(n-1)} \cdot \left(\vecg{\Delta}_f - \vec{S}_f\right)  }  \right)
        + \mu \left( \frac{\partial u_j}{\partial x_i} \right)_f^{(n-1)} n_{f_i} \\[1em]
        &= \mu \left(\underline{  S_f \frac{u_{P_i} - u_{F_i}}{ || \vec{x}_P - \vec{x}_F ||_2 }  
    - \left( \frac{\partial u_i}{\partial x_j}\right)_f^{(n-1)} \left(\xi_{f_i} - n_{f_i}\right)S_f  } \right)
      + \mu \left( \frac{\partial u_j}{\partial x_i} \right)_f^{(n-1)} n_{f_i},
    \end{align*}
      where the unit vector pointing in direction of the straight line connecting control volume \(P\) and control volume \(F\) is denoted as
      \begin{displaymath}
        \vecg{\xi}_f = \frac{\vec{x}_P - \vec{x}_F}{|| \vec{x}_p - \vec{x}_F ||_2}.
      \end{displaymath}
      The interpolation of the cell center gradients to the boundary faces is performed as in (\ref{eq:interpolgrad}). Now the contribution of the diffusive part to the matrix coefficients and the right hand side can be calculated as
      \begin{subequations}
        \begin{align}
          a_{F,u_i}^d &= - \frac{\mu S_f}{||\vec{x}_P - \vec{x}_F||_2}, 
          \quad \quad a_{P,u_i}^d = \sum_{F \in NB(P)} \frac{\mu S_f}{|| \vec{x}_P - \vec{x}_F ||} \\[1em]
          b_{F,u_i}^d &=  \sum_{F \in NB(P)} \left( \frac{\partial u_i}{\partial x_j}\right)_f^{(n-1)} \left(\xi_{f_i} - n_{f_i}\right)S_f  
          - \mu \left( \frac{\partial u_j}{\partial x_i} \right)_f^{(n-1)} n_{f_i} S_f   \nonumber \\[0.5em]
          &=   \left( \frac{\partial u_i}{\partial x_j}\right)_f^{(n-1)} \xi_{f_i} S_f
          - \mu \left( \left( \frac{\partial u_i}{\partial x_j} \right)_f^{(n-1)}
          - \left( \frac{\partial u_j}{\partial x_i} \right)_f^{(n-1)} \right) n_{f_i} S_f.
        \end{align}
      \end{subequations}

      \subsubsection{Discretization of the Source Terms}

      Since in the segregated solution approach in every equation all other variables but the dominant one are treated as constants and furthermore the source terms in equation (\ref{eq:semidiscrete}) do not depend on the dominant variable the discretization is straightforward. The source terms of the momentum balance are discretized using the midpoint rule of integration, which leads to the source term
      \begin{equation}
        - \iiint\limits_V \frac{\partial p}{\partial x_i} \mathrm{d}V
        - \iiint\limits_V \rho \beta \left(T - T_0\right) \mathrm{d}V
        \approx
        - \left(\frac{\partial p}{\partial x_i}\right)_P^{(n-1)} V_P
        - \rho \beta \left(T_P^{(n-1)} - T_0\right) V_P
        = b_{P,u_i}^{sc}
      \end{equation}

  \subsection{Discretization of the Pressure Correction Equation}

  \subsection{Discretization of the Temperature Equation}
  \label{sec:discretetemperature}

    The discretization of the temperature equation is performed by the same means as for the momentum balance. The only difference is a simpler diffusion term. The integral form of the temperature equation after applying the Gauss' theorem of integration is
    \begin{displaymath}
    \underbrace{ \iint\limits_S \rho u_j T n_j \mathrm{d}S }_{\text{advective term}}
    - \underbrace{ \iint\limits_S \kappa \frac{\partial T}{\partial x_j} n_j \mathrm{d}V }_{\text{diffusive term}}
    = \underbrace{\iiint\limits_V \vphantom{\frac{ T}{ x_j} } q_T \mathrm{d}V }_{\text{source term}}.
    \end{displaymath}
    Proceeding as in the previous subsections one can now discretize the advective, the diffusive term and the source term. Since this process does not provide further insight, just the final results will be presented. The discretization yields the matrix coefficients as
    \begin{subequations}
      \begin{align}
        a_{F,T} &= \min(\dot{m}_f,0) + \frac{\kappa S_f}{||\vec{x}_P - \vec{x}_F||_2} \\[1em]
        a_{P,T} &= \sum_{F \in NB(P)}\max(0,\dot{m}_f) - \frac{\kappa S_f}{||\vec{x}_P - \vec{x}_F||_2} \\[1em]
        b_{P,T} &= \sum_{F \in NB(P)} \eta  \left(T_F^{(n-1)} \left( \min(\dot{m}_f,0) - \gamma_f \right)\right) \nonumber \\
                &\quad \quad \quad  \quad+ \eta \left( T_{P}^{(n-1)} \left( \max(0,\dot{m}_f) - \left(1 - \gamma_f\right) \right)\right) \nonumber \\[0.5em]
                &\quad + \sum_{F \in NB(P)} \left( \frac{\partial T}{\partial x_j}\right)_f^{(n-1)} \left(\xi_{f_j} - n_{f_j}\right)S_f \nonumber \\[0.5em]
                &\quad + q_{T_P} V_P.
      \end{align}
    \end{subequations}
    Again it is possible though not always necessary, as in the case of the velocities, to under-relax the solution of the resulting linear system with a factor \(\alpha_T\). This can be accomplished as shown in the previous sections.


  \subsection{Boundary Conditions}

  \subsection{Structure of the Assembled Linear Systems}

    The objective of a finite volume method is to create a set of linear algebraic equations by discretizing partial differential equations. For the momentum balance all necessary components have been calculated. Taking all contributions together leads to the following linear algebraic equation for each control volume \(P\)
    \begin{displaymath}
      a_{P,u_i} u_{P_i} + \sum_{F \in NB(P)} a_F u_{F_i} = b_{P,u_i},
    \end{displaymath}
    where the coefficients are composed as
    \begin{align}
      a_{P,u_i} &= a_{P,u_i}^c - a_{P,u_i}^d \\
      a_{F,u_i} &= a_{F,u_i}^c - a_{F,u_i}^d \\
      b_{P,u_i} &= b_{P,u_i}^c - b_{P,u_i}^d + b_{P,u_i}^{sc}.
    \end{align}
    In the case of control volumes located at boundaries some of the coefficients will be calculated in a different manner. This aspect will be addressed in a later section. For the decoupled iterative solution process of the Navier-Stokes equations it is necessary to reduce the change of each dependent variable in each iteration. Normally this is done by a \emph{under-relaxation} technique, a convex combination of the solution of the linear system present iteration \((n)\) and from the previous iteration \((n-1)\) with the under-relaxation parameter \(\alpha_{u_i}\). Generally speaking this parameter can be chosen individually for each equation. Since there are no rules for choosing this parameters in a general setting the under-relaxation parameter for the velocities is chosen to be equal for all three velocities, \(\alpha_{u_i} = \alpha_{\vec{u}}\). This has the further advantage that, in case the boundary conditions are implemented with the same intention, the linear system for each of the velocities remains unchanged except for the right hand side. This helps to increase memory efficiency.

    Let  the solution for the linear system without under-relaxation be denoted as
    \begin{displaymath}
      \tilde{u}_{P_i}^{(n)} := \frac{b_{P,u_i} - \sum_{F \in NB(P)} a_F u_{F_i}}{a_{P,u_i}},
    \end{displaymath}
    Which is only a formal expression. A convex combination as described yields
    \begin{align*}
      u_{P_i}^{(n)} :&= \alpha_{\vec{u}} \tilde{u}_{P_i}^{(n)} + (1 - \alpha_{\vec{u}} )\, u_{P_i}^{(n-1)} \\[0.5em]
                     &= \alpha_{\vec{u}} \frac{b_{P,u_i} - \sum_{F \in NB(P)} a_F u_{F_i}}{a_{P,u_i}} + (1 - \alpha_{\vec{u}} )\, u_{P_i}^{(n-1)},
    \end{align*}
    an expression that can be modified to derive a linear system whose solution is the under-relaxed velocity
    \begin{displaymath}
      \frac{a_{P,u_i}}{\alpha_{\vec{u}}} u_{i,P} + \sum_{F \in NB(P)} a_{F,u_i} u_{i,F} 
      = 
      b_{P,u_i} + \frac{(1 - \alpha_{\vec{u}})\, a_{P,u_i}}{\alpha_{\vec{u}}}\, u_{i,P}^{(n-1)}. 
    \end{displaymath}
    It must be noted that under-relaxation has not been accounted for in the derivation of the Rhie-Chow interpolation method (REFERENCE). As it has been shown the results depend on the choice of the under-relaxation factor. However under-relaxation is necessary for overall convergence, so in order to be able to compare the results of the different solver algorithms this dependency has to be eliminated. Subsection (REFERENCE) will present a common approach to resolve this dependency.

