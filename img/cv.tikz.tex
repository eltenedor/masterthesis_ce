    \begin{tikzpicture}
	%%% Edit the following coordinate to change the shape of your
	%%% cuboid
      
	%% Vanishing points for perspective handling
	\coordinate (P1) at (-25cm,6.5cm); % left vanishing point (To pick)
	\coordinate (P2) at (18cm,5.5cm); % right vanishing point (To pick)

	%% (A1) and (A2) defines the 2 central points of the cuboid
	\coordinate (A1) at (0em,2.0cm); % central top point (To pick)
	\coordinate (A2) at (0em,0.0cm); % central bottom point (To pick)

	%% (A3) to (A8) are computed given a unique parameter (or 2) .8
	% You can vary .8 from 0 to 1 to change perspective on left side
	\coordinate (A3) at ($(P1)!.9!(A2)$); % To pick for perspective 
	\coordinate (A4) at ($(P1)!.9!(A1)$);

	% You can vary .8 from 0 to 1 to change perspective on right side
	\coordinate (A7) at ($(P2)!.92!(A2)$);
	\coordinate (A8) at ($(P2)!.92!(A1)$);

	%% Automatically compute the last 2 points with intersections
	\coordinate (A5) at
	  (intersection cs: first line={(A8) -- (P1)},
			    second line={(A4) -- (P2)});
	\coordinate (A6) at
	  (intersection cs: first line={(A7) -- (P1)}, 
			    second line={(A3) -- (P2)});

	%%% Depending of what you want to display, you can comment/edit
	%%% the following lines

	%% Possibly draw back faces

	\fill[gray!90] (A2) -- (A3) -- (A6) -- (A7) -- cycle; % face 6
        \node (C6) at (barycentric cs:A2=1,A3=1,A6=1,A7=1) { $S_b$};
	
	\fill[gray!50] (A3) -- (A4) -- (A5) -- (A6) -- cycle; % face 3
        \node (C3) at (barycentric cs:A3=1,A4=1,A5=1,A6=1) { $S_w$};
	
	\fill[gray!30] (A5) -- (A6) -- (A7) -- (A8) -- cycle; % face 4
        \node (C4) at (barycentric cs:A5=1,A6=1,A7=1,A8=1) { $S_n$};
	
	\draw[thick,dashed] (A5) -- (A6);
	\draw[thick,dashed] (A3) -- (A6);
	\draw[thick,dashed] (A7) -- (A6);

	%% Possibly draw front faces

	% \fill[orange] (A1) -- (A8) -- (A7) -- (A2) -- cycle; % face 1
        \node (C1) at (barycentric cs:A1=1,A8=1,A7=1,A2=1) { $S_e$};
	\fill[gray!50,opacity=0.2] (A1) -- (A2) -- (A3) -- (A4) -- cycle; % f2
        \node (C2) at (barycentric cs:A1=1,A2=1,A3=1,A4=1) { $S_s$};
	\fill[gray!90,opacity=0.2] (A1) -- (A4) -- (A5) -- (A8) -- cycle; % f5
        \node (C5) at (barycentric cs:A1=1,A4=1,A5=1,A8=1) { $S_t$};

        %construct center points
	\coordinate (D1) at ($(C3)!.5!(C1)$);
        \draw[fill=black] (D1) circle (0.15em)
          node[above right]{P} ;

        %east west
	\coordinate (D2) at ($(C3)!1.5!(C1)$);
        \draw[fill=black] (D2) circle (0.15em)
          node[above right]{W} ;
	\coordinate (D3) at ($(C3)!-.5!(C1)$);
        \draw[fill=black] (D3) circle (0.15em)
          node[above right]{E} ;

        %north south
	\coordinate (D2) at ($(C2)!1.5!(C4)$);
        \draw[fill=black] (D2) circle (0.15em)
          node[above right]{N} ;
	\coordinate (D3) at ($(C2)!-.5!(C4)$);
        \draw[fill=black] (D3) circle (0.15em)
          node[above right]{S} ;

        %bottom top
	\coordinate (D4) at ($(C5)!1.5!(C6)$);
        \draw[fill=black] (D4) circle (0.15em)
          node[above right]{B} ;
	\coordinate (D5) at ($(C5)!-.5!(C6)$);
        \draw[fill=black] (D5) circle (0.15em)
          node[above right]{T} ;

	%% Possibly draw front lines
	\draw[thick] (A1) -- (A2);
	\draw[thick] (A3) -- (A4);
	\draw[thick] (A7) -- (A8);
	\draw[thick] (A1) -- (A4);
	\draw[thick] (A1) -- (A8);
	\draw[thick] (A2) -- (A3);
	\draw[thick] (A2) -- (A7);
	\draw[thick] (A4) -- (A5);
	\draw[thick] (A8) -- (A5);

        %draw help lines for neighbouring cvs
        % north south
        \coordinate (N1) at ($(A1)!-0.8!(A8)$);
        \coordinate (N2) at ($(A2)!-0.8!(A7)$);
        \coordinate (N3) at ($(A3)!-0.8!(A6)$);
        \coordinate (N4) at ($(A4)!-0.8!(A5)$);

        \draw[gray,dashed] (A1) -- (N1);
        \draw[gray,dashed] (A2) -- (N2);
        \draw[gray,dashed] (A3) -- (N3);
        \draw[gray,dashed] (A4) -- (N4);

        \coordinate (N8) at ($(A1)!1.8!(A8)$);
        \coordinate (N7) at ($(A2)!1.8!(A7)$);
        \coordinate (N6) at ($(A3)!1.8!(A6)$);
        \coordinate (N5) at ($(A4)!1.8!(A5)$);

        \draw[gray,dashed] (A8) -- (N8);
        \draw[gray,dashed] (A7) -- (N7);
        \draw[gray,dashed] (A6) -- (N6);
        \draw[gray,dashed] (A5) -- (N5);

        %east west
        \coordinate (N33) at ($(A2)!1.4!(A3)$);
        \coordinate (N44) at ($(A1)!1.4!(A4)$);
        \coordinate (N55) at ($(A8)!1.4!(A5)$);
        \coordinate (N66) at ($(A7)!1.4!(A6)$);

        \draw[gray,dashed] (A3) -- (N33);
        \draw[gray,dashed] (A4) -- (N44);
        \draw[gray,dashed] (A5) -- (N55);
        \draw[gray,dashed] (A6) -- (N66);

        \coordinate (N22) at ($(A2)!-0.4!(A3)$);
        \coordinate (N11) at ($(A1)!-0.4!(A4)$);
        \coordinate (N88) at ($(A8)!-0.4!(A5)$);
        \coordinate (N77) at ($(A7)!-0.4!(A6)$);

        \draw[gray,dashed] (A1) -- (N11);
        \draw[gray,dashed] (A2) -- (N22);
        \draw[gray,dashed] (A7) -- (N77);
        \draw[gray,dashed] (A8) -- (N88);

        %bottom top
        \coordinate (N333) at ($(A4)!1.4!(A3)$);
        \coordinate (N222) at ($(A1)!1.4!(A2)$);
        \coordinate (N777) at ($(A8)!1.4!(A7)$);
        \coordinate (N666) at ($(A5)!1.4!(A6)$);

        \draw[gray,dashed] (A3) -- (N333);
        \draw[gray,dashed] (A2) -- (N222);
        \draw[gray,dashed] (A7) -- (N777);
        \draw[gray,dashed] (A6) -- (N666);

        \coordinate (N444) at ($(A4)!-0.8!(A3)$);
        \coordinate (N111) at ($(A1)!-0.8!(A2)$);
        \coordinate (N888) at ($(A8)!-0.8!(A7)$);
        \coordinate (N555) at ($(A5)!-0.8!(A6)$);

        \draw[gray,dashed] (A4) -- (N444);
        \draw[gray,dashed] (A1) -- (N111);
        \draw[gray,dashed] (A8) -- (N888);
        \draw[gray,dashed] (A5) -- (N555);
	
	% Possibly draw points
	% (it can help you understand the cuboid structure)
	%\foreach \i in {1,2,...,8}
	%{
	%  \draw[fill=black] (A\i) circle (0.15em)
	%}
	%\draw[fill=black] (P1) circle (0.1em) node[below] {\tiny p1};
	%\draw[fill=black] (P2) circle (0.1em) node[below] {\tiny p2};
\end{tikzpicture}
